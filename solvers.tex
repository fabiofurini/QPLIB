\subsubsection{Solvers}\label{subsec:solver}

When compiling QPLIB, we have worked with QP solvers that come with the GAMS distribution\footnote{\url{https://www.gams.com}}.
A list of these solvers is provided by Table \ref{t:solvers}.

We mark a pair (solver, problem) with ``I'' if the solver accepts the problem in input but it is an incomplete solver for the problem, with ``A'' if it is asymptotically complete, with ``C'' if it is complete, and leave it blank if the solver won't accept the problem in input.
When a solver implements several algorithms, we have chosen, for each problem class, the algorithm that potentially provides the ``strongest'' results (``C'' $>$ ``A'' $>$ ``I'' $>$ blank).

\begin{table}
{
 \centering                
 \scriptsize                
 \setlength{\tabcolsep}{3pt}                
% \renewcommand \arraystretch{1}                
                
\begin{tabular}{lccccccc}
\toprule  
             & CGL & QGL & CGC & QGQ & CCC & QCQ \\
\hline
\alphaecp    &  C  &  I  &  C  &  I  &  C  &  I  \\
\antigone    &  C  &  C  &  C  &  C  &  C  &  C  \\
\baron       &  C  &  C  &  C  &  C  &  C  &  C  \\
\bonmin      &  C  &  I  &  C  &  I  &  C  &  I  \\
\conopt      &     &     &     &     &  C  &  I  \\
\couenne     &  C  &  C  &  C  &  C  &  C  &  C  \\
\cplex       &  C  &  C  &  C  &     &  C  &     \\
\dicopt      &  C  &  I  &  C  &  I  &  C  &  I  \\
\gurobi      &  C  &     &  C  &     &  C  &     \\
\ipopt       &     &     &     &     &  C  &  I  \\
\knitro      &  C  &  I  &  C  &  I  &  C  &  A  \\
\lindo       &  C  &  C  &  C  &  C  &  C  &  C  \\
\lgo         &     &     &     &     &  A  &  A  \\
\minos       &     &     &     &     &  C  &  I  \\
\mosek       &  C  &     &  C  &     &  C  &     \\
\msnlp       &     &     &     &     &  C  &  A  \\
\oqnlp       &  A  &  A  &  A  &  A  &  C  &  A  \\
\sbb         &  C  &  I  &  C  &  I  &  C  &  I  \\
\scip        &  C  &  C  &  C  &  C  &  C  &  C  \\
\snopt       &     &     &     &     &  C  &  I  \\
\xpress      &  C  &     &  C  &     &  C  &     \\
\hline
\end{tabular}                 
\caption{Families of QP problems that can be tackled by each solver} \label{t:solvers}
}

\end{table}              

% antigone/glomiqo: sBB
% baron: sBB
% couenne: sBB
% knitro: IPM with CG, SLQP, convexBB + outerapproxBB
% lindo/lindoglobal: sBB
% scip: sBB
% msnlp/oqnlp: multistart
% alphaecp: cutting plane for pseudoconvex MINLP
% bonmin: convexBB
% dicopt: alternating projection NLP/MILP
% sbb: convexBB
% conopt: GRG (active set -- like SQP)
% ipopt: IPM w/ line search filter
% lgo: Lipschitz BB, random search, multistart (default) [cassato?]
% minos: active set method
% snopt: active set SQP
% xpress: convexBB
% gurobi: convexBB
% cplex: sBB
