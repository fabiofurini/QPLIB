\subsubsection{Solvers}\label{subsec:solver}

When compiling QPLIB, we have worked with the QP solvers in the GAMS distribution\footnote{\url{https://www.gams.com}}.
Table \ref{t:solvers} provides a list of these solvers, together with a classification of their algorithm, and references.
For more details on the solvers, we refer to the given references, solver manuals, and the survey \cite{BuVi10}.
In the table, we mark a pair (solver, problem) with ``I'' if the solver accepts the problem as input but it is an incomplete solver for the problem, with ``A'' if it is asymptotically complete, with ``C'' if it is complete, and leave it blank if the solver won't accept the provided problem.
When a solver implements several algorithms, we have chosen, for each problem class, the algorithm that potentially provides the ``strongest'' results (``C'' $>$ ``A'' $>$ ``I'' $>$ blank).

\begin{table}
 \centering
 \scriptsize
 \setlength{\tabcolsep}{3pt}

\begin{tabular}{llccccccc}
\toprule
             & & CGL & QGL & CGC & QGQ & CCC & QCQ \\
\midrule
\alphaecp    & \cite{WeLu03,WePoe02} &  C  &  I  &  C  &  I  &  C  &  I  \\
\antigone    & \cite{MiFl13,MiFl14} &  C  &  C  &  C  &  C  &  C  &  C  \\
\baron       & \cite{TaSa02,TaSa04,TaSa05} &  C  &  C  &  C  &  C  &  C  &  C  \\
\bonmin      & \cite{BBCCGLLLMSW} &  C  &  I  &  C  &  I  &  C  &  I  \\
\conopt      & \cite{Drud1985,Drud1994} &     &     &     &     &  C  &  I  \\
\couenne     & \cite{BeLeLiMaWa08} &  C  &  C  &  C  &  C  &  C  &  C  \\
\cplex       & \cite{Bixby2000,cplex} &  C  &  C  &  C  &     &  C  &     \\
\dicopt      & \cite{DuGr86,KoGr89,ViGr90} &  C  &  I  &  C  &  I  &  C  &  I  \\
\gurobi      & \cite{Ro12} &  C  &     &  C  &     &  C  &     \\
\ipopt       & \cite{WaBi06} &     &     &     &     &  C  &  I  \\
\knitro      & \cite{ByNoWa06} &  C  &  I  &  C  &  I  &  C  &  A  \\
\lindo       & \cite{LiSc09} &  C  &  C  &  C  &  C  &  C  &  C  \\
\lgo         & \cite{Pinter1997,Pinter1998} &     &     &     &     &  A  &  A  \\
\minos       & \cite{murtagh78,murtagh82} &     &     &     &     &  C  &  I  \\
\mosek       & \cite{Andersen2000,Andersen2003} &  C  &     &  C  &     &  C  &     \\
\msnlp       & \cite{ULPGKM02,LasdonPlummerUgrayBussieck2006} &     &     &     &     &  C  &  A  \\
\oqnlp       & \cite{ULPGKM02,LasdonPlummerUgrayBussieck2006} &  A  &  A  &  A  &  A  &  C  &  A  \\
\sbb         & \cite{sbb} &  C  &  I  &  C  &  I  &  C  &  I  \\
\scip        & \cite{Ach09,VigerskeGleixner2017} &  C  &  C  &  C  &  C  &  C  &  C  \\
\snopt       & \cite{GMS02,GMS05} &     &     &     &     &  C  &  I  \\
\xpress      & \cite{xpress} &  C  &     &  C  &     &  C  &     \\
\bottomrule
\end{tabular}
\caption{Families of QP problems that can be tackled by each solver} \label{t:solvers}

\end{table}

% antigone/glomiqo: sBB
% baron: sBB
% couenne: sBB
% knitro: IPM with CG, SLQP, convexBB + outerapproxBB
% lindo/lindoglobal: sBB
% scip: sBB
% msnlp/oqnlp: multistart
% alphaecp: cutting plane for pseudoconvex MINLP
% bonmin: convexBB
% dicopt: alternating projection NLP/MILP
% sbb: convexBB
% conopt: GRG (active set -- like SQP)
% ipopt: IPM w/ line search filter
% lgo: Lipschitz BB, random search, multistart (default) [cassato?]
% minos: active set method
% snopt: active set SQP
% xpress: convexBB
% gurobi: convexBB
% cplex: sBB
