%- - - - - - - - - - - - - - - - - - - - - - - - - - - - - - - - - - - -
%- - - - - - - - - - - - - - - - - - - - - - - - - - - - - - - - - - - -
%  QPLIB-2.4.tex
%- - - - - - - - - - - - - - - - - - - - - - - - - - - - - - - - - - - -
%- - - - - - - - - - - - - - - - - - - - - - - - - - - - - - - - - - - -
\subsection{Solvers}\label{subsec:solver}

We now provide a succinct list of the solvers we have worked with. In Table \ref{t:solvers}, we mark with ``I'' a pair (solver, problem) if the solver accepts the problem in input but it is an incomplete solver for the problem, with ``A'' if it is asymptotically complete, with ``C'' if it is complete, and leave it blank if the solver won't accept the problem in input.

\begin{table}
{
 \centering                
 \scriptsize                
 \setlength{\tabcolsep}{3pt}                
% \renewcommand \arraystretch{1}                
                
\begin{tabular}{lccccccc}
\toprule  
             & CGL & QGL & CGC & QGQ & CCC & QCQ \\
\hline
\alphaecp    &  C  &  I  &  C  &  I  &  C  &  I  \\
\antigone    &  C  &  C  &  C  &  C  &  C  &  C  \\
\baron       &  C  &  C  &  C  &  C  &  C  &  C  \\
\bonmin      &  C  &  I  &  C  &  I  &  C  &  I  \\
\conopt      &     &     &     &     &  C  &  I  \\
\couenne     &  C  &  C  &  C  &  C  &  C  &  C  \\
\cplex       &  C  &  C  &  C  &     &  C  &     \\
\dicopt      &  C  &  I  &  C  &  I  &  C  &  I  \\
\gurobi      &  C  &     &  C  &     &  C  &     \\
\ipopt       &     &     &     &     &  C  &  I  \\
\knitro      &  C  &  I  &  C  &  I  &  C  &  A  \\
\lindo       &  C  &  C  &  C  &  C  &  C  &  C  \\
\lgo         &     &     &     &     &  A  &  A  \\
\minos       &     &     &     &     &  C  &  I  \\
\mosek       &  C  &     &  C  &     &  C  &     \\
\msnlp       &     &     &     &     &  C  &  A  \\
\oqnlp       &  A  &  A  &  A  &  A  &  C  &  A  \\
\sbb         &  C  &  I  &  C  &  I  &  C  &  I  \\
\scip        &  C  &  C  &  C  &  C  &  C  &  C  \\
\snopt       &     &     &     &     &  C  &  I  \\
\xpress      &  C  &     &  C  &     &  C  &     \\
\hline
\end{tabular}                 
\caption{Families of QP problems that can be tackled by each solver} \label{t:solvers}
}

\end{table}              

%- - - - - - - - - - - - - - - - - - - - - - - - - - - - - - - - - - - -
%- - - - - - - - - - - - - - - - - - - - - - - - - - - - - - - - - - - -
%  End QPLIB-2.4.tex
%- - - - - - - - - - - - - - - - - - - - - - - - - - - - - - - - - - - -
%- - - - - - - - - - - - - - - - - - - - - - - - - - - - - - - - - - - -
