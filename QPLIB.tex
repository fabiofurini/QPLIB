%- - - - - - - - - - - - - - - - - - - - - - - - - - - - - - - - - - - -
%- - - - - - - - - - - - - - - - - - - - - - - - - - - - - - - - - - - -
%  QPLIB.tex
%- - - - - - - - - - - - - - - - - - - - - - - - - - - - - - - - - - - -
%- - - - - - - - - - - - - - - - - - - - - - - - - - - - - - - - - - - -

\newif\ifMPC
\global\MPCtrue

%- - - - - - - - - - - - - - - - - - - - - - - - - - - - - - - - - - - -

\ifMPC
 \documentclass[smallextended]{svjour3}
 \smartqed
 \journalname{Mathematical Programming Computation} 
\else
 \documentclass[12pt,A4wide]{article}
\fi

\usepackage{url}
\usepackage{secdot}
\usepackage{color}
\usepackage{booktabs}
\usepackage{amsmath,amsfonts,amssymb}
\usepackage{amssymb}
\usepackage{amsmath}
\usepackage{multirow}

%- - - - - - - - - - - - - - - - - - - - - - - - - - - - - - - - - - - -
% page size

\ifMPC
\else
 \setlength{\textwidth}{16 cm}
 \setlength{\topmargin}{0 cm}
 \setlength{\headsep}{0 cm}
 \setlength{\textheight}{24 cm}
 \setlength{\evensidemargin}{0cm}
 \setlength{\oddsidemargin}{0cm}
\fi

%- - - - - - - - - - - - - - - - - - - - - - - - - - - - - - - - - - - -
%- - - - - - - - - - - - - - - - - - - - - - - - - - - - - - - - - - - -

\def\NN{\mathbb{N}}
\def\RR{\mathbb{R}}
\def\Mcq{\mathcal{Q}}
\def\Mcl{\mathcal{L}}
\def\Mcm{\mathcal{M}}
\def\Mcn{\mathcal{N}}
\def\Mcb{\mathcal{B}}
\def\Mcr{\mathcal{R}}
\def\Mcu{\mathcal{U}}
\def\Mcx{\mathcal{X}}
\def\Mcz{\mathcal{Z}}

%- - - - - - - - - - - - - - - - - - - - - - - - - - - - - - - - - - - -
% Theorem-like environment definitions

\ifMPC
\else
\newtheorem{theorem}{Theorem}
\newtheorem{assumption}[theorem]{Assumption}
\newtheorem{corollary}[theorem]{Corollary}
\newtheorem{definition}[theorem]{Definition}
\newtheorem{lemma}[theorem]{Lemma}
\newtheorem{remark}{Remark}

\newenvironment{example}{\begin{EX}\rm}{\end{EX}}

\newenvironment{proof}[1][Proof]{\textbf{#1.} }{\ \rule{0.5em}{0.5em}}
%\renewenvironment{proof}{{\bfseries Proof}}{\qed}

\def\remark{\stepcounter{theorem}\par{\bf Remark}.
\ignorespaces}
\def\endremark{ \vbox{\hrule\hbox{%
    \vrule height1.3ex\hskip0.8ex\vrule}\hrule
   }\par}

\newenvironment{keywords}{\vspace{0.4in}\begin{quote}\small \em
{\bf Keywords\/}:}{\end{quote}}
\fi

%- - - - - - - - - - - - - - - - - - - - - - - - - - - - - - - - - - - -
%- - - - - - - - - - - - - - - - - - - - - - - - - - - - - - - - - - - -

\begin{document}

%- - - - - - - - - - - - - - - - - - - - - - - - - - - - - - - - - - - -
%- - - - - - - - - - - - - - - - - - - - - - - - - - - - - - - - - - - -

\title{QPLIB: a Library of a Quadratic Programming Instances}

\ifMPC
\titlerunning{QPLIB}
\fi

%- - - - - - - - - - - - - - - - - - - - - - - - - - - - - - - - - - - -
%- - - - - - - - - - - - - - - - - - - - - - - - - - - - - - - - - - - -

\ifMPC
\author{Fabio Furini \and
        Emiliano Traversi \and
        Alper Atamturk \and
        Pietro Belotti \and
        Pierre Bonami \and 
        Sourour Elloumi \and
        Antonio Frangioni \and 
        Ambros Gleixner \and
        Nick Gould \and 
        Leo Liberti \and
        Andrea Lodi \and
        Ruth Misener \and
        Hans Mittelmann \and
        Nick Sahinidis \and 
        Frederic Roupin \and 
        Stefan Vigerske \and 
        Angelika Wiegele
        }

\institute{Fabio Furini \at
           LAMSADE, Universit{\'e} Paris Dauphine, 75775 Paris, France,
           \email{fabio.furini@dauphine.fr}
%
\and       
           Emiliano Traversi \at
           LIPN, Universit{\'e} de Paris 13, 93430 Villetaneuse, France.
           \email{emiliano.traversi@lipn.fr}
%
\and       Alper Atamturk \at
%
\and       Pietro Belotti \at
%
\and       Pierre Bonami \at 
%
\and       Sourour Elloumi \at
%
\and       Antonio Frangioni \at
           Dipartimento di Informatica, Universit\`a di Pisa,
           Largo B. Pontecorvo 2, 56127 Pisa, Italy,
           \email{frangio@di.unipi.it}
%
\and       Ambros Gleixner \at
           Zuse Institute Berlin, Takustr. 7, 14195 Berlin, Germany. 
           \email{gleixner@zib.de}
%
\and       Nick Gould \at 
%
\and       Leo Liberti \at
           CNRS LIX, Ecole Polytechnique, 91128 Palaiseau, France.
           \email{liberti@lix.polytechnique.fr}
%
\and       Andrea Lodi \at
           Ecole Polytechnique de Montréal, Canada.               \email{andrea.lodi@polymtl.ca}
%
\and       Ruth Misener \at
%
\and       Hans Mittelmann \at
           School of Mathematical and Statistical Sciences, Arizona State University,      Tempe, AZ 85287-1804, U.S.A.
\email{mittelmann@asu.edu}
%
\and       Nick Sahinidis \at 
%
\and       Frederic Roupin \at 
%
\and       Stefan Vigerske \at 
           GAMS Software GmbH, P.O. Box 40 59,50216 Frechen, Germany
           \email{stefan@gams.com}
%
\and       Angelika Wiegele \at
           }

\else

\author{Fabio Furini\\
        LAMSADE, Universit{\'e} Paris Dauphine, 75775 Paris, France\\
        {\tt fabio.furini@dauphine.fr}
%
\and   
        Emiliano Traversi\\
        LIPN, Universit{\'e} de Paris 13, 93430 Villetaneuse, France\\
        {\tt emiliano.traversi@lipn.fr}
%
\and    Alper Atamturk\\
%
\and    Pietro Belotti\\
%
\and    Pierre Bonami\\
%
\and    Sourour Elloumi\\
%
\and    Antonio Frangioni\\
        Dipartimento di Informatica, Universit\`a di Pisa\\
        Largo B. Pontecorvo 2, 56127 Pisa, Italy\\
        {\tt frangio@di.unipi.it}
%
\and    Ambros Gleixner\\
%
\and    Nick Gould\\
%
\and    Leo Liberti\\
        CNRS LIX, Ecole Polytechnique, 91128 Palaiseau, France\\
        {\tt liberti@lix.polytechnique.fr}
%
\and    Andrea Lodi\\
%
\and    Ruth Misener\\
%
\and    Hans Mittelmann\\
%
\and    Nick Sahinidis\\
%
\and    Frederic Roupin\\
%
\and    Stefan Vigerske\\
%
\and    Angelika Wiegele
        }
\fi

%- - - - - - - - - - - - - - - - - - - - - - - - - - - - - - - - - - - -
%- - - - - - - - - - - - - - - - - - - - - - - - - - - - - - - - - - - -

\ifMPC
 \date{Received: date / Accepted: date}
\else
 \date{}
\fi

\maketitle

\begin{abstract}
 This paper describes a new instance library for Quadratic Programming (QP). QP is a very ``varied'' class, comprising sub-classes of problems ranging from trivial to indecidable. Correspondingly, solution methods for QP are very diverse, ranging from entirely combinatorial ones to completely continuous ones, to many where both aspects are fundamental. Selecting a set of instances of QP that is at the same time not overwhelmingly numerous and significant for the many different interested communities is therefore challenging. In order to help our selection process we propose a simple taxonomy for QP instances. We then briefly survey the field of QP, giving an overview of theory, methods and solvers. Finally we describe how the library was put together, and how the final results look.
\end{abstract}

\ifMPC
 \keywords{Instance Library, Quadratic Programming}

 \subclass{90C06 \and 90C25}
\else
 \begin{keywords}
  Instance Library, Quadratic Programming
 \end{keywords}
\fi


%- - - - - - - - - - - - - - - - - - - - - - - - - - - - - - - - - - - -

%- - - - - - - - - - - - - - - - - - - - - - - - - - - - - - - - - - - -
%- - - - - - - - - - - - - - - - - - - - - - - - - - - - - - - - - - - -
%  QPLIB-1.0.tex
%- - - - - - - - - - - - - - - - - - - - - - - - - - - - - - - - - - - -
%- - - - - - - - - - - - - - - - - - - - - - - - - - - - - - - - - - - -

%- - - - - - - - - - - - - - - - - - - - - - - - - - - - - - - - - - - -
%- - - - - - - - - - - - - - - - - - - - - - - - - - - - - - - - - - - -
\section{Introduction}\label{sec:intro}

Quadratic Programming (QP) problems, where either the objective function \cite{wiki:qp}, or the constraints \cite{wiki:qcqp}, or both contain (at most) expressions in the variables of degree $2$, include a surprisingly diverse set of rather different instances. This is not surprising, given the vast scope of practical applications of these problems, and of solution methods designed to solve them \cite{GoulToin00a}. According to the fine details of the formulation, solving a QP may require employing either fundamentally combinatorial techniques, or ideas rooted in nonlinear optimization principles, or a mix of the two. In this sense, QP is likely one of the classes of problems where the collaboration between the communities interested in combinatorial and nonlinear optimization is more necessary, and potentially fruitful.

However, this diversity also implies that QP means very different things to different researchers. This is illustrated by the simple fact that the class of problems that we simply refer to here as ``QP'' is called in different ways, among which Mixed-Integer Quadratically Constrained Quadratic Problems (MI-QCQP). It is perhaps therefore not surprising that, unlike for ``simpler'' problems classes \cite{Koch2011}, so far there has never been a single library containing all different kinds of instances of QP. Several libraries devoted to special cases of QP are indeed available; however, each of them is either focussed on one application (a specific problem that can be modeled as QP), or on QPs with specific structural properties that make them suitable to be solved with some given class of algorithmic approaches. To the best of our knowledge, collecting a set of QP instances that is at the same time not overwhelmingly numerous and significant for the many different interested communities has not been attempted, yet. This work constitutes a first step in this direction.

In this paper we report the steps that have been done to collect a (hopefully) significant library of QP instances, filtering the large set of available (or specifically provided) instances in order to end up with a manageable set that still contains a meaningful sample of all possible QP types. A particularly thorny issue in this process is how to select instances that are ``interesting''. Our choice has been to take this to mean ``challenging for a significant set of solution methods''. Our filtering process has then been in part based on the idea that if a significant fraction of the solvers that can solve a QP instance do so in a ``short'' time, then the instance is not challenging enough to be included in the library. Yet, if very few (maybe one) solvers can solve it very efficiently by exploiting some specific structure, but most other approaches cannot, then the instance can still be deemed ``interesting''. Putting this approach in practice requires a nontrivial number of technical steps and decisions that are detailed in the paper. We hope that our work can provide useful guidelines for other interested researchers.

A consequence of our focus is that this paper is \emph{not} concerned about the different performance of the very diverse QP solvers; we will \emph{not} report any data comparing them. The only reason why solvers are used (and, therefore, described) in this context is to ensure that the instances of the library are nontrivial at least for a significant fraction of the current solution methods; providing guidance about which solver is most suited to some specific class of QPs is entirely outside the scope of our work.

%- - - - - - - - - - - - - - - - - - - - - - - - - - - - - - - - - - - -
\subsection{Motivation}\label{subsec:motiv}

\framebox{TASK 2 : write motivation}

Optimization problems with quadratic constraints and/or objective function (QP) have been the subject of a considerable amount of research over the last 60 years. At least some of the rationale for this interest is likely due to the fact that they are the ``least-nonlinear nonlinear problems''. Hence, in particular for the convex case, tools and techniques that have been honed during decades of research for Linear Programming (LP), typically with integrality constraints (MILP), can often be extended to the quadratic case with at least less effort than what would be required for tackling general Non Linear Programming (NLP) problems, without or with integrality constraints (MINLP). This has indeed happened over and over again with different algorithmic techniques, such as Interior Point methods, active-set methods (of which the simplex method is a prototypical example), enumeration methods, cut-generation techniques, reformulation techniques, and many others ({\bf citations?}). Similarly, nonconvex continuous QP are perhaps the ``simplest'' class of problems that require features like spatial enumeration techniques to be solved. Hence, they are both the natural basis for the development of general techniques for nonconvex NLP, and a very specific class so that specialized approaches can be developed \cite{Dur2010,Burer2012}.

On the other hand, (MI)QP is, in some sense, a considerably more expressive class than (MI)LP. Quadratic expressions are found, either naturally or after appropriate reformulations, in very many practical problems (\cite{Kochenberger2014}, {\bf citations?}). In general, any continuous function can be approximated with arbitrary accuracy (over a compact set) by a polynomial of arbitrary degree; in turn, every polynomial can be broken down to a system of quadratic expressions. Hence, (MI)QP is, in some sense, roughly as expressive as the whole of (MI)NLP. Of course this is, in principle, true for (MI)LP as well, but at the cost of much larger and much more complex formulations ({\bf citations?}). Hence, for many applications QP may represent the ``sweet spot'' between the effectiveness, but lower expressive power, of (MI)LP and the higher expressive power, but much higher computational cost, of (MI)NLP.

\ldots

The structure of this paper is the following. In \S~\ref{sec:QPbasic} we review the basic notions about QP. In particular, \S~\ref{subsec:notation} sets out the notation, \S~\ref{sec:classification} proposes a---to the best of our knowledge, new---taxonomy of QP that helps us in discussing the (very) different classes of QPs, \S~\ref{sec:algo} very briefly reviews the solution methods for QP upon which the solvers we have employed are based, and \S~\ref{subsec:solver} describes the solvers. Then, \S~\ref{sec:lib} describes the process used to obtain the library and its results; the software tools that we have used, and that are freely released together with the library, are discussed in \S~\ref{subsec:tools}. Some conclusions are drawn in \S~\ref{sec:conclusions}, after which in the Appendix a complete description of all the instances of the library is provided.

%- - - - - - - - - - - - - - - - - - - - - - - - - - - - - - - - - - - -
%- - - - - - - - - - - - - - - - - - - - - - - - - - - - - - - - - - - -
%  End QPLIB-1.0.tex
%- - - - - - - - - - - - - - - - - - - - - - - - - - - - - - - - - - - -
%- - - - - - - - - - - - - - - - - - - - - - - - - - - - - - - - - - - -


%- - - - - - - - - - - - - - - - - - - - - - - - - - - - - - - - - - - -

%- - - - - - - - - - - - - - - - - - - - - - - - - - - - - - - - - - - -

%- - - - - - - - - - - - - - - - - - - - - - - - - - - - - - - - - - - -
%- - - - - - - - - - - - - - - - - - - - - - - - - - - - - - - - - - - -
%  QPLIB-2.1.tex
%- - - - - - - - - - - - - - - - - - - - - - - - - - - - - - - - - - - -
%- - - - - - - - - - - - - - - - - - - - - - - - - - - - - - - - - - - -

%- - - - - - - - - - - - - - - - - - - - - - - - - - - - - - - - - - - -
%- - - - - - - - - - - - - - - - - - - - - - - - - - - - - - - - - - - -
\section{Quadratic Programming in a nutshell}\label{sec:QPbasic}

%- - - - - - - - - - - - - - - - - - - - - - - - - - - - - - - - - - - -
\subsection{Notation}\label{subsec:notation}

In mathematical optimization, a Quadratic Program (QP) is an optimization problem in which either the objective function, or some of the constraints, or both, are quadratic functions. More specifically, the problem has the form
%
\begin{align*}
 \min \textrm{or} \max \;\;
 & \half x^\top Q^0 x + b^0 x + q^0 \\
 \textrm{such that}\;\;
 & c^i_l \leq \half x^\top Q^i x + b^i x \leq c^i_u & i \in \Mcm, \\
 %
 & l_j \leq x_j \leq u_j & j \in \Mcn,  \\
 %
 \textrm{and}\;\;
 & x_j \in \mathbb{Z} & j \in \Mcz,
\end{align*}
%
%\begin{align*}
% \min \;\;
% & x^\top Q^0 x + b^0 x \\
% %
% & x^\top Q^i x + b^i x \leq c^i & i \in \Mcm \\
% %
% & l_j \leq x_j \leq u_j & j \in \Mcn  \\
% %
% & x_j \in \mathbb{Z} & j \in \Mcz
%\end{align*}
%
where
%
\begin{itemize}
 \item $\Mcn = \{ 1, \ldots, n \}$ is the set of (indices) of variables, and $\Mcm = \{ 1, \ldots, m \}$ is the set of (indices) of constraints;
 %
 \item $x = [x_j]_{j = 1}^n$ %\in \RR^n$
is a finite vector of real variables;
 %
 \item $Q^i$ for $i \in \{ 0 \} \cup \Mcm$ are symmetric $n \times n$ real (Hessian) matrices: since one is only interested in the value of quadratic forms of the type $x^\top Q^i x$, symmetry can be assumed without loss of generality by just replacing off diagonal pairs $Q^i_{hk}$ and $Q^i_{kh}$ with their average $(Q^i_{hk} + Q^i_{kh}) / 2$;
 %
 \item $b^i$, $c_u^i$, $c_l^i$ for $i \in \{ 0 \} \cup \Mcm$, and $q^0$ are, respectively, real $n$-vectors and real constants;
 %
 \item $-\infty \leq l_j \leq u_j \leq \infty$ are the (extended) real lower and upper bounds on each variable $x_j$ for $j \in \Mcn$;
 %
 \item $\Mcm = \Mcq \cup \Mcl$ where $Q^i = 0$ for all $i \in \Mcl$ (i.e., these are the linear constraints, as opposed to the truly quadratic ones); and
 %
 \item the variables in $\Mcz \subseteq \Mcm$ are restricted to only attain integer values.
\end{itemize}
%
Due to the presence of integrality requirements on the variables and of quadratic constraints, this class of problems is often referred to as Mixed-Integer Quadratically Constraint Quadratic Program (MIQCQP). It will be sometimes useful to refer to the (sub)set $\Mcb =  \{ \, j \in \Mcz : l_j = 0 , u_j = 1 \, \} \subseteq \Mcz$ of the binary variables, and to $\Mcr = \Mcn \setminus \Mcz$ as the set of continuous ones. Similarly, it will be sometimes useful to distinguish the (sub)set $\Mcx = \{ \, j : l_j > -\infty \lor u_j < \infty \, \}$ of the box-constrained variables from $\Mcu = \Mcn \setminus \Mcx$ of the unconstrained ones (in the sense that finite bounds are not explicitly provided in the data of the problem, although they may be implied by the other constraints).

The relative flexibility offered by quadratic functions, as opposed e.g.~to linear ones, allows several reformulation techniques to be applicable to this family of problems in order to emphasize different properties of the various components. Some of these reformulation techniques will be commented later on; here we remark that, for instance, integrality requirements, in particular in the form of binary variables could always be ``hidden'' by introducing (nonconvex) quadratic constraints utilizing the celebrated relationship $x_j \in \{0, 1\} \iff x_j^2 = x_j$. Therefore, when discussing these problems some effort has to be made to distinguish between features that come from the original model, and those that can be introduced by reformulation techniques in order to extract (and algorithmically exploit) specific properties.

%- - - - - - - - - - - - - - - - - - - - - - - - - - - - - - - - - - - -
%- - - - - - - - - - - - - - - - - - - - - - - - - - - - - - - - - - - -
%  End QPLIB-2.1.tex
%- - - - - - - - - - - - - - - - - - - - - - - - - - - - - - - - - - - -
%- - - - - - - - - - - - - - - - - - - - - - - - - - - - - - - - - - - -


%- - - - - - - - - - - - - - - - - - - - - - - - - - - - - - - - - - - -

%- - - - - - - - - - - - - - - - - - - - - - - - - - - - - - - - - - - -

%- - - - - - - - - - - - - - - - - - - - - - - - - - - - - - - - - - - -
%- - - - - - - - - - - - - - - - - - - - - - - - - - - - - - - - - - - -
%  QPLIB-2.2.tex
%- - - - - - - - - - - - - - - - - - - - - - - - - - - - - - - - - - - -
%- - - - - - - - - - - - - - - - - - - - - - - - - - - - - - - - - - - -


%- - - - - - - - - - - - - - - - - - - - - - - - - - - - - - - - - - - -
\subsection{Classification}\label{sec:classification}

Despite the apparent simplicity of the definition given in \S \ref{subsec:notation}, Quadratic Programming instances can be of several rather different ``types'' in practice, depending on the fine details of the data. In particular, many algorithmic approaches can only be applied to QP when the data of the problem has specific properties. A taxonomy of QP instances should therefore strive to identify the set of properties that an instance should have in order to apply the most relevant computational methods. However, the sheer number of different existing approaches, and the fact that new ones are proposed, makes it hard to provide a taxonomy that is both simple and covers all possible special cases. This is why, in this paper, we propose an approach that aims at finding a good balance between simplicity and coverage of the main families of computational methods.

% -  -  -  -  -  -  -  -  -  -  -  -  -  -  -  -  -  -  -  -  -  -  -  -
\subsubsection{Classification}\label{ssec:taxonomy}

Our taxonomy is based on a three-fields code of the form ``\textit{OVC}'', where \textit{O} indicates the objective function, \textit{V} the variables, and \textit{C} the constraints of the problem. The fields can be given the following values:
%
\begin{itemize}
 \item objective function: (L)inear, (D)iagonal convex quadratic, (C)onvex quadra-tic, nonconvex (Q)uadratic;
 %
 \item variables: (C)ontinuous only, (B)inary only, (M)ixed binary and continuous, (I)nteger only, (G)eneral (all three types);
 %
 \item constraints: (N)one, (B)ox, (L)inear, (D)iagonal convex quadratic, (C)onvex quadratic, nonconvex (Q)ua-dratic.
\end{itemize}
%
The wildcard ``*'' will be used to indicate any possible choice, and lists of the form ``\{X, Y, Z\}'' will indicate that the value of the given field can freely attain any of the specified values.

The ordering of the values in the previous lists is not irrelevant; in general, problems become ``harder'' when going from left to right. More specifically, for the \textit{F} and \textit{C} fields the order is that of \emph{strict} containment between problem classes: for instance, Linear objective functions are strictly contained in Diagonal convex quadratic ones (by just allowing the diagonal elements to be all-zero), which are strictly contained into general Convex quadratic ones (by allowing the off-diagonal elements to be all-zero), which in turn are strictly contained into general nonconvex Quadratic ones (by allowing any symmetric $Q^0$, hence possibly positive semidefinite ones as well). The only field for which the containment relationship is not a total order is \textit{V}, for which only the partial orderings
\[
 \mbox{C} \subset \mbox{M} \subset \mbox{G}
 \qquad,\qquad
 \mbox{B} \subset \mbox{M} \subset \mbox{G}
 \qquad,\qquad
 \mbox{B} \subset \mbox{I} \subset \mbox{G}
\]
hold. In the following discussion we will repeatedly exploit this by assuming that, unless otherwise mentioned, when a method can be applied to a given problem, it can be applied as well to all simpler problems where the value of each field is arbitrarily replaced with a value denoting a less-general class.

% -  -  -  -  -  -  -  -  -  -  -  -  -  -  -  -  -  -  -  -  -  -  -  -
\subsubsection{Examples and reformulations}\label{ssec:reform}
\todo{\small Ambros: maybe switch this with the next section? the next one contains more familiar material to the reader and might be an easier first introduction to the notation.}

We will now provide a first general discussion about the different problem classes that our proposed taxonomy defines. Some of them are actually ``too simple'' to make sense in our context. For instance, D*B problems have only diagonal quadratic (hence separable) objective function and bound constraints; as such, they read
\[
 \textstyle
 \min \; \Big\{ \,
 \sum_{j \in \Mcn} \big( \, Q^0_j x_j^2 + b^0_j x_j \, \big) \;:\;
 l_j \leq x_j \leq u_j \quad j \in \Mcn \;\;,\;\;
 x_j \in \mathbb{Z} \quad j \in \Mcz \, \Big\}
 \;\; .
\]
Hence, their solution only requires the independent minimization of a convex quadratic univariate function in each single variable $x_j$ over a box constraint and possibly integrality requirements, which can be attained trivially in $O(1)$ (per variable) by closed-form formul{\ae}, projection and rounding arguments. \emph{A fortiori}, the even simpler cases L*B, D*N and L*N (the latter obviously unbounded unless $b^0 = 0$) will not be discussed here. Similarly, CCN are immediately solved by linear algebra techniques, and therefore are of no interest in this context. On the other end of the spectrum, in general QP is a hard problem. Actually, LIQ---linear objective function and quadratic constraints in integer variables with no finite bounds, i.e. 
\[
 \textstyle
 \min \; \Big\{ \, b^0 x \;:\;
 x^\top Q^i x + b^i x \leq c^i \quad i \in \Mcm \;\;,\;\;
 x_j \in \mathbb{Z} \quad j \in \Mcn \, \Big\}
 \;\; ,
\]
is not only $\mathcal{NP}$-hard, but downright undecidable \cite{jeroslow}. Hence so are the ``harder'' \{C,Q\}IQ.

\smallskip
It is important to note that the relationships between the different classes can be somehow blurred because reformulation techniques may allow to move one instance from one class to the other. The already recalled fact that $x^2 = x \iff x \in \{0, 1\}$, for instance, says that *M*---instances with only binary and continuous variables---can be recast as *CQ: nonconvex quadratic constraints can always take the place of binary variables. Actually, this is also true for *G* as long as $\Mcu = \emptyset$, as bounded general integer variables can be represented by binary ones.

Another relevant reformulation trick concerns the fact that, as soon as quadratic constraints are allowed, then a linear objective function can be assumed w.l.o.g.. Indeed, any Q** (C*C) problem can always be rewritten as
%
\begin{align*}	
 \min \;\;
 & x^0 \\
 %
 & x^\top Q^0 x + b^0 x \leq x^0 & \\ 
 %
 & x^\top Q^i x + b^i x \leq c^i & i \in \Mcm \\ 
 %
 & l_j \leq x_j \leq u_j & j \in \Mcn  \\
 %
 & x_j \in \mathbb{Z} & j \in \Mcz
\end{align*}
%
i.e., a L*Q (L*C). Hence, it is clear that quadratic constraints are, in a well-defined sense, the most general situation (cf.~also the result above about hardness of LIQ).

When a $Q^i$ is positive semidefinite (PSD), i.e., the corresponding constraint/objective function is convex, general quadratic constraints are in fact equivalent to diagonal ones. In fact, every PSD matrix can be factorized as $Q^i = L^i (L^i)^\top$, e.g.~by the (incomplete) Cholesky factorization, $f^i(x) = x^\top Q^i x = \sum_{j \in \Mcn} z^2_jz$ where $z = x^\top L^i$. Hence, one could think that D** problems need not be distinguished from C** ones; however, this is true only for ``complicated'' constraints, but not for ``simple'' ones, because the above reformulation technique introduces linear constraints. Indeed, while C*L (and, a fortiori, C*\{C,Q\}) can always be brought to D*L (D*\{C,Q\}), using the same technique C*B becomes D*L, which is significantly different from D*B. In practice, a diagonal convex objective function under linear constraints is found in many applications ({\bf citations?}), so that D*L still makes sense to distinguish the case where the objective function is ``naturally'' separable from that where separability is artificially introduced, although this is in theory always possible.

% -  -  -  -  -  -  -  -  -  -  -  -  -  -  -  -  -  -  -  -  -  -  -  -
\subsubsection{QP classes}\label{ssec:classes}

The proposed taxonomy can then be used to describe the main classes of QP according to the type of algorithms that can be applied for their solution:
%
\begin{itemize}
 \item \emph{Linear Programs} LCL and \emph{Mixed-Integer Linear Programs} LGL have been subject of an enormous amount of research and have their well-established instance libraries \cite{Koch2011}, so they won't be explicitly addressed here.
 %
 \item \emph{Convex Continuous Quadratic Programs} CCC can be solved in polynomial time by Interior-Point techniques; the simpler CCL can also be solved by means of ``simplex-like'' techniques, usually referred to as active-set methods ({\bf citations?}). Actually, a slightly larger class of problems can be solved with Interior-Point methods: those that can be represented by Second-Order Cone Programs. When written as QPs the corresponding $Q^i$ may not be positive semidefinite, but still the problems can be efficiently solved. Of course, these problems can still require considerable time, like LCL, when the size of the instance grows. In this sense, like in the linear case, a significant divide is from solvers that need all the data of QP to work, and those that are ``matrix-free'', i.e., only require the solution of simple operations (typically, matrix-vector products) with the data of the problem to work ({\bf citations?}). While in our library we have never exploited such a characteristic, which is not suitable to the use of standard modeling tools, this may be relevant for the solution of very-large-scale CIC.
 %
 \item \emph{Nonconvex Continuous Quadratic Programs} QCQ are instead in general $\mathcal{NP}$-hard, even if the constraints are very specific (QCB) and only one eigenvalue of $Q^0$ is negative \cite{Hemmecke2010}. They therefore require enumerative techniques like the spatial Branch\&Bound ({\bf citations?}), to be solved to optimality. Of course, local approaches are available that are able to efficiently provide saddle points (hopefully, local optima) of the CQC, but providing global guarantees about the quality of the obtained solutions is challenging. In our library we have specifically focussed on  \emph{exact} solution of the instances.
  %
 \item \emph{Convex Integer Quadratic Programs} CGC are in general $\mathcal{NP}$-hard, and therefore require enumerative techniques to be solved. However, convexity of the objective function and constraints implies that efficient techniques (see CCC) can be used at least to solve the continuous relaxation. The general view is that CGC are not, all other things being equal,  substantially more difficult than LGL to solve, especially if the objective function and/or the constraints have specific properties (e.g., DGL, CGL). Often integer variables are in fact binary ones, so several CCC models are C\{B,M\}C ones. In practice binary variables are considered to lead to somewhat easier problems than general integer ones (cf.~the cited result about hardness of unbounded integer quadratic programs), and several algorithmic techniques have been specifically developed for this special case. However, the general approaches for CBC are basically the same as for CGC, so there is seldom the need to distinguish between the two classes as far as solvability is concerned, although matters can be different regarding actual solution cost. Programs with only binary variables CBC can be easier than mixed-binary or integer ones C\{M,I\}C because some techniques are specifically known for the binary-only case, cf.~the next point ({\bf citations?}). Absence of continuous variables, even in the presence of integer ones CIC, can also lead to specific techniques ({\bf citations??}).  
  %
  \item \emph{Nonconvex Binary Quadratic Programs} QB\{B, N, L\} obviously are $\mathcal{NP}$-hard. However, the special nature of binary variables combined with quadratic forms allows for quite specific techniques to be developed, among which is the reformulation of the problem as a LBL. Also, many well-known combinatorial problems can be naturally reformulated as problems of this class, and therefore a considerable number of results have been obtained by exploiting specific properties of the set of constraints ({\bf some citations? some specific problem to mention apart from Max-Cut?}).
  %
  \item \emph{Nonconvex Integer Quadratic Programs} QGQ is the most general, and therefore is the most difficul, class. Due to the lack of convexity even when integrality requirements are removed, solution methods must typically combine several algorithmic ideas, such as enumeration (distinguishing the role of integral variables from that of continuous ones involved into nonconvex terms) and techniques (e.g., outer approximation, SDP relaxation, \ldots) that allow to efficiently compute bounds. As in the convex case, QBQ, QMQ, and QIQ can benefit from more specific properties of the variables ({\bf citations?}).
\end{itemize}
%
This description is purposely coarse; each of these classes can be subdivided into several others on the grounds of more detailed information about structures present in their constraints/objective function. These can have a significant algorithmic impact, and therefore can be of interest to researchers. Common structures are, e.g., network flow or knapsack-type linear constraints, semi-continuous variables, or the fact that a nonconvex quadratic objective function/constraint can be reformulated as a second-order cone (hence, convex) one. It would be rather hard to collect a comprehensive list of all types of structures that may be of interest to any individual researcher, since these are as varied as the different possible approaches for specialized sub-classes of QP. For this reason we do not attempt such a more refined classification, and limit ourselves to the coarser one described in this paragraph.

%- - - - - - - - - - - - - - - - - - - - - - - - - - - - - - - - - - - -
%- - - - - - - - - - - - - - - - - - - - - - - - - - - - - - - - - - - -
%  End QPLIB-2.2.tex
%- - - - - - - - - - - - - - - - - - - - - - - - - - - - - - - - - - - -
%- - - - - - - - - - - - - - - - - - - - - - - - - - - - - - - - - - - -


%- - - - - - - - - - - - - - - - - - - - - - - - - - - - - - - - - - - -

%- - - - - - - - - - - - - - - - - - - - - - - - - - - - - - - - - - - -

\input{QPLIB-2.3.tex}

%- - - - - - - - - - - - - - - - - - - - - - - - - - - - - - - - - - - -

%- - - - - - - - - - - - - - - - - - - - - - - - - - - - - - - - - - - -
%- - - - - - - - - - - - - - - - - - - - - - - - - - - - - - - - - - - -
%  QPLIB-2.4.tex
%- - - - - - - - - - - - - - - - - - - - - - - - - - - - - - - - - - - -
%- - - - - - - - - - - - - - - - - - - - - - - - - - - - - - - - - - - -
\subsection{Solvers}\label{subsec:solver}
\todo{\small Ambros: merge with Section~\ref{s:complete} or put list of solvers in Section~\ref{s:complete} here}

We now provide a succinct list of the solvers we have tested, using the approaches described in \S \ref{sec:algo}. In Table \ref{t:solvers}, we mark with ``I'' a pair (solver, problem) if the solver accepts the problem in input but it is an incomplete solver for the problem, with ``A'' if it is asymptotically complete, with ``C'' if it is complete, and leave it blank if the solver won't accept the problem in input.

{\bf The table has to be checked, as I've extrapolated from the text but I'm not 100\% sure. Also, ``?''s have to be removed.}


\begin{table}
{
 \centering                
 \scriptsize                
 \setlength{\tabcolsep}{3pt}                
% \renewcommand \arraystretch{1}                
                
\begin{tabular}{lccccccc}
\toprule  
                 & CGL & QGL & CGC & QGQ & CCC & QCQ \\
\hline
{\sc Antigone}    &  C  &  C  &  C  &  C  &  C  &  C  \\
{\sc Baron}       &  C  &  C  &  C  &  C  &  C  &  C  \\
{\sc Couenne}     &  C  &  C  &  C  &  C  &  C  &  C  \\
{\sc KNitro}      &  C  &  I  &  C  &  I  &  C  &  A  \\
{\sc lindo api}   &  C  &  C  &  C  &  C  &  C  &  C  \\
{\sc SCIP}        &  C  &  C  &  C  &  C  &  C  &  C  \\
{\sc oqnlp}       &  A  &  A  &  A  &  A  &  C  &  A  \\
{\sc AlphaECP}    &  C  &  I  &  C  &  I  &  C  &  I  \\
{\sc BonMin}      &  C  &  I  &  C  &  I  &  C  &  I  \\
{\sc DICOpt}      &  C  &  I  &  C  &  I  &  C  &  I  \\
{\sc sBB}         &  C  &  I  &  C  &  I  &  C  &  I  \\
{\sc ConOpt}      &     &     &     &     &  C  &  I  \\
{\sc IpOpt}       &     &     &     &     &  C  &  I  \\
{\sc lgo}         &     &     &     &     &  C? &  A  \\  %try it out
{\sc Minos}       &     &     &     &     &  C  &  I  \\
{\sc msnlp}       &     &     &     &     &  C  &  A  \\
{\sc SnOpt}       &     &     &     &     &  C  &  I  \\
{\sc XPress}      &  C  &     &  C  &     &  C  &  ?  \\ %try it out
{\sc Gurobi}      &  C  &     &  C  &     &  C  &  ?  \\ %try it out
{\sc CPLEX}       &  C  &  C  &  C  &     &  C  &  I? \\ %try it out
{\sc mosek}       &  C  &     &  C  &     &  C  &     \\
\hline
\end{tabular}                 
\caption{Families of QP problems that can be tackled each solver} \label{t:solvers}             
} 

\end{table}              

%- - - - - - - - - - - - - - - - - - - - - - - - - - - - - - - - - - - -
%- - - - - - - - - - - - - - - - - - - - - - - - - - - - - - - - - - - -
%  End QPLIB-2.4.tex
%- - - - - - - - - - - - - - - - - - - - - - - - - - - - - - - - - - - -
%- - - - - - - - - - - - - - - - - - - - - - - - - - - - - - - - - - - -


%- - - - - - - - - - - - - - - - - - - - - - - - - - - - - - - - - - - -

%- - - - - - - - - - - - - - - - - - - - - - - - - - - - - - - - - - - -

%- - - - - - - - - - - - - - - - - - - - - - - - - - - - - - - - - - - -
%- - - - - - - - - - - - - - - - - - - - - - - - - - - - - - - - - - - -
%  QPLIB-3.tex
%- - - - - - - - - - - - - - - - - - - - - - - - - - - - - - - - - - - -
%- - - - - - - - - - - - - - - - - - - - - - - - - - - - - - - - - - - -
\section{Library Construction}\label{sec:lib}

In this section we present all the steps we performed in order to build
the new instance library. In \S\ref{subsec:instColl}, we describe the set
of gathered instances, and
in \S\ref{subsec:selection} we present the features used to
classify the instances.
%and discuss issues concerning the format of the instances.
We describe the selection process used to filter the instances, and
graphically present the main features of the selected instances in
\S\ref{subsec:final set}, while in \S\ref{subsec:website}
we provide information on how to access the test collection.

%- - - - - - - - - - - - - - - - - - - - - - - - - - - - - - - - - - - -
\subsection{Instance Collection}\label{subsec:instColl}

In this section we describe the procedure we adopted to gather the
instances. In January $2014$, we issued an online call for instances
using main international mailing lists of the mathematical
optimization and numerical analysis communities, reaching in this way
a \textcolor{red}{large set of possibly interested} researchers and practitioners.
The call remained open for ten months, during which we received a large
number of contributions of different nature. The instances we gathered
come both from theoretical studies as well as from real-world
applications.


In addition to these spontaneous contributions we analyzed existing generic
libraries of instances available  on the internet that contain QP
instances. Namely, the libraries from which we gathered instances are
%
\begin{itemize}
 \item the \texttt{BARON} library
 \url{http://www.minlp.com/nlp-and-minlp-test-problems};
%
\item the \texttt{CUTEst} library
 \url{https://ccpforge.cse.rl.ac.uk/gf/project/cutest};
%
\item the \texttt{GAMS Performance} libraries
 \url{http://www.gamsworld.org/performance/performlib.htm};
%
\item the \texttt{MacMINLP} library
 \url{https://wiki.mcs.anl.gov/leyffer/index.php/MacMINLP};
%
\item the \texttt{Maros-M{\'e}sz{\'a}ros} library
 \url{http://www.doc.ic.ac.uk/~im/00README.QP};
%
\item the \texttt{MINLPLib} library
 \url{http://www.gamsworld.org/minlp/minlplib.htm};
%
\item the \texttt{POLIP} library
 \url{http://polip.zib.de/pipformat.php}.
\end{itemize}

Other quadratic instances were found in online libraries devoted to
specific QP problems as Max-Cut, Quadratic Assignment, Portfolio
Optimization, and several others. \textcolor{red}{In addition, we mention
that other generic libraries exist, e.g., 
Conic library CBLIB (\url{http://cblib.zib.de}) and 
MIPLIB~2010 (\url{http://miplib.zib.de/}), to mention just a few.}

At the end of this process we had gathered more than eight thousand
instances. Three quarters of them contained discrete variables, while
the remainder contained only continuous variables. In more detail,
we gathered $\approx 1800$ Quadratic Binary Linear (\textit{QBL}) instances,
$\approx 2000$ Quadratic Continuous Quadratic (\textit{QCQ}) instances,
and $\approx 2500$ Quadratic General Quadratic (\textit{QGQ}) instances. We
also received $\approx 1000$ Convex General Convex (\textit{CGC}) instances. We
obtained relatively fewer Quadratic Binary Quadratic (\textit{QBQ}), Convex
Continuous Convex (\textit{CCC}) and Convex Mixed Convex (\textit{CMC}) instances,
($\approx 150$, $\approx 200$, and $\approx 200$ instances, respectively).
Finally, we found only $17$ Quadratic Mixed
Linear (\textit{QML}) instances. In the call for instances, no specific format
requirements were imposed for the submissions.

To evaluate the instances we decided, for practical reasons, to use
GAMS as common platform for all our final selection computations.
%involving commercial and non-commercial solvers.
For this reason, we translated all the
instances we received into the GAMS format (\texttt{.gms}).
% In a preliminary phase, all the instances received were divided
%according to their format and subsequently translated.
%In \S\ref{subsec:tools}
%%~\ref{subsec:CP_convex}
%the tools used
%to translate an instance from a given format to the \texttt{.gms} format
%%to translate an instance from and to a given format to the
%\texttt{.gms} format
%are described in more detail.\\
%

%\subsection{Instance Features}\label{subsec:feature}

For each instance in this large starting set,
we collected important characteristics
which allowed us to classify the instances into the QP categories described in
\S\ref{sec:QPbasic}. As far as the variable types are concerned, we
collected the following information:
%
\begin{itemize}
 \item the number of binary variables; % (\emph{\# bin}),
 %
 \item the number of integer variables; and % (\emph{\# int}),
 %
 \item the number of continuous variables. % (\emph{\# cont}).
\end{itemize}
%
If at least one binary or integer variable is present, then the instance is
categorized as \emph{discrete}, otherwise it is categorized as \emph{continuous}.
As far as the objective function is concerned, we gathered the following
information:
%
\begin{itemize}
 \item the percentage of positive and negative eigenvalues of the Hessian
      $Q^0$; and
       % (\emph{\% neg eig}), and
 %
 \item the density of the Hessian $Q^0$ (number of nonzero entries divided by the total
       number of entries). %(\emph{\% dens}).
\end{itemize}
%

The number of positive (i.e., larger than $10^{-12}$) and negative (i.e., smaller than $-10^{-12}$) eigenvalues of $Q^0$ allowed us to identify the
objective function type, as in presence of at least one negative (positive) eigenvalue
the objective function is nonconvex (nonconcave). Finally, as far as the constraint types
are concerned, we collected the following information:
%
\begin{itemize}
 \item the number of linear constraints, % (\emph{\# lin}),
 %
 \item the number of quadratic constraints, % (\emph{\# quad}),
 %
 \item the number of convex constraints, and % (\emph{\# conv}),
 %
 \item the number of variable bounds (for non-binary variables). %(\emph{\# box}).
\end{itemize}
%
A constraint is considered quadratic if it contains at least one nonzero in
a quadratic term (if present). Among the quadratic constraints, the ones whose
Hessians have only non-negative eigenvalues (when $c_u^i < \infty$) and
non-positive eigenvalues (when $c_l^i > - \infty$)
are classified as convex constraints; thus, a quadratic constraint with
two sided, finite bounds is nonconvex.
Note that this might occasionally lead us to classify some instances that have conic constraints as nonconvex ones, although their feasible region is in fact convex---fortunately, only some solvers are capable of properly exploiting this property.
 All this information allowed us to analyse the gathered instances and to perform the
filters described in the next paragraph.

%- - - - - - - - - - - - - - - - - - - - - - - - - - - - - - - - - - - -
\subsection{Instance Selection}\label{subsec:selection}

During the development of the library, a discussion ensued about
the expected goals that we wished to achieve. The following four goals
were finally identified:
%
\begin{enumerate}
 \item to represent as far as possible all the different categories of QP
       problems;
 %
 \item to gather ``challenging'' instances, i.e., ones which can not be easily
       solved by  state-of-the-art solvers;
 %
 \item to include, for each of the categories, a set of well-diversified
       instances; and
 %
 \item to obtain a set of instances which is neither too small, so as to
       preserve statistical relevance, nor too large so as to being
       computationally manageable.
\end{enumerate}
%
To achieve such goals, we performed the following two filters, applied in
a cascade.
%
\begin{itemize}
 \item \emph{First Instances Filter.}\\
       The first filter was designed to drastically reduce the number of
       instances by eliminating the ``easy'' ones. An empirical measure
       for the hardness of an instance is the CPU time needed by a
       complete solver (cf.~\S\ref{sec:algo}) to solve it to
       global optimality. Accordingly, for each of the gathered instance we
       ran the complete solvers in GAMS, which number depends on the category
       of the instance under consideration, cf.~Table \ref{t:solvers}.
       We then filtered according to a first
       measure of computational difficulty, i.e., we discarded all instances
       that are solved by at least 30\% of the complete solvers within a time
       limit of 30 seconds.
 %
 \item \emph{Second Instances Filter.}\\
       The goal of the second filter was to eliminate ``similar'' instances.
       We carefully analyzed the instances one by one, and we clustered them
       according to their features; for each cluster we kept only a few
       representatives, e.g. by eliminating all but a few of those with very similar size and coming from the same donor. Finally, in order to only
       keep computationally challenging instances we ran a complete solver for \textit{QGQ} with a time limit of 120 seconds; all the
       instances which have been solved to proven optimality within this time limit
       were discarded.
\end{itemize}
%
In Table~\ref{tab:filters} we summarize the two filter steps, which
allowed us to identify the final set of $319$ discrete instances and
$134$ continuous instances.

\begin{table}
 \centering
 \setlength{\tabcolsep}{5pt}
%  \arraystretch{1}
\begin{tabular}{cccc}
Starting set& \multicolumn{ 2}{c}{ $\approx$ 8500 Instances }& \\
& \multicolumn{ 2}{c}{$\Downarrow$}& \\
& $\approx$ 6000 Discr. Inst.  & $\approx$ 2500 Cont. Inst. & \\
First Filter  & $\Downarrow$  & $\Downarrow$ & \\
 & $\approx$ 3000 Discr. Inst.  & $\approx$ 1000 Cont. Inst. & \\
Second Filter & $\Downarrow$  & $\Downarrow$  & \\
% & 600 Discr. Inst.  & 250  Cont. inst. & \\
  & 319 Discr. Inst.  & 134  Cont. inst. & \\
\end{tabular}
%\begin{center}\end{center}
\caption{Instance filter steps} \label{tab:filters}
\end{table}

%- - - - - - - - - - - - - - - - - - - - - - - - - - - - - - - - - - - -
\subsection{Analysis of the Final Set of Instances}\label{subsec:final set}

\textcolor{red}{We now analyze the features of the instances selected to be part of the
 library. In Table \ref{tab:DDDDD}, we provide a global overview. The instances have been divided in \textit{continuous} vs \textit{discrete} and \textit{convex} vs \textit{nonconvex}, forming in this way, a  classification of 4 macro categories. As previously mentioned, an instance is classified \textit{discrete} if it contains at least a binary or integer variable, and \textit{continuous} otherwise. On the other hand, an instance is classified as \textit{nonconvex} if the objective function is nonconvex (if minimization) or nonconcave (if maximization) and/or at least one of the constraints is nonconvex, and \textit{convex} otherwise. }


\begin{table}
\textcolor{red}{
 \centering
 %\scriptsize
 \setlength{\tabcolsep}{18pt}
 \renewcommand \arraystretch{1.1}
\begin{tabular}{llr}
\toprule
Variables & Convexity & \#\\
\cmidrule(lr){1-3}
%
 \textit{continuous}	& \textit{convex}		&  32 \\[1.2 ex]
 \textit{continuous}	& \textit{nonconvex} 	&  102 \\[1.2 ex]
 \textit{discrete}	& \textit{convex}		&  31 \\[1.2 ex]
 \textit{discrete}	& \textit{nonconvex}	&  288 \\[1.2 ex]
\cmidrule(lr){1-3}
Total		&				& 453\\
%
\bottomrule
\end{tabular}
\caption{Macro classification of the final set of instances}
\label{tab:DDDDD}
}
\end{table}

 The detailed characteristics of the instances are presented in Table
\ref{tab:DD} for \emph{discrete} instances
(\textit{*}\{\textit{B,M,I,G}\}\textit{*}) and in Table
\ref{tab:CC} for \emph{continuous} ones (\textit{*C*}).
For each category, the tables
report in column ``$\#$'' the corresponding number of instances. It can be seen
that the final set well respects the original distribution of the gathered
instances among the different categories. Indeed, the discrete categories
\textit{LMQ} and \textit{QBL} are well represented by \textcolor{red}{134} and \textcolor{red}{91}
instances, respectively. Similarly, the continuous categories
\textit{LCQ} and \textit{QCQ} are well
represented by \textcolor{red}{52} and \textcolor{red}{30} instances, respectively. Moreover, the library
actually covers the large majority of all possible categories of instances.

\begin{table}
 \centering
 %\scriptsize
 \setlength{\tabcolsep}{18pt}
 \renewcommand \arraystretch{1.1}
\begin{tabular}{lllr}
\toprule
Obj. Fun. & Variables & Constraints & \#\\
\cmidrule(lr){1-4}
%
\multirow{5}*{Linear}
          & \multirow{1}*{Binary}
%                    & None      &   \\
%          &         & Linear    &  \\
                    & Quadratic &   9 \\[1.2 ex]
\cmidrule(lr){2-4}
          & \multirow{2}*{Mixed}
                    & Convex    &   \textcolor{red}{14}\\[1.2 ex]
          &         & Quadratic &  \textcolor{red}{134}\\[1.2 ex]
\cmidrule(lr){2-4}
          & \multirow{1}*{Integer}
%                    & Linear    &    \\
                   & Quadratic &    2\\[1.2 ex]
\cmidrule(lr){2-4}
          & \multirow{1}*{General}
%                    & Linear    &    \\
                   & Quadratic &    3\\[1.2 ex]
\cmidrule(lr){1-4}
%\multirow{5}*{Linear}
%          & Binary  & Quadratic &   9\\
%          & \multirow{2}*{Mixed}
%                    & Convex    &   2\\
%          &         & Quadratic &  118\\
%%          & \multirow{2}*{Integer}
%%                    & Convex    &  15 \\
%%          &         & Quadratic &   5 \\
%          & {General} & Quadratic &   3 \\
%\hline
\multirow{1}*{Convex (if min)}
          & Binary  & Linear    &  \textcolor{red}{5} \\[1.2 ex]
\cmidrule(lr){2-4}
\multirow{1}*{or}
          & \multirow{2}*{Mixed}
                    & Linear    &   12\\[1.2 ex]
\multirow{1}*{Concave (if max)}
          &         & Quadratic &    \textcolor{red}{6}\\[1.2 ex]
%          & General & Linear    &    \\
%\cmidrule(lr){2-4}
%          & \multirow{1}*{Integer}
%                    & Linear    &   1\\[1.2 ex]
%\hline
\cmidrule(lr){1-4}
\multirow{6}*{Quadratic}
          & \multirow{3}*{Binary}
                    & None      &   23\\[1.2 ex]
          &         & Linear    &  \textcolor{red}{91}\\[1.2 ex]
          &         & Quadratic &   5 \\[1.2 ex]
\cmidrule(lr){2-4}
          & \multirow{2}*{Mixed}
                    & Linear    &   11\\[1.2 ex]
          &         & Quadratic &    1\\[1.2 ex]
\cmidrule(lr){2-4}
          & Integer & Linear    &    2\\[1.2 ex]
\cmidrule(lr){2-4}
          & \multirow{1}*{General}
                    & Quadratic    &    1\\[1.2 ex]
%          &         & Quadratic &    \\
\hline
Total     &         &           & \textcolor{red}{319}\\
%
\bottomrule
\end{tabular}
\caption{Classification of the final set of discrete instances}
\label{tab:DD}
\end{table}

\begin{table}
 \centering
 %\scriptsize
 \setlength{\tabcolsep}{18pt}
 \renewcommand \arraystretch{1.1}
\begin{tabular}{llr}
\toprule
Obj. Fun. & Constraints & \#\\
\cmidrule(lr){1-3}
%
\multirow{2}*{Linear}    & Convex    &   13\\[1.2 ex]
                         & Quadratic &   \textcolor{red}{52}\\[1.2 ex]
\cmidrule(lr){1-3}
\multirow{1}*{Convex (if min)}
                         & Box       &   3 \\[1.2 ex]
\multirow{1}*{or}
                         & Linear    &   \textcolor{red}{16}\\[1.2 ex]
                        % & Convex    &    2\\[1.2 ex]
\multirow{1}*{Concave (if max)}
                         & Quadratic &    \textcolor{red}{11}\\[1.2 ex]
\cmidrule(lr){1-3}
\multirow{3}*{Quadratic}
                         & Linear    &   6\\[1.2 ex]
                         & Convex    &   \textcolor{red}{3}\\[1.2 ex]
                         & Quadratic &   \textcolor{red}{30}\\[1.2 ex]
\hline
Total                    &           & \textcolor{red}{134} \\
%
\bottomrule
\end{tabular}
\caption{Classification of the final set of continuous instances}
\label{tab:CC}
\end{table}

%\textcolor{red}{
%One of the nontrivial choices in our library is that we made no effort to reformulate the instances, and inserted them in the %library in the very same form as they have been provided to us by the original contributors. Section \ref{ssec:reform} is %crucial in justifying this choice, as it shows that there are several degrees of freedom to move the instances from one class %to another. Tailoring the structure of a problem to a solver is, however, a bias that we did not want to add.
%}

We now report some graphs that help in illustrating the main features
of the instances. In Figure~\ref{fig:distribution}~(left) we plot the number
of variables (horizontal axis) versus the number of constrains
(vertical axis), both in logarithmic scale. Continuous instances are
marked with ``$+$'', and discrete ones with ``$\times$''. The figure shows
that the library contains a quite diverse set of instances in
terms of number of variables and constraints. \textcolor{red}{The record on the maximal number of variables and constraints (both $\approx 1,000,000$) is set by the instances QPLIB\_8547 and QPLIB\_9008.}
Figure~\ref{fig:distribution}~(right) plots the number of nonzero elements in the gradient of the objective function and the Jacobian and the number of these nonzeros corresponding to nonlinear variables, that is, it counts the appearances of variables in objectives and constraints and how often such an appearance is in a quadratic term.

%%%%%%%%%%%%%%%%%%%%%%%%%%%%%%%%%%%
\begin{figure}\centering
  \includegraphics[width=0.50\textwidth]{pic_overview.pdf}
  \includegraphics[width=0.49\textwidth]{pic_nz.pdf}
  \caption{Distribution of number of variables and constraints of QPLIB instances
(left). Number of (nonlinear) nonzeros of QPLIB instances (right).\label{fig:distribution}}
\end{figure}

%Figure~\ref{fig:pic_var_small}, Figure~\ref{fig:pic_var_medium} and
Figure~\ref{fig:pic_var}
describes how discrete and continuous
variables are distributed within the instances. The instances are
sorted accordingly to the total number of variables.
% and, for reason of
%readability, the graph is splitted in the three figures:
%Figure~\ref{fig:pic_var_small} describes the 100 smaller instances,
%Figure~\ref{fig:pic_var_medium} the 200 medium ones, and
%Figure~\ref{fig:pic_var_large} the 67 largest ones.
For each instance we report the total number of variables with a ``$+$'', and the total number of discrete variables (binary or general integer) with a ``$\times$''. The pictures clearly show that instances with different percentages of integer and continuous variables are present in the library, and that these differences are well distributed across the whole spectrum of variable sizes.

%%%%%%%%%%%%%%%%%%%%%%%%%%%%%%%
\begin{figure}\centering
  \includegraphics[width=0.55\textwidth]{pic_var.pdf}
  \caption{Number of variables of QPLIB instances. \label{fig:pic_var}}
\end{figure}

Similarly, Figure~\ref{fig:pic_constr} (left)
%Figure~\ref{fig:pic_constr_medium}
%and Figure~\ref{fig:pic_constr_big}
describes how the number of linear and quadratic
constraints are distributed within the instances. The instances are sorted
accordingly to the total number of constraints.
% and divided in the same manner as for Figures~\ref{fig:pic_var_small}--\ref{fig:pic_var_large}.
For each instance we report the total number of constraints with a ``$+$''
and the total number of quadratic constraints
with a ``$\times$''. Also in this case, different percentages of linear and
quadratic constraints are present and well-distributed across the spectrum
of constraint sizes, although both medium- and large-size instances show a
prevalence of lower percentages of quadratic constraints. In particular,
from Figure~\ref{fig:pic_constr} (left) we learn that while the maximum number
of linear constraints exceeds 1,000,000, the maximum number of quadratic
constraints tops up at 140,000. This is, however, reasonable, considering
how quadratic constraints can, in general, be expected to be much more
computationally challenging than linear ones, especially if nonconvex.

%%%%%%%%%%%%%%%%%%%%%%%%%%%%%%%%%%%%%%%
\begin{figure}\centering
  \includegraphics[width=0.49\textwidth]{pic_constr.pdf}
  \includegraphics[width=0.49\textwidth]{pic_quad_conv_vs_nonconv.pdf}
  \caption{Number of constraints, quadratic constraints, and nonconvex quadratic constraints of QPLIB instances. \label{fig:pic_constr}}
\end{figure}

Figure~\ref{fig:pic_constr} (right) shows the instances with at least one quadratic constraint sorted according to the number of quadratic constraints (vertical axis). For each instance we report the total number of constraints with a ``$+$''
and the total number of nonconvex quadratic constraints
with a ``$\times$''.
It can be seen that the majority of instances only have nonconvex constraints. %; of those that have nonconvex ones, however, a significant fraction have no convex ones at all.




On the theme of nonconvexity, Figure~\ref{fig:pic_neg_eig} (left) focuses on
the instances with a quadratic objective function, ordered by 
 percentage of ``problematic'' eigenvalues in the Hessian $Q^0$ (vertical axis), by which
we mean \textcolor{red}{eigenvalues below $-10^{-12}$} in case of a minimization problem and \textcolor{red}{eigenvalues above $10^{-12}$} in case of a maximization problem.
The instances are mostly clustered around two values. About 25\% of the instances have a convex (if minimization) or concave (if maximization) objective function,
i.e., they have 0\% of ``problematic'' eigenvalues. Among the others, a vast
majority has around 50\% of ``problematic'' eigenvalues. However, instances
with high or low percentages of ``problematic'' eigenvalues are present, too.

\begin{figure}\centering
  \includegraphics[width=0.49\textwidth]{pic_neg_eig.pdf}
  \includegraphics[width=0.49\textwidth]{pic_density.pdf}
  \caption{``Problematic'' eigenvalues (left) and density (right) of the Hessian $Q^0$ for QPLIB instances with a quadratic objective function. \label{fig:pic_neg_eig}}
\end{figure}

Similarly, Figure~\ref{fig:pic_neg_eig} (right) shows the instances with a
quadratic objective function sorted according to the density of the
Hessian $Q^0$ (vertical axis). The majority of the instances have either
a very low or a rather high density: indeed, about 30\% of the instances
have density smaller than 5\%, and about 30\% of the instances have
density larger than 50\%. However, also intermediate values are present.


Additional details on the instance features can be found in Appendix~\ref{sec:instance_details}.

%The all set of instances can be downloaded at \url{http://qplib.zib.de/html/index.html}.
%Since, as described in Section \ref{subsec:final set} the library contains instances of different nature, the website allows to filter the instances according to the classification proposed in Section \ref{ssec:taxonomy}.

\subsection{Website}\label{subsec:website}
%Select subset or categories of instances (EXTRACT FROM THE LIBRARY A SUBSET OF INSTANCES WITH SPECIFIC CHARACTERISTICS)


The QPLIB instances are publicly accessible at the website \url{http://qplib.zib.de}, which was created by extending
scripts and tools initially developed for MINLPLib\,2~\cite{Vigerske2014}.
%
We provide all instances in GAMS (\texttt{.gms}), AMPL (\texttt{.mod}), CPLEX (\texttt{.lp})
\cite{,cplex}, and  QPLIB (\texttt{.qplib}) formats. The latter is a new format
specifically for QP instances. In comparison to more \emph{high level}
formats such as \texttt{.gms} and \texttt{.lp}, the new
format offers three main advantages: it is easier to read by a stand-alone
parser,
it typically produces smaller files, and it permits the
inclusion of two-sided inequalities without needless repetition of data.
%the order of variables and constraints is fixed.
See Appendix~\ref{sec:format} for more details.

Beyond the instances, the website provides a rich set of metadata for
each instance: the three letter problem classification (as described in \S\ref{subsec:final set}), \textcolor{red}{the contributor of the instance}, basic properties such as the number of variables and constraints of
different types, the sense and convexity/concavity of the objective function, and information on the nonzero structure of the
problem.
%
In addition, we display a visualization of the sparsity patterns of the Jacobian and the Hessian matrix of the
Lagrangian function, \textcolor{red}{if the instance size allows}.  In the plots of the Jacobian nonzero pattern, the linear and nonlinear entries are distinguished
by color.  Figure \ref{fig:sparsitypattern} shows an example for instance QPLIB\_2967.
\textcolor{red}{Finally, feasible solution points are provided for most instances.}

\begin{figure}
\centering
\includegraphics[scale=0.5]{{QPLIB_2967.jac}.png} \qquad
\includegraphics{{QPLIB_2967.hess}.png}
\caption{Example for the sparsity pattern of the Jacobian of the constraint functions (left) and
  of \textcolor{red}{the upper-right triangle of} the Hessian of the Lagrangian function (right) for instance QPLIB\_2967.
  The gradient of the objective function is displayed as the first row
  of the Jacobian matrix. \textcolor{red}{Non-constant entries are shown in red.}}
\label{fig:sparsitypattern}
\end{figure}

The entire set of instances can be explored in a searchable and sortable table of selected instance features: problem
classification, convexity of the continuous relaxation, number of (all, binary, integer) variables, (all, quadratic) constraints, nonzeros, \textcolor{red}{problematic} eigenvalues in~$Q^0$, and density of~$Q^0$.
%
Finally, a statistics page displays diagrams on the composition of the library according to different criteria: the
number of instances according to problem type, variable and constraint types, convexity, problem size, and density.
%
A file containing the relevant metadata for each instance can be downloaded in comma-separated-values (\texttt{csv}) format,
so that researchers can easily compile their own subset of instances according to these statistics.

The complete library can be downloaded as one archive, which contains the website for offline browsing and exploration.
%
In the future, we plan to extend the website by references to the literature.



%- - - - - - - - - - - - - - - - - - - - - - - - - - - - - - - - - - - -
%- - - - - - - - - - - - - - - - - - - - - - - - - - - - - - - - - - - -
%  End QPLIB-3.tex
%- - - - - - - - - - - - - - - - - - - - - - - - - - - - - - - - - - - -
%- - - - - - - - - - - - - - - - - - - - - - - - - - - - - - - - - - - -


%- - - - - - - - - - - - - - - - - - - - - - - - - - - - - - - - - - - -

%- - - - - - - - - - - - - - - - - - - - - - - - - - - - - - - - - - - -

%- - - - - - - - - - - - - - - - - - - - - - - - - - - - - - - - - - - -
%- - - - - - - - - - - - - - - - - - - - - - - - - - - - - - - - - - - -
%  QPLIB-4.tex
%- - - - - - - - - - - - - - - - - - - - - - - - - - - - - - - - - - - -
%- - - - - - - - - - - - - - - - - - - - - - - - - - - - - - - - - - - -

\section{Software tools}\label{subsec:tools}

%- - - - - - - - - - - - - - - - - - - - - - - - - - - - - - - - - - - -
\subsection{instance translator}
GAMS--LP--QPFORMAT
\framebox{TASK X : write }


%- - - - - - - - - - - - - - - - - - - - - - - - - - - - - - - - - - - -
\subsection{code that computes the features of an instance}
\framebox{TASK X : write }

%- - - - - - - - - - - - - - - - - - - - - - - - - - - - - - - - - - - -
\subsection{code that selects subsets of instances}
\framebox{TASK X : write }

%- - - - - - - - - - - - - - - - - - - - - - - - - - - - - - - - - - - -
\subsection{website, instance collector}
Select subset or categories of instances (EXTRACT FROM THE LIBRARY A SUBSET OF INSTANCES WITH SPECIFIC CHARACTERISTICS)
\framebox{TASK X : write }

%- - - - - - - - - - - - - - - - - - - - - - - - - - - - - - - - - - - -
\subsection{testing environment}
RUN GAMS USING A SUBSET OF SOLVERS
\framebox{TASK X : write }

%- - - - - - - - - - - - - - - - - - - - - - - - - - - - - - - - - - - -
%- - - - - - - - - - - - - - - - - - - - - - - - - - - - - - - - - - - -
%  End QPLIB-4.tex
%- - - - - - - - - - - - - - - - - - - - - - - - - - - - - - - - - - - -
%- - - - - - - - - - - - - - - - - - - - - - - - - - - - - - - - - - - -


%- - - - - - - - - - - - - - - - - - - - - - - - - - - - - - - - - - - -


%- - - - - - - - - - - - - - - - - - - - - - - - - - - - - - - - - - - -

\section{Final Remarks}\label{sec:conclusions}

This paper described the first comprehensive library of instances for Quadratic Programming (QP). Since QP comprises different and ``varied'' categories of problems, we proposed a classification and we briefly discussed the main classes of solution methods for QP.
We then described the steps of the adopted process used to filter the gathered instances  in order to build the new library. Our design goals were to build a library which is computationally challenging and as broad as possible, i.e., it represents the largest possible categories of QP problems, while remaining of manageable size. We also proposed a stand-alone QP format that is intended
for the convenient exchange and use of our QP instances.

We want to stress once again that we intentionally avoided to perform a
computational comparison of the performances of different solution methods or
solver implementations. Our goal was instead to provide a broad test bed of
instances for researchers and practitioners in the field. This new library will
hopefully serve as a point of reference to inspire and test new ideas and
algorithms for QP problems.

Finally, we want to emphasize that this QP collection can only be a snapshot of
the types of problems that researchers and practitioners have worked on in the
past.  With the growing interest in this area, we hope that new applications and
instances will become available and that the library can be extended dynamically
in the future.


%- - - - - - - - - - - - - - - - - - - - - - - - - - - - - - - - - - - -


%- - - - - - - - - - - - - - - - - - - - - - - - - - - - - - - - - - - -
%- - - - - - - - - - - - - - - - - - - - - - - - - - - - - - - - - - - -
\section{Acknowledgements}

%- - - - - - - - - - - - - - - - - - - - - - - - - - - - - - - - - - - -
%- - - - - - - - - - - - - - - - - - - - - - - - - - - - - - - - - - - -

\bibliography{biblio}

\ifMPC
 \bibliographystyle{spmpsci}
\else
 \bibliographystyle{plain}
\fi

%- - - - - - - - - - - - - - - - - - - - - - - - - - - - - - - - - - - -
%- - - - - - - - - - - - - - - - - - - - - - - - - - - - - - - - - - - -
\appendix

%- - - - - - - - - - - - - - - - - - - - - - - - - - - - - - - - - - - -
%- - - - - - - - - - - - - - - - - - - - - - - - - - - - - - - - - - - -
%  Appendix.tex
%- - - - - - - - - - - - - - - - - - - - - - - - - - - - - - - - - - - -
%- - - - - - - - - - - - - - - - - - - - - - - - - - - - - - - - - - - -

\section{Instance details}\label{sec:instance_details}


In this Appendix we provide detailed data on all the instances of the final library. This is done in Tables \ref{tab:A1}--\ref{tab:A10} for \emph{discrete} instances (*\{B,M,I,G\}*) and in Tables \ref{tab:B1}--\ref{tab:B5} for \emph{continuous} ones (*C*). In the former, the features of the instances are described by three sets of columns. The first (``\% n.e.'') describes the objective function by reporting the fraction of eigenvalues of $Q^0$ that are negative: a positive number implies that $Q^0$ is not SDP (hence, the instance is a Q**), ``0.0'' implies that $Q^0$ is SDP (hence, the instance is a C**), a blank implies that $Q^0 = 0$, i.e., the objective function is linear (hence, the instance is a L**). 
The second (``\% dens'') describes the density of the $Q^0$ matrix; a blank implies that the corresponding instance has a linear objective function.
For both columns (``\% n.e.'' and ``\% d.'') we report $0.1$ in case an instance has smaller positive value.
The following three columns describe the variables by reporting the number of binary ones (``\# b.''), general integer ones (``\# i.''), and continuous ones (``\# c.''). Finally, the last three columns describe the constraints reporting the number of linear ones (``\# l.''), nonconvex quadratic ones (``\# q.''), and convex quadratic ones (``\# c.''). Tables \ref{tab:B1}--\ref{tab:B5} are similarly structured except that all variables are continuous, and hence only one column is present.



{\tiny
\begin{longtable}{lrrrrrrrrrrrr}
% \centering
%\scriptsize
%\setlength{\tabcolsep}{1pt}
%\renewcommand \arraystretch{1}
%\begin{tabular}{lrrrrrrrrrrrr}
\toprule

	&		\multicolumn{2}{c}{$Q^0$}		&	&	\multicolumn{3}{c}{Variables}					&	&	\multicolumn{4}{c}{Constraints}							\\
%\cmidrule(lr){5-7} \cmidrule(lr){9-12}							 														
% &	 	&	 	&	&	vars 	&	 &	 	&	&	cons &	 	& 	&	\\	
%name	&	\% n.e.	&	\% dens	&	&	\# bin	&	\# int 	&	\# cont 	&	&	\# lin 	&	\# quad 	&	\# conv 	&	\# box	\\[2 ex]
name	&	\% n.e.	&	\% d.	&	&	\# b.	&	\# i. 	&	\# c. 	&	&	\# l. 	&	\# q. 	&	\# c. 	&	\# b.	\\[2 ex]

\endhead 														

\texttt{ 	qplib\_0018	}	&	48.0	&	100.0	&	&	0	&	0	&	50	&	&	1	&	0	&	0	&	0	\\
\texttt{ 	qplib\_0031	}	&	18.4	&	25.0	&	&	30	&	0	&	30	&	&	32	&	0	&	0	&	0	\\
\texttt{ 	qplib\_0032	}	&	25.0	&	25.0	&	&	50	&	0	&	50	&	&	52	&	0	&	0	&	0	\\
\texttt{ 	qplib\_0067	}	&	47.5	&	88.9	&	&	80	&	0	&	0	&	&	1	&	0	&	0	&	0	\\
\texttt{ 	qplib\_0343	}	&	48.0	&	100.0	&	&	0	&	0	&	50	&	&	1	&	0	&	0	&	50	\\
\texttt{ 	qplib\_0633	}	&	58.7	&	98.7	&	&	75	&	0	&	0	&	&	1	&	0	&	0	&	0	\\
\texttt{ 	qplib\_0678	}	&		&		&	&	9600	&	0	&	5537	&	&	7457	&	960	&	480	&	737	\\
\texttt{ 	qplib\_0681	}	&		&		&	&	72	&	0	&	143	&	&	419	&	48	&	0	&	143	\\
\texttt{ 	qplib\_0682	}	&		&		&	&	71	&	0	&	190	&	&	501	&	96	&	0	&	190	\\
\texttt{ 	qplib\_0684	}	&		&		&	&	101	&	0	&	260	&	&	815	&	128	&	0	&	260	\\
\texttt{ 	qplib\_0685	}	&		&		&	&	256	&	0	&	519	&	&	1603	&	192	&	0	&	519	\\
\texttt{ 	qplib\_0686	}	&		&		&	&	692	&	0	&	1512	&	&	4440	&	640	&	0	&	1512	\\
\texttt{ 	qplib\_0687	}	&		&		&	&	672	&	0	&	1651	&	&	4875	&	800	&	0	&	1651	\\
{\tt 	qplib\_0688	}	&		&		&	&	1964	&	0	&	3824	&	&	20568	&	1600	&	0	&	3824	\\
\texttt{ 	qplib\_0689	}	&		&		&	&	756	&	0	&	1112	&	&	9800	&	288	&	0	&	1112	\\
\texttt{ 	qplib\_0690	}	&		&		&	&	6428	&	0	&	10048	&	&	112400	&	3200	&	0	&	10048	\\
\texttt{ 	qplib\_0696	}	&		&		&	&	187	&	0	&	207	&	&	390	&	33	&	0	&	207	\\
\texttt{ 	qplib\_0698	}	&		&		&	&	55	&	0	&	63	&	&	126	&	15	&	0	&	63	\\
\texttt{ 	qplib\_0752	}	&	50.0	&	10.0	&	&	250	&	0	&	0	&	&	1	&	0	&	0	&	0	\\
\texttt{ 	qplib\_0911	}	&	44.0	&	50.5	&	&	0	&	0	&	50	&	&	0	&	50	&	50	&	50	\\
\texttt{ 	qplib\_0975	}	&	50.0	&	50.7	&	&	0	&	0	&	50	&	&	0	&	10	&	10	&	50	\\
\texttt{ 	qplib\_1055	}	&	50.0	&	100.0	&	&	0	&	0	&	40	&	&	0	&	20	&	19	&	40	\\
\texttt{ 	qplib\_1143	}	&	50.0	&	97.2	&	&	0	&	0	&	40	&	&	4	&	20	&	0	&	40	\\
\texttt{ 	qplib\_1157	}	&	25.0	&	94.5	&	&	0	&	0	&	40	&	&	8	&	1	&	1	&	40	\\
\texttt{ 	qplib\_1353	}	&	26.0	&	95.8	&	&	0	&	0	&	50	&	&	5	&	1	&	1	&	50	\\
\texttt{ 	qplib\_1423	}	&	75.0	&	95.4	&	&	0	&	0	&	40	&	&	4	&	20	&	20	&	40	\\
\texttt{ 	qplib\_1437	}	&	50.0	&	95.6	&	&	0	&	0	&	50	&	&	10	&	1	&	1	&	50	\\
\texttt{ 	qplib\_1451	}	&	50.0	&	49.1	&	&	0	&	0	&	60	&	&	6	&	60	&	0	&	60	\\
\texttt{ 	qplib\_1493	}	&	50.0	&	97.4	&	&	0	&	0	&	40	&	&	4	&	1	&	0	&	40	\\
\texttt{ 	qplib\_1507	}	&	26.7	&	95.8	&	&	0	&	0	&	30	&	&	3	&	30	&	30	&	30	\\
\texttt{ 	qplib\_1535	}	&	50.0	&	94.4	&	&	0	&	0	&	60	&	&	6	&	60	&	60	&	60	\\
\texttt{ 	qplib\_1619	}	&	50.0	&	95.6	&	&	0	&	0	&	50	&	&	5	&	25	&	25	&	50	\\
\texttt{ 	qplib\_1661	}	&	50.0	&	95.4	&	&	0	&	0	&	60	&	&	12	&	1	&	1	&	60	\\
\texttt{ 	qplib\_1675	}	&	51.7	&	48.8	&	&	0	&	0	&	60	&	&	12	&	1	&	0	&	60	\\
\texttt{ 	qplib\_1703	}	&	51.7	&	98.0	&	&	0	&	0	&	60	&	&	6	&	30	&	0	&	60	\\
\texttt{ 	qplib\_1745	}	&	50.0	&	48.8	&	&	0	&	0	&	50	&	&	5	&	50	&	0	&	50	\\
\texttt{ 	qplib\_1773	}	&	50.0	&	94.8	&	&	0	&	0	&	60	&	&	6	&	1	&	1	&	60	\\
\texttt{ 	qplib\_1886	}	&	50.0	&	50.0	&	&	0	&	0	&	50	&	&	0	&	50	&	0	&	50	\\
\texttt{ 	qplib\_1913	}	&	50.0	&	25.0	&	&	0	&	0	&	48	&	&	0	&	48	&	0	&	48	\\
\texttt{ 	qplib\_1922	}	&	50.0	&	49.6	&	&	0	&	0	&	30	&	&	0	&	60	&	0	&	30	\\
\texttt{ 	qplib\_1931	}	&	50.0	&	49.9	&	&	0	&	0	&	40	&	&	0	&	40	&	0	&	40	\\
\texttt{ 	qplib\_1940	}	&	50.0	&	25.0	&	&	0	&	0	&	48	&	&	0	&	96	&	0	&	48	\\
\texttt{ 	qplib\_1967	}	&	50.0	&	99.8	&	&	0	&	0	&	50	&	&	0	&	75	&	0	&	50	\\
\texttt{ 	qplib\_1976	}	&	38.2	&	7.0	&	&	152	&	0	&	0	&	&	136	&	16	&	0	&	0	\\
\texttt{ 	qplib\_2017	}	&	39.3	&	5.5	&	&	252	&	0	&	0	&	&	231	&	21	&	0	&	0	\\
\texttt{ 	qplib\_2022	}	&	38.6	&	5.3	&	&	275	&	0	&	0	&	&	253	&	22	&	0	&	0	\\
\texttt{ 	qplib\_2029	}	&	40.2	&	5.1	&	&	299	&	0	&	0	&	&	276	&	23	&	0	&	0	\\
\texttt{ 	qplib\_2036	}	&	39.2	&	4.9	&	&	324	&	0	&	0	&	&	300	&	24	&	0	&	0	\\
\texttt{ 	qplib\_2047	}	&		&		&	&	136	&	0	&	0	&	&	2040	&	17	&	0	&	0	\\
\texttt{ 	qplib\_2055	}	&		&		&	&	153	&	0	&	0	&	&	2448	&	18	&	0	&	0	\\
\texttt{ 	qplib\_2060	}	&		&		&	&	171	&	0	&	0	&	&	2907	&	19	&	0	&	0	\\
\texttt{ 	qplib\_2067	}	&		&		&	&	190	&	0	&	0	&	&	3420	&	20	&	0	&	0	\\
\texttt{ 	qplib\_2073	}	&		&		&	&	210	&	0	&	0	&	&	3990	&	21	&	0	&	0	\\
\texttt{ 	qplib\_2077	}	&		&		&	&	231	&	0	&	0	&	&	4620	&	22	&	0	&	0	\\
\texttt{ 	qplib\_2085	}	&		&		&	&	253	&	0	&	0	&	&	5313	&	23	&	0	&	0	\\
\texttt{ 	qplib\_2087	}	&		&		&	&	276	&	0	&	0	&	&	6072	&	24	&	0	&	0	\\
\texttt{ 	qplib\_2096	}	&		&		&	&	300	&	0	&	0	&	&	6900	&	25	&	0	&	0	\\
\texttt{ 	qplib\_2165	}	&		&		&	&	683	&	0	&	1376	&	&	1366	&	683	&	0	&	0	\\
\texttt{ 	qplib\_2166	}	&		&		&	&	345	&	0	&	697	&	&	690	&	345	&	0	&	0	\\
\texttt{ 	qplib\_2167	}	&		&		&	&	61	&	0	&	131	&	&	122	&	61	&	0	&	0	\\
\texttt{ 	qplib\_2168	}	&		&		&	&	214	&	0	&	438	&	&	428	&	214	&	0	&	0	\\
\texttt{ 	qplib\_2169	}	&		&		&	&	297	&	0	&	608	&	&	594	&	297	&	0	&	0	\\
\texttt{ 	qplib\_2170	}	&		&		&	&	351	&	0	&	736	&	&	702	&	351	&	0	&	0	\\
\texttt{ 	qplib\_2171	}	&		&		&	&	150	&	0	&	305	&	&	300	&	150	&	0	&	0	\\
\texttt{ 	qplib\_2173	}	&		&		&	&	215	&	0	&	436	&	&	430	&	215	&	0	&	0	\\
\texttt{ 	qplib\_2174	}	&		&		&	&	768	&	0	&	1545	&	&	1536	&	768	&	0	&	0	\\
\texttt{ 	qplib\_2181	}	&		&		&	&	90	&	0	&	190	&	&	180	&	90	&	0	&	0	\\
\texttt{ 	qplib\_2187	}	&		&		&	&	90	&	0	&	195	&	&	180	&	90	&	0	&	0	\\
\texttt{ 	qplib\_2192	}	&		&		&	&	90	&	0	&	200	&	&	180	&	90	&	0	&	0	\\
\texttt{ 	qplib\_2195	}	&		&		&	&	90	&	0	&	205	&	&	180	&	90	&	0	&	0	\\
\texttt{ 	qplib\_2202	}	&		&		&	&	90	&	0	&	185	&	&	180	&	90	&	0	&	0	\\
\texttt{ 	qplib\_2203	}	&		&		&	&	100	&	0	&	205	&	&	200	&	100	&	0	&	0	\\
\texttt{ 	qplib\_2204	}	&		&		&	&	110	&	0	&	225	&	&	220	&	110	&	0	&	0	\\
\texttt{ 	qplib\_2205	}	&		&		&	&	958	&	0	&	1926	&	&	1916	&	958	&	0	&	0	\\
\texttt{ 	qplib\_2206	}	&		&		&	&	194	&	0	&	421	&	&	388	&	194	&	0	&	0	\\
\texttt{ 	qplib\_2315	}	&	44.8	&	7.5	&	&	595	&	0	&	0	&	&	13090	&	0	&	0	&	0	\\
\texttt{ 	qplib\_2353	}	&	50.0	&	23.8	&	&	147	&	0	&	93	&	&	2240	&	0	&	0	&	93	\\
\texttt{ 	qplib\_2357	}	&	50.0	&	7.9	&	&	240	&	0	&	0	&	&	2240	&	0	&	0	&	0	\\
\texttt{ 	qplib\_2359	}	&	47.1	&	3.8	&	&	306	&	0	&	0	&	&	3264	&	0	&	0	&	0	\\
\texttt{ 	qplib\_2416	}	&		&		&	&	0	&	0	&	25	&	&	153	&	533	&	46	&	25	\\
\texttt{ 	qplib\_2430	}	&		&		&	&	0	&	0	&	125	&	&	27	&	65	&	0	&	125	\\
\texttt{ 	qplib\_2445	}	&		&		&	&	0	&	0	&	143	&	&	14	&	66	&	0	&	143	\\
\texttt{ 	qplib\_2456	}	&		&		&	&	0	&	0	&	5477	&	&	4131	&	1369	&	1369	&	0	\\
\texttt{ 	qplib\_2468	}	&		&		&	&	0	&	0	&	14885	&	&	11203	&	3721	&	3721	&	0	\\
\texttt{ 	qplib\_2480	}	&		&		&	&	0	&	0	&	399	&	&	199	&	201	&	0	&	398	\\
\texttt{ 	qplib\_2482	}	&		&		&	&	0	&	0	&	1806	&	&	1418	&	361	&	0	&	0	\\
\texttt{ 	qplib\_2483	}	&		&		&	&	0	&	0	&	760	&	&	40	&	240	&	0	&	760	\\
\texttt{ 	qplib\_2492	}	&	35.8	&	86.3	&	&	196	&	0	&	0	&	&	28	&	0	&	0	&	0	\\
\texttt{ 	qplib\_2505	}	&		&		&	&	0	&	0	&	1039	&	&	302	&	480	&	0	&	1039	\\
\texttt{ 	qplib\_2512	}	&	46.0	&	77.4	&	&	100	&	0	&	0	&	&	20	&	0	&	0	&	0	\\
\texttt{ 	qplib\_2519	}	&		&		&	&	0	&	0	&	4806	&	&	3802	&	961	&	961	&	0	\\
\texttt{ 	qplib\_2540	}	&		&		&	&	0	&	0	&	498	&	&	341	&	210	&	0	&	498	\\
\texttt{ 	qplib\_2546	}	&	0.0	&	0.2	&	&	0	&	0	&	1015	&	&	592	&	400	&	0	&	0	\\
\texttt{ 	qplib\_2590	}	&		&		&	&	0	&	0	&	25	&	&	93	&	401	&	35	&	25	\\
\texttt{ 	qplib\_2626	}	&		&		&	&	0	&	0	&	22327	&	&	14763	&	3721	&	3721	&	0	\\
\texttt{ 	qplib\_2635	}	&		&		&	&	0	&	0	&	176	&	&	0	&	1154	&	384	&	0	\\
\texttt{ 	qplib\_2650	}	&		&		&	&	0	&	0	&	1110	&	&	228	&	904	&	27	&	1110	\\
\texttt{ 	qplib\_2658	}	&		&		&	&	0	&	0	&	184	&	&	57	&	133	&	23	&	184	\\
\texttt{ 	qplib\_2676	}	&		&		&	&	0	&	0	&	1445	&	&	1095	&	361	&	361	&	0	\\
\texttt{ 	qplib\_2693	}	&		&		&	&	0	&	0	&	791	&	&	183	&	631	&	20	&	791	\\
\texttt{ 	qplib\_2696	}	&	1.8	&	0.1	&	&	0	&	0	&	3500	&	&	1995	&	1500	&	0	&	0	\\
\texttt{ 	qplib\_2698	}	&		&		&	&	0	&	0	&	196	&	&	36	&	11	&	0	&	196	\\
\texttt{ 	qplib\_2702	}	&	4.7	&	1.2	&	&	259	&	0	&	1	&	&	212	&	0	&	0	&	0	\\
\texttt{ 	qplib\_2703	}	&		&		&	&	0	&	0	&	799	&	&	399	&	401	&	1	&	798	\\
\texttt{ 	qplib\_2707	}	&		&		&	&	0	&	0	&	634	&	&	151	&	466	&	42	&	586	\\
\texttt{ 	qplib\_2708	}	&		&		&	&	108	&	0	&	526	&	&	327	&	30	&	0	&	526	\\
\texttt{ 	qplib\_2712	}	&	50.0	&	100.0	&	&	0	&	0	&	200	&	&	1	&	0	&	0	&	200	\\
\texttt{ 	qplib\_2714	}	&		&		&	&	0	&	0	&	352	&	&	301	&	298	&	0	&	0	\\
\texttt{ 	qplib\_2733	}	&	40.2	&	89.2	&	&	324	&	0	&	0	&	&	36	&	0	&	0	&	0	\\
\texttt{ 	qplib\_2738	}	&		&		&	&	0	&	0	&	199	&	&	99	&	101	&	1	&	198	\\
\texttt{ 	qplib\_2758	}	&		&		&	&	0	&	0	&	303	&	&	139	&	112	&	0	&	303	\\
\texttt{ 	qplib\_2761	}	&	50.0	&	100.0	&	&	0	&	0	&	500	&	&	1	&	0	&	0	&	500	\\
\texttt{ 	qplib\_2784	}	&		&		&	&	0	&	0	&	4501	&	&	3680	&	900	&	900	&	0	\\
\texttt{ 	qplib\_2819	}	&		&		&	&	0	&	0	&	334	&	&	24	&	132	&	0	&	334	\\
\texttt{ 	qplib\_2823	}	&		&		&	&	0	&	0	&	390	&	&	103	&	283	&	0	&	374	\\
\texttt{ 	qplib\_2834	}	&		&		&	&	0	&	0	&	156	&	&	14	&	72	&	0	&	156	\\
\texttt{ 	qplib\_2862	}	&		&		&	&	0	&	0	&	40501	&	&	32640	&	8100	&	8100	&	0	\\
\texttt{ 	qplib\_2880	}	&	48.8	&	90.4	&	&	625	&	0	&	0	&	&	50	&	0	&	0	&	0	\\
\texttt{ 	qplib\_2881	}	&		&		&	&	0	&	0	&	1512	&	&	0	&	720	&	0	&	0	\\
\texttt{ 	qplib\_2882	}	&		&		&	&	56	&	0	&	88	&	&	257	&	16	&	0	&	16	\\
\texttt{ 	qplib\_2894	}	&		&		&	&	0	&	0	&	17	&	&	55	&	154	&	15	&	17	\\
\texttt{ 	qplib\_2935	}	&		&		&	&	72	&	0	&	108	&	&	325	&	18	&	0	&	18	\\
\texttt{ 	qplib\_2957	}	&	45.7	&	60.4	&	&	484	&	0	&	0	&	&	44	&	0	&	0	&	0	\\
\texttt{ 	qplib\_2958	}	&		&		&	&	42	&	0	&	70	&	&	197	&	14	&	0	&	14	\\
\texttt{ 	qplib\_2967	}	&	47.4	&	5.0	&	&	0	&	0	&	38	&	&	1	&	190	&	0	&	19	\\
\texttt{ 	qplib\_2981	}	&	0.0	&	0.1	&	&	0	&	0	&	2015	&	&	1192	&	800	&	0	&	0	\\
\texttt{ 	qplib\_2987	}	&		&		&	&	0	&	0	&	208	&	&	114	&	90	&	0	&	208	\\
\texttt{ 	qplib\_2993	}	&		&		&	&	0	&	0	&	266	&	&	235	&	84	&	0	&	266	\\
\texttt{ 	qplib\_3029	}	&		&		&	&	0	&	0	&	5767	&	&	3783	&	961	&	961	&	0	\\
\texttt{ 	qplib\_3034	}	&		&		&	&	0	&	0	&	780	&	&	40	&	240	&	0	&	780	\\
\texttt{ 	qplib\_3049	}	&	0.6	&	0.1	&	&	0	&	0	&	7000	&	&	3995	&	3000	&	0	&	0	\\
\texttt{ 	qplib\_3060	}	&	0.3	&	0.1	&	&	48	&	0	&	792	&	&	1192	&	0	&	0	&	0	\\
\texttt{ 	qplib\_3080	}	&	0.0	&	0.1	&	&	0	&	0	&	4015	&	&	2392	&	1600	&	0	&	0	\\
\texttt{ 	qplib\_3083	}	&		&		&	&	0	&	0	&	243	&	&	107	&	126	&	0	&	243	\\
\texttt{ 	qplib\_3088	}	&		&		&	&	0	&	0	&	3601	&	&	2780	&	900	&	900	&	0	\\
\texttt{ 	qplib\_3089	}	&		&		&	&	0	&	0	&	132	&	&	12	&	72	&	0	&	132	\\
\texttt{ 	qplib\_3105	}	&		&		&	&	0	&	0	&	18606	&	&	14802	&	3721	&	3721	&	0	\\
\texttt{ 	qplib\_3120	}	&		&		&	&	0	&	0	&	662	&	&	40	&	204	&	0	&	662	\\
\texttt{ 	qplib\_3122	}	&	0.0	&	0.1	&	&	17136	&	0	&	3988	&	&	36703	&	0	&	0	&	776	\\
\texttt{ 	qplib\_3147	}	&		&		&	&	0	&	0	&	419	&	&	32	&	108	&	0	&	419	\\
\texttt{ 	qplib\_3170	}	&		&		&	&	0	&	0	&	660	&	&	40	&	160	&	0	&	660	\\
\texttt{ 	qplib\_3177	}	&		&		&	&	0	&	0	&	1599	&	&	799	&	801	&	0	&	1598	\\
\texttt{ 	qplib\_3181	}	&		&		&	&	84	&	0	&	308	&	&	180	&	16	&	0	&	308	\\
\texttt{ 	qplib\_3185	}	&		&		&	&	0	&	0	&	18001	&	&	14560	&	3600	&	3600	&	0	\\
\texttt{ 	qplib\_3192	}	&		&		&	&	0	&	0	&	479	&	&	32	&	145	&	0	&	479	\\
\texttt{ 	qplib\_3225	}	&		&		&	&	0	&	0	&	136	&	&	14	&	66	&	0	&	136	\\
\texttt{ 	qplib\_3240	}	&		&		&	&	0	&	0	&	516	&	&	187	&	220	&	0	&	516	\\
\texttt{ 	qplib\_3247	}	&		&		&	&	0	&	0	&	361	&	&	322	&	156	&	0	&	0	\\
\texttt{ 	qplib\_3279	}	&		&		&	&	56	&	0	&	251	&	&	148	&	16	&	0	&	251	\\
\texttt{ 	qplib\_3297	}	&	0.0	&	0.1	&	&	0	&	0	&	8015	&	&	4792	&	3200	&	0	&	0	\\
\texttt{ 	qplib\_3307	}	&	48.1	&	61.6	&	&	256	&	0	&	0	&	&	32	&	0	&	0	&	0	\\
\texttt{ 	qplib\_3312	}	&		&		&	&	0	&	0	&	41406	&	&	33002	&	8281	&	0	&	0	\\
\texttt{ 	qplib\_3318	}	&		&		&	&	0	&	0	&	25	&	&	93	&	381	&	9	&	25	\\
\texttt{ 	qplib\_3326	}	&	2.6	&	0.1	&	&	0	&	0	&	1750	&	&	995	&	750	&	0	&	0	\\
\texttt{ 	qplib\_3334	}	&		&		&	&	0	&	0	&	715	&	&	40	&	210	&	0	&	715	\\
\texttt{ 	qplib\_3337	}	&		&		&	&	0	&	0	&	297	&	&	0	&	198	&	0	&	199	\\
\texttt{ 	qplib\_3338	}	&		&		&	&	0	&	0	&	320	&	&	26	&	110	&	0	&	320	\\
\texttt{ 	qplib\_3347	}	&	51.8	&	85.8	&	&	676	&	0	&	0	&	&	52	&	0	&	0	&	0	\\
\texttt{ 	qplib\_3358	}	&		&		&	&	0	&	0	&	158	&	&	66	&	106	&	5	&	158	\\
\texttt{ 	qplib\_3361	}	&	45.1	&	31.3	&	&	1024	&	0	&	0	&	&	64	&	0	&	0	&	0	\\
\texttt{ 	qplib\_3369	}	&		&		&	&	0	&	0	&	485	&	&	32	&	116	&	0	&	485	\\
\texttt{ 	qplib\_3380	}	&	3.5	&	0.1	&	&	8904	&	0	&	0	&	&	823	&	0	&	0	&	0	\\
\texttt{ 	qplib\_3385	}	&		&		&	&	0	&	0	&	155	&	&	77	&	60	&	0	&	155	\\
\texttt{ 	qplib\_3387	}	&		&		&	&	0	&	0	&	170	&	&	18	&	65	&	0	&	170	\\
\texttt{ 	qplib\_3402	}	&	47.3	&	81.5	&	&	144	&	0	&	0	&	&	24	&	0	&	0	&	0	\\
\texttt{ 	qplib\_3413	}	&	50.0	&	9.1	&	&	400	&	0	&	0	&	&	40	&	0	&	0	&	0	\\
\texttt{ 	qplib\_3416	}	&		&		&	&	0	&	0	&	424	&	&	32	&	96	&	0	&	424	\\
\texttt{ 	qplib\_3496	}	&		&		&	&	200	&	56	&	72	&	&	623	&	64	&	8	&	64	\\
\texttt{ 	qplib\_3502	}	&		&		&	&	10920	&	0	&	2090	&	&	209	&	3130	&	0	&	0	\\
\texttt{ 	qplib\_3505	}	&		&		&	&	201	&	0	&	603	&	&	605	&	2	&	0	&	201	\\
\texttt{ 	qplib\_3506	}	&	50.5	&	0.8	&	&	496	&	0	&	0	&	&	0	&	0	&	0	&	0	\\
\texttt{ 	qplib\_3508	}	&		&		&	&	2450	&	0	&	891	&	&	99	&	1332	&	0	&	0	\\
\texttt{ 	qplib\_3510	}	&		&		&	&	105	&	0	&	919	&	&	4568	&	21	&	0	&	786	\\
\texttt{ 	qplib\_3511	}	&		&		&	&	2450	&	0	&	3292	&	&	4950	&	1283	&	0	&	0	\\
\texttt{ 	qplib\_3512	}	&		&		&	&	72	&	0	&	119	&	&	403	&	24	&	0	&	119	\\
\texttt{ 	qplib\_3513	}	&		&		&	&	123	&	0	&	1897	&	&	2569	&	763	&	0	&	504	\\
\texttt{ 	qplib\_3514	}	&		&		&	&	15	&	0	&	1800	&	&	960	&	900	&	0	&	0	\\
\texttt{ 	qplib\_3515	}	&		&		&	&	352	&	0	&	382	&	&	720	&	48	&	0	&	382	\\
\texttt{ 	qplib\_3522	}	&		&		&	&	42	&	0	&	588	&	&	212	&	42	&	0	&	0	\\
\texttt{ 	qplib\_3523	}	&	50.0	&	13.2	&	&	155	&	0	&	27	&	&	1456	&	0	&	0	&	27	\\
\texttt{ 	qplib\_3524	}	&		&		&	&	132	&	0	&	949	&	&	3165	&	192	&	0	&	697	\\
\texttt{ 	qplib\_3525	}	&	47.6	&	0.1	&	&	0	&	1662	&	87	&	&	52	&	39	&	0	&	1710	\\
\texttt{ 	qplib\_3529	}	&		&		&	&	38	&	0	&	1488	&	&	1580	&	544	&	0	&	944	\\
\texttt{ 	qplib\_3533	}	&		&		&	&	240	&	0	&	143	&	&	176	&	25	&	0	&	29	\\
\texttt{ 	qplib\_3547	}	&	0.0	&	0.1	&	&	462	&	0	&	1536	&	&	3137	&	0	&	0	&	0	\\
\texttt{ 	qplib\_3549	}	&		&		&	&	650	&	0	&	1033	&	&	1326	&	583	&	0	&	0	\\
\texttt{ 	qplib\_3554	}	&	13.9	&	23.3	&	&	14	&	0	&	370	&	&	556	&	0	&	0	&	0	\\
\texttt{ 	qplib\_3562	}	&		&		&	&	7	&	56	&	0	&	&	35	&	7	&	0	&	56	\\
\texttt{ 	qplib\_3565	}	&	50.4	&	1.4	&	&	276	&	0	&	0	&	&	0	&	0	&	0	&	0	\\
\texttt{ 	qplib\_3580	}	&		&		&	&	108	&	0	&	24	&	&	45	&	18	&	0	&	0	\\
\texttt{ 	qplib\_3582	}	&		&		&	&	184	&	0	&	32	&	&	60	&	24	&	0	&	0	\\
\texttt{ 	qplib\_3584	}	&	44.0	&	8.1	&	&	528	&	0	&	0	&	&	10912	&	0	&	0	&	0	\\
\texttt{ 	qplib\_3587	}	&	50.0	&	12.7	&	&	240	&	0	&	0	&	&	46	&	0	&	0	&	0	\\
\texttt{ 	qplib\_3588	}	&		&		&	&	600	&	0	&	392	&	&	49	&	584	&	0	&	0	\\
\texttt{ 	qplib\_3592	}	&	50.0	&	0.3	&	&	225	&	0	&	225	&	&	255	&	0	&	0	&	0	\\
\texttt{ 	qplib\_3596	}	&		&		&	&	104	&	0	&	921	&	&	1054	&	132	&	0	&	370	\\
\texttt{ 	qplib\_3600	}	&		&		&	&	112	&	0	&	16	&	&	45	&	12	&	0	&	0	\\
\texttt{ 	qplib\_3605	}	&		&		&	&	160	&	0	&	1076	&	&	4315	&	192	&	0	&	818	\\
\texttt{ 	qplib\_3614	}	&	50.0	&	12.7	&	&	210	&	0	&	0	&	&	44	&	0	&	0	&	0	\\
\texttt{ 	qplib\_3620	}	&		&		&	&	187	&	0	&	3285	&	&	4071	&	1344	&	0	&	943	\\
\texttt{ 	qplib\_3621	}	&		&		&	&	109	&	0	&	1655	&	&	2213	&	665	&	0	&	432	\\
\texttt{ 	qplib\_3622	}	&		&		&	&	25	&	0	&	2000	&	&	1040	&	1000	&	0	&	0	\\
\texttt{ 	qplib\_3624	}	&		&		&	&	40	&	0	&	6400	&	&	3280	&	3200	&	0	&	0	\\
\texttt{ 	qplib\_3625	}	&		&		&	&	46	&	0	&	598	&	&	191	&	46	&	0	&	0	\\
\texttt{ 	qplib\_3631	}	&		&		&	&	750	&	0	&	143	&	&	210	&	25	&	0	&	29	\\
\texttt{ 	qplib\_3642	}	&	49.8	&	0.4	&	&	1035	&	0	&	0	&	&	0	&	0	&	0	&	0	\\
\texttt{ 	qplib\_3643	}	&		&		&	&	216	&	72	&	72	&	&	825	&	68	&	18	&	80	\\
\texttt{ 	qplib\_3645	}	&		&		&	&	101	&	0	&	302	&	&	304	&	2	&	1	&	101	\\
\texttt{ 	qplib\_3646	}	&		&		&	&	20	&	0	&	2000	&	&	1050	&	1000	&	0	&	0	\\
\texttt{ 	qplib\_3648	}	&		&		&	&	40	&	0	&	680	&	&	306	&	40	&	0	&	0	\\
\texttt{ 	qplib\_3650	}	&	50.5	&	0.5	&	&	946	&	0	&	0	&	&	0	&	0	&	0	&	0	\\
\texttt{ 	qplib\_3651	}	&		&		&	&	137	&	0	&	2139	&	&	2942	&	861	&	0	&	576	\\
\texttt{ 	qplib\_3659	}	&		&		&	&	0	&	960	&	4577	&	&	5537	&	960	&	480	&	1697	\\
\texttt{ 	qplib\_3661	}	&		&		&	&	10816	&	0	&	12997	&	&	11024	&	3221	&	0	&	0	\\
\texttt{ 	qplib\_3662	}	&		&		&	&	144	&	0	&	32	&	&	55	&	24	&	0	&	0	\\
\texttt{ 	qplib\_3670	}	&		&		&	&	54	&	0	&	864	&	&	305	&	54	&	0	&	0	\\
\texttt{ 	qplib\_3676	}	&		&		&	&	30	&	0	&	9000	&	&	4650	&	4500	&	0	&	0	\\
\texttt{ 	qplib\_3677	}	&		&		&	&	30	&	0	&	6000	&	&	3100	&	3000	&	0	&	0	\\
\texttt{ 	qplib\_3678	}	&		&		&	&	200	&	0	&	400	&	&	402	&	1	&	0	&	200	\\
\texttt{ 	qplib\_3680	}	&		&		&	&	92	&	0	&	16	&	&	40	&	12	&	0	&	0	\\
\texttt{ 	qplib\_3683	}	&		&		&	&	126	&	0	&	24	&	&	48	&	18	&	0	&	0	\\
\texttt{ 	qplib\_3690	}	&		&		&	&	20	&	0	&	6000	&	&	3150	&	3000	&	0	&	0	\\
\texttt{ 	qplib\_3692	}	&		&		&	&	128	&	0	&	1091	&	&	751	&	528	&	528	&	248	\\
\texttt{ 	qplib\_3693	}	&	49.4	&	0.4	&	&	1128	&	0	&	0	&	&	0	&	0	&	0	&	0	\\
\texttt{ 	qplib\_3694	}	&	0.0	&	0.1	&	&	40	&	0	&	3200	&	&	3280	&	0	&	0	&	0	\\
\texttt{ 	qplib\_3697	}	&		&		&	&	168	&	0	&	32	&	&	58	&	24	&	0	&	0	\\
\texttt{ 	qplib\_3698	}	&	0.0	&	0.1	&	&	30	&	0	&	3000	&	&	3100	&	0	&	0	&	0	\\
\texttt{ 	qplib\_3699	}	&		&		&	&	116	&	0	&	792	&	&	1668	&	192	&	0	&	541	\\
\texttt{ 	qplib\_3701	}	&		&		&	&	60	&	0	&	1080	&	&	377	&	60	&	0	&	0	\\
\texttt{ 	qplib\_3703	}	&	46.7	&	84.7	&	&	225	&	0	&	0	&	&	30	&	0	&	0	&	0	\\
\texttt{ 	qplib\_3705	}	&	50.6	&	1.1	&	&	378	&	0	&	0	&	&	0	&	0	&	0	&	0	\\
\texttt{ 	qplib\_3706	}	&	50.3	&	0.6	&	&	703	&	0	&	0	&	&	0	&	0	&	0	&	0	\\
\texttt{ 	qplib\_3708	}	&	0.0	&	0.1	&	&	14	&	0	&	12916	&	&	12917	&	0	&	0	&	0	\\
\texttt{ 	qplib\_3709	}	&	48.0	&	91.9	&	&	600	&	0	&	0	&	&	50	&	0	&	0	&	0	\\
\texttt{ 	qplib\_3713	}	&		&		&	&	42	&	0	&	630	&	&	254	&	42	&	0	&	0	\\
\texttt{ 	qplib\_3714	}	&	97.5	&	32.5	&	&	120	&	0	&	0	&	&	40	&	0	&	0	&	0	\\
\texttt{ 	qplib\_3719	}	&		&		&	&	133	&	0	&	28	&	&	51	&	21	&	0	&	0	\\
\texttt{ 	qplib\_3725	}	&		&		&	&	81	&	0	&	1171	&	&	1552	&	469	&	0	&	288	\\
\texttt{ 	qplib\_3726	}	&		&		&	&	116	&	0	&	816	&	&	2190	&	192	&	0	&	565	\\
\texttt{ 	qplib\_3727	}	&		&		&	&	20	&	0	&	1600	&	&	840	&	800	&	0	&	0	\\
\texttt{ 	qplib\_3728	}	&		&		&	&	72	&	0	&	16	&	&	35	&	12	&	0	&	0	\\
\texttt{ 	qplib\_3729	}	&		&		&	&	650	&	0	&	408	&	&	51	&	608	&	0	&	0	\\
\texttt{ 	qplib\_3733	}	&		&		&	&	46	&	0	&	644	&	&	237	&	46	&	0	&	0	\\
\texttt{ 	qplib\_3734	}	&		&		&	&	38	&	0	&	7533	&	&	7690	&	2754	&	0	&	4779	\\
\texttt{ 	qplib\_3738	}	&	49.9	&	0.9	&	&	435	&	0	&	0	&	&	0	&	0	&	0	&	0	\\
\texttt{ 	qplib\_3745	}	&	49.6	&	1.2	&	&	325	&	0	&	0	&	&	0	&	0	&	0	&	0	\\
\texttt{ 	qplib\_3748	}	&		&		&	&	75	&	0	&	20	&	&	37	&	15	&	0	&	0	\\
\texttt{ 	qplib\_3750	}	&	98.6	&	32.9	&	&	210	&	0	&	0	&	&	70	&	0	&	0	&	0	\\
\texttt{ 	qplib\_3751	}	&	98.0	&	32.7	&	&	150	&	0	&	0	&	&	50	&	0	&	0	&	0	\\
\texttt{ 	qplib\_3752	}	&	47.7	&	3.8	&	&	462	&	0	&	0	&	&	6160	&	0	&	0	&	0	\\
\texttt{ 	qplib\_3757	}	&	38.1	&	1.0	&	&	552	&	0	&	0	&	&	8096	&	0	&	0	&	0	\\
\texttt{ 	qplib\_3762	}	&	50.0	&	28.0	&	&	90	&	0	&	0	&	&	480	&	0	&	0	&	0	\\
\texttt{ 	qplib\_3772	}	&	50.0	&	3.9	&	&	380	&	0	&	0	&	&	4560	&	0	&	0	&	0	\\
\texttt{ 	qplib\_3775	}	&	98.4	&	32.8	&	&	180	&	0	&	0	&	&	60	&	0	&	0	&	0	\\
\texttt{ 	qplib\_3780	}	&		&		&	&	12	&	156	&	0	&	&	60	&	12	&	0	&	156	\\
\texttt{ 	qplib\_3785	}	&		&		&	&	200	&	0	&	32	&	&	62	&	24	&	0	&	0	\\
\texttt{ 	qplib\_3790	}	&	8.8	&	23.3	&	&	7	&	0	&	188	&	&	283	&	0	&	0	&	0	\\
\texttt{ 	qplib\_3792	}	&	0.0	&	0.1	&	&	20	&	0	&	3000	&	&	3150	&	0	&	0	&	0	\\
\texttt{ 	qplib\_3794	}	&		&		&	&	576	&	0	&	986	&	&	624	&	602	&	0	&	0	\\
\texttt{ 	qplib\_3797	}	&		&		&	&	48	&	0	&	296	&	&	623	&	56	&	0	&	223	\\
\texttt{ 	qplib\_3798	}	&		&		&	&	54	&	0	&	810	&	&	251	&	54	&	0	&	0	\\
\texttt{ 	qplib\_3803	}	&	42.7	&	14.1	&	&	190	&	0	&	0	&	&	2280	&	0	&	0	&	0	\\
\texttt{ 	qplib\_3809	}	&		&		&	&	224	&	0	&	32	&	&	65	&	24	&	0	&	0	\\
\texttt{ 	qplib\_3813	}	&		&		&	&	15	&	0	&	2400	&	&	1280	&	1200	&	0	&	0	\\
\texttt{ 	qplib\_3814	}	&	4.2	&	0.7	&	&	2	&	0	&	46	&	&	13	&	28	&	0	&	44	\\
\texttt{ 	qplib\_3815	}	&	50.0	&	3.2	&	&	192	&	0	&	0	&	&	64	&	0	&	0	&	0	\\
\texttt{ 	qplib\_3816	}	&		&		&	&	70	&	0	&	117	&	&	363	&	24	&	0	&	117	\\
\texttt{ 	qplib\_3822	}	&	49.9	&	0.5	&	&	861	&	0	&	0	&	&	0	&	0	&	0	&	0	\\
\texttt{ 	qplib\_3825	}	&		&		&	&	60	&	0	&	1020	&	&	317	&	60	&	0	&	0	\\
\texttt{ 	qplib\_3832	}	&	50.3	&	0.7	&	&	561	&	0	&	0	&	&	0	&	0	&	0	&	0	\\
\texttt{ 	qplib\_3834	}	&	60.0	&	98.0	&	&	50	&	0	&	0	&	&	1	&	0	&	0	&	0	\\
\texttt{ 	qplib\_3838	}	&	49.7	&	0.5	&	&	780	&	0	&	0	&	&	0	&	0	&	0	&	0	\\
\texttt{ 	qplib\_3840	}	&		&		&	&	2401	&	0	&	3334	&	&	2499	&	1374	&	0	&	0	\\
\texttt{ 	qplib\_3841	}	&	44.0	&	10.3	&	&	300	&	0	&	0	&	&	4600	&	0	&	0	&	0	\\
\texttt{ 	qplib\_3850	}	&	50.3	&	0.4	&	&	1225	&	0	&	0	&	&	0	&	0	&	0	&	0	\\
\texttt{ 	qplib\_3852	}	&	50.7	&	1.7	&	&	231	&	0	&	0	&	&	0	&	0	&	0	&	0	\\
\texttt{ 	qplib\_3854	}	&		&		&	&	40	&	0	&	640	&	&	266	&	40	&	0	&	0	\\
\texttt{ 	qplib\_3855	}	&		&		&	&	400	&	0	&	2118	&	&	791	&	1284	&	0	&	0	\\
\texttt{ 	qplib\_3856	}	&		&		&	&	168	&	0	&	183	&	&	50	&	267	&	0	&	0	\\
\texttt{ 	qplib\_3857	}	&		&		&	&	201	&	0	&	602	&	&	604	&	2	&	0	&	201	\\
\texttt{ 	qplib\_3859	}	&		&		&	&	600	&	0	&	968	&	&	1225	&	560	&	0	&	0	\\
\texttt{ 	qplib\_3860	}	&	44.9	&	8.7	&	&	435	&	0	&	0	&	&	8120	&	0	&	0	&	0	\\
\texttt{ 	qplib\_3861	}	&	0.0	&	0.1	&	&	30	&	0	&	4500	&	&	4650	&	0	&	0	&	0	\\
\texttt{ 	qplib\_3863	}	&		&		&	&	625	&	0	&	1053	&	&	675	&	628	&	0	&	0	\\
\texttt{ 	qplib\_3865	}	&	48.0	&	90.7	&	&	525	&	0	&	0	&	&	50	&	0	&	0	&	0	\\
\texttt{ 	qplib\_3870	}	&	49.0	&	23.1	&	&	116	&	0	&	66	&	&	1456	&	0	&	0	&	66	\\
\texttt{ 	qplib\_3871	}	&	0.0	&	0.1	&	&	25	&	0	&	1000	&	&	1040	&	0	&	0	&	0	\\
\texttt{ 	qplib\_3872	}	&		&		&	&	95	&	0	&	1413	&	&	1874	&	567	&	0	&	360	\\
\texttt{ 	qplib\_3877	}	&	50.4	&	0.7	&	&	630	&	0	&	0	&	&	0	&	0	&	0	&	0	\\
\texttt{ 	qplib\_3879	}	&		&		&	&	10920	&	0	&	12906	&	&	21945	&	3026	&	0	&	0	\\
\texttt{ 	qplib\_3883	}	&	50.0	&	17.8	&	&	182	&	0	&	0	&	&	1456	&	0	&	0	&	0	\\
\texttt{ 	qplib\_3913	}	&	0.0	&	100.0	&	&	300	&	0	&	0	&	&	61	&	0	&	0	&	0	\\
\texttt{ 	qplib\_3923	}	&	53.7	&	5.2	&	&	395	&	0	&	0	&	&	80	&	0	&	0	&	0	\\
\texttt{ 	qplib\_3931	}	&	50.4	&	4.5	&	&	316	&	0	&	0	&	&	80	&	0	&	0	&	0	\\
\texttt{ 	qplib\_3980	}	&	0.0	&	100.0	&	&	235	&	0	&	0	&	&	48	&	0	&	0	&	0	\\
\texttt{ 	qplib\_4095	}	&	0.0	&	4.0	&	&	400	&	0	&	1600	&	&	1603	&	400	&	0	&	0	\\
\texttt{ 	qplib\_4270	}	&	0.0	&	6.3	&	&	400	&	0	&	1200	&	&	1603	&	0	&	0	&	0	\\
\texttt{ 	qplib\_4455	}	&		&		&	&	3000	&	0	&	12000	&	&	9001	&	3000	&	0	&	0	\\
\texttt{ 	qplib\_4722	}	&		&		&	&	2000	&	0	&	8000	&	&	6001	&	2000	&	0	&	0	\\
\texttt{ 	qplib\_4805	}	&		&		&	&	2000	&	0	&	8000	&	&	6074	&	2000	&	0	&	2000	\\
\texttt{ 	qplib\_5023	}	&		&		&	&	3000	&	0	&	12000	&	&	9155	&	3000	&	0	&	3000	\\
\texttt{ 	qplib\_5442	}	&		&		&	&	2000	&	0	&	7999	&	&	6088	&	2000	&	0	&	1999	\\
\texttt{ 	qplib\_5527	}	&	0.0	&	0.1	&	&	4492	&	0	&	21117	&	&	64348	&	0	&	0	&	12305	\\
\texttt{ 	qplib\_5543	}	&	0.0	&	0.1	&	&	4514	&	0	&	21186	&	&	64096	&	0	&	0	&	12328	\\
\texttt{ 	qplib\_5554	}	&		&		&	&	4492	&	0	&	30878	&	&	64769	&	4800	&	0	&	12158	\\
\texttt{ 	qplib\_5573	}	&		&		&	&	4450	&	0	&	23692	&	&	72976	&	4800	&	0	&	4987	\\
\texttt{ 	qplib\_5577	}	&	0.0	&	0.1	&	&	1118	&	0	&	4896	&	&	15690	&	0	&	0	&	2703	\\
\texttt{ 	qplib\_5721	}	&	51.0	&	76.9	&	&	300	&	0	&	0	&	&	0	&	0	&	0	&	0	\\
\texttt{ 	qplib\_5725	}	&	49.9	&	1.8	&	&	343	&	0	&	0	&	&	0	&	0	&	0	&	0	\\
\texttt{ 	qplib\_5755	}	&	50.0	&	1.0	&	&	400	&	0	&	0	&	&	0	&	0	&	0	&	0	\\
\texttt{ 	qplib\_5875	}	&	50.0	&	78.9	&	&	200	&	0	&	0	&	&	0	&	0	&	0	&	0	\\
\texttt{ 	qplib\_5881	}	&	50.9	&	29.5	&	&	120	&	0	&	0	&	&	0	&	0	&	0	&	0	\\
\texttt{ 	qplib\_5882	}	&	50.7	&	78.2	&	&	150	&	0	&	0	&	&	0	&	0	&	0	&	0	\\
\texttt{ 	qplib\_5909	}	&	50.0	&	9.7	&	&	250	&	0	&	0	&	&	0	&	0	&	0	&	0	\\
\texttt{ 	qplib\_5922	}	&	50.2	&	9.9	&	&	500	&	0	&	0	&	&	0	&	0	&	0	&	0	\\
\texttt{ 	qplib\_5924	}	&	0.0	&	0.1	&	&	300	&	0	&	15220	&	&	36060	&	0	&	0	&	0	\\
\texttt{ 	qplib\_5925	}	&		&		&	&	100	&	0	&	1301	&	&	271	&	100	&	0	&	0	\\
\texttt{ 	qplib\_5926	}	&		&		&	&	2400	&	0	&	31201	&	&	11923	&	2400	&	0	&	0	\\
\texttt{ 	qplib\_5927	}	&		&		&	&	2400	&	0	&	31201	&	&	11963	&	2400	&	0	&	0	\\
\texttt{ 	qplib\_5935	}	&	51.0	&	99.0	&	&	100	&	0	&	0	&	&	1237	&	0	&	0	&	0	\\
\texttt{ 	qplib\_5944	}	&	51.0	&	99.0	&	&	100	&	0	&	0	&	&	2475	&	0	&	0	&	0	\\
\texttt{ 	qplib\_5962	}	&	50.7	&	99.4	&	&	150	&	0	&	0	&	&	2793	&	0	&	0	&	0	\\
\texttt{ 	qplib\_5971	}	&	50.7	&	99.4	&	&	150	&	0	&	0	&	&	5587	&	0	&	0	&	0	\\
\texttt{ 	qplib\_5980	}	&	50.7	&	99.4	&	&	150	&	0	&	0	&	&	8381	&	0	&	0	&	0	\\
\texttt{ 	qplib\_6287	}	&		&		&	&	0	&	0	&	171	&	&	36	&	81	&	0	&	171	\\
\texttt{ 	qplib\_6310	}	&		&		&	&	0	&	0	&	208	&	&	22	&	390	&	0	&	208	\\
\texttt{ 	qplib\_6311	}	&		&		&	&	0	&	0	&	212	&	&	43	&	128	&	0	&	212	\\
\texttt{ 	qplib\_6324	}	&	51.0	&	31.3	&	&	640	&	0	&	0	&	&	16	&	0	&	0	&	0	\\
\texttt{ 	qplib\_6487	}	&	51.0	&	20.9	&	&	618	&	0	&	0	&	&	309	&	0	&	0	&	0	\\
\texttt{ 	qplib\_6597	}	&	45.7	&	97.4	&	&	600	&	0	&	0	&	&	60	&	0	&	0	&	0	\\
\texttt{ 	qplib\_6647	}	&	70.1	&	7.3	&	&	627	&	0	&	0	&	&	33	&	0	&	0	&	0	\\
\texttt{ 	qplib\_6757	}	&	36.1	&	4.8	&	&	2046	&	0	&	0	&	&	297	&	0	&	0	&	0	\\
\texttt{ 	qplib\_6764	}	&	35.2	&	4.8	&	&	2071	&	0	&	0	&	&	297	&	0	&	0	&	0	\\
\texttt{ 	qplib\_6799	}	&	36.2	&	4.8	&	&	2075	&	0	&	0	&	&	297	&	0	&	0	&	0	\\
\texttt{ 	qplib\_6941	}	&	35.9	&	4.5	&	&	2203	&	0	&	0	&	&	315	&	0	&	0	&	0	\\
\texttt{ 	qplib\_7127	}	&	50.6	&	6.8	&	&	1000	&	0	&	0	&	&	50	&	0	&	0	&	0	\\
\texttt{ 	qplib\_7139	}	&	53.4	&	89.3	&	&	180	&	0	&	0	&	&	100	&	0	&	0	&	0	\\
\texttt{ 	qplib\_7144	}	&	53.2	&	89.7	&	&	220	&	0	&	0	&	&	121	&	0	&	0	&	0	\\
\texttt{ 	qplib\_7149	}	&	53.1	&	89.7	&	&	264	&	0	&	0	&	&	144	&	0	&	0	&	0	\\
\texttt{ 	qplib\_7154	}	&	52.9	&	89.7	&	&	312	&	0	&	0	&	&	169	&	0	&	0	&	0	\\
\texttt{ 	qplib\_7159	}	&	52.5	&	89.7	&	&	364	&	0	&	0	&	&	196	&	0	&	0	&	0	\\
\texttt{ 	qplib\_7164	}	&	52.4	&	89.7	&	&	420	&	0	&	0	&	&	225	&	0	&	0	&	0	\\
\texttt{ 	qplib\_7579	}	&		&		&	&	100	&	0	&	200	&	&	202	&	1	&	1	&	100	\\
\texttt{ 	qplib\_8009	}	&		&		&	&	101	&	0	&	303	&	&	305	&	2	&	0	&	101	\\
\texttt{ 	qplib\_8153	}	&		&		&	&	31	&	0	&	93	&	&	95	&	2	&	0	&	31	\\
\texttt{ 	qplib\_8381	}	&		&		&	&	51	&	0	&	153	&	&	155	&	2	&	0	&	51	\\
\texttt{ 	qplib\_8495	}	&	0.0	&	0.1	&	&	0	&	0	&	27543	&	&	8000	&	0	&	0	&	0	\\
\texttt{ 	qplib\_8505	}	&	0.0	&	0.1	&	&	0	&	0	&	20050	&	&	10001	&	0	&	0	&	20050	\\
\texttt{ 	qplib\_8515	}	&	0.0	&	0.1	&	&	0	&	0	&	16002	&	&	8002	&	0	&	0	&	8001	\\
\texttt{ 	qplib\_8559	}	&	0.0	&	0.1	&	&	0	&	0	&	10000	&	&	5000	&	0	&	0	&	10000	\\
\texttt{ 	qplib\_8567	}	&	0.0	&	0.1	&	&	0	&	0	&	10000	&	&	7500	&	0	&	0	&	10000	\\
\texttt{ 	qplib\_8602	}	&	0.0	&	0.1	&	&	0	&	0	&	34552	&	&	52983	&	0	&	0	&	34552	\\
\texttt{ 	qplib\_8605	}	&	0.0	&	0.1	&	&	0	&	0	&	5000	&	&	0	&	1	&	1	&	0	\\
\texttt{ 	qplib\_8616	}	&	0.0	&	0.1	&	&	0	&	0	&	13870	&	&	10404	&	0	&	0	&	4	\\
\texttt{ 	qplib\_8685	}	&	0.0	&	0.2	&	&	0	&	0	&	772	&	&	0	&	10000	&	0	&	0	\\
\texttt{ 	qplib\_8777	}	&	34.6	&	0.1	&	&	0	&	0	&	10000	&	&	2500	&	0	&	0	&	10000	\\
\texttt{ 	qplib\_8785	}	&	0.0	&	0.1	&	&	0	&	0	&	10399	&	&	11362	&	0	&	0	&	10399	\\
\texttt{ 	qplib\_8790	}	&	0.0	&	0.1	&	&	0	&	0	&	39204	&	&	0	&	0	&	0	&	19602	\\
\texttt{ 	qplib\_8792	}	&	0.0	&	0.1	&	&	0	&	0	&	15129	&	&	0	&	0	&	0	&	15129	\\
\texttt{ 	qplib\_8845	}	&	0.0	&	4.9	&	&	0	&	0	&	1546	&	&	777	&	0	&	0	&	15	\\
\texttt{ 	qplib\_8906	}	&	0.0	&	0.5	&	&	0	&	0	&	5223	&	&	838	&	0	&	0	&	0	\\
\texttt{ 	qplib\_8938	}	&	0.0	&	0.1	&	&	0	&	0	&	4001	&	&	11999	&	0	&	0	&	0	\\
\texttt{ 	qplib\_8991	}	&	0.0	&	0.1	&	&	0	&	0	&	14400	&	&	0	&	0	&	0	&	14400	\\
\texttt{ 	qplib\_9002	}	&	0.0	&	0.1	&	&	0	&	0	&	2890	&	&	1649	&	0	&	0	&	727	\\
\texttt{ 	qplib\_9004	}	&	0.0	&	0.1	&	&	0	&	0	&	40000	&	&	10001	&	10001	&	10001	&	20000	\\
\texttt{ 	qplib\_9030	}	&	0.3	&	0.1	&	&	0	&	10000	&	0	&	&	5000	&	0	&	0	&	10000	\\
\texttt{ 	qplib\_9048	}	&	29.8	&	18.3	&	&	0	&	202	&	0	&	&	1	&	0	&	0	&	202	\\


\bottomrule

%\end{tabular}  
%\label{tab:A1}
%\caption{Discrete Instance Feature 1-60} 

\end{longtable}

}




%- - - - - - - - - - - - - - - - - - - - - - - - - - - - - - - - - - - -
%- - - - - - - - - - - - - - - - - - - - - - - - - - - - - - - - - - - -
%  End Appendix.tex
%- - - - - - - - - - - - - - - - - - - - - - - - - - - - - - - - - - - -
%- - - - - - - - - - - - - - - - - - - - - - - - - - - - - - - - - - - -


%- - - - - - - - - - - - - - - - - - - - - - - - - - - - - - - - - - - -
%- - - - - - - - - - - - - - - - - - - - - - - - - - - - - - - - - - - -

\end{document}

%- - - - - - - - - - - - - - - - - - - - - - - - - - - - - - - - - - - -
%- - - - - - - - - - - - - - - - - - - - - - - - - - - - - - - - - - - -
%  End QPLIB.tex
%- - - - - - - - - - - - - - - - - - - - - - - - - - - - - - - - - - - -
%- - - - - - - - - - - - - - - - - - - - - - - - - - - - - - - - - - - -
