%- - - - - - - - - - - - - - - - - - - - - - - - - - - - - - - - - - - -
%- - - - - - - - - - - - - - - - - - - - - - - - - - - - - - - - - - - -
%  QPLIB-2.2.tex
%- - - - - - - - - - - - - - - - - - - - - - - - - - - - - - - - - - - -
%- - - - - - - - - - - - - - - - - - - - - - - - - - - - - - - - -


%- - - - - - - - - - - - - - - - - - - - - - - - - - - - - - - - - - - -
\subsection{Classification}\label{sec:classification}

Despite the apparent simplicity of the definition given in \S\ref{subsec:notation}, Quadratic Programming instances can be of several rather different ``types'' in practice, depending on fine details of the data. In particular, many algorithmic approaches can only be applied to QP when the data of the problem has specific properties. A taxonomy of QP instances should thus strive to identify the set of properties that an instance should have in order to apply the most relevant computational methods. However, the sheer number of different existing approaches, and the fact that new ones are frequently proposed, makes it hard to provide a taxonomy that is both simple and covers all possible special cases. This is why, in this paper, we propose an approach that aims at finding a good balance between simplicity and coverage of the main families of computational methods.

% -  -  -  -  -  -  -  -  -  -  -  -  -  -  -  -  -  -  -  -  -  -  -  -
\subsubsection{Taxonomy}\label{ssec:taxonomy}


Our taxonomy is based on a three-fields code of the form ``\textit{OVC}'', where \textit{O} indicates the type of objective function considered, \textit{V} records the types of variables, and \textit{C} designates the types of constraints imposed on the variables. The fields can be given the following values:
%
\begin{itemize}
 \item objective function: (\textit{L})inear, (\textit{D})iagonal convex (if minimization) or concave (if maximization) quadratic, (\textit{C})onvex (if minimization) or (\textit{C})oncave (if maximization) quadratic,  (\textit{Q})uadratic (all other cases);
 %
 \item variables: (\textit{C})ontinuous only, (\textit{B})inary only, (\textit{M})ixed binary and continuous, (\textit{I})nteger (including binary) only, (\textit{G})eneral (all other cases);
 %
 \item constraints: (\textit{N})one, (\textit{B})ox, (\textit{L})inear, (\textit{D})iagonal convex quadratic, (\textit{C})on\-vex quadratic, nonconvex (\textit{Q})uadratic.
  %Note that (\textit{D}) and (\textit{C}) are intended to mean that either $Q^i$ is positive semidefinite and $c_l^i = -\infty$, or $Q^i$ is negative semidefinite and $c_u^i = \infty$. 
    {Note that (positive or negative) definiteness of $Q^i$ is a sufficient, but not in general necessary, condition for convexity. As detailed in \S\ref{subsec:final set}, in our taxonomy we mark the constraints ``$C$'' based on the sufficient condition alone, the rationale of this choice being discussed in \S\ref{ssec:reform}.}
  Quadratic constraints with both finite bounds cannot ever be convex (unless $Q^i = 0$, i.e., they are not ``truly'' quadratic constraints).
\end{itemize}

The ordering of the values in the previous lists is not irrelevant; in general, problems become ``harder'' when going from left to right. More specifically, for the \textit{O} and \textit{C} fields the order is that of \emph{strict} containment between problem classes: for instance, linear objective functions are strictly a special case of diagonal convex quadratic ones (by allowing the diagonal elements all to be zero), the latter are a strict subset of general convex quadratic objectives (by allowing the off-diagonal elements all to be zero), and these are strictly subsets of general nonconvex quadratic ones (since these permit any symmetric Hessian including positive semidefinite ones). The only field for which the containment relationship is not a total order is \textit{V}, for which only the partial orderings
\[
 \mbox{\textit{C}} \subset \mbox{\textit{M}} \subset \mbox{\textit{G}},
 \qquad
 \mbox{\textit{B}} \subset \mbox{\textit{M}} \subset \mbox{\textit{G}},
 \qquad
 \mbox{and}
 \qquad
 \mbox{\textit{B}} \subset \mbox{\textit{I}} \subset \mbox{\textit{G}}
\]
hold. In the following discussion we will repeatedly exploit this by assuming that, unless otherwise mentioned, when a method can be applied to a given problem, it can be applied as well to all simpler problems where the value of each field is arbitrarily replaced with a value denoting a less-general class.

The wildcard ``*'' will be used below to indicate any possible choice, and lists of the form ``\{\textit{X}, \textit{Y}, \textit{Z}\}'' will indicate that the value of the given field can freely attain any of the specified values.



% -  -  -  -  -  -  -  -  -  -  -  -  -  -  -  -  -  -  -  -  -  -  -  -
\subsubsection{Examples and Reformulations}\label{ssec:reform}

We now give a general discussion about the different problem classes that our proposed taxonomy defines.
For simplicity, we will assume minimization problems for the remaining of this section.
Some problem classes are actually ``too simple'' to make sense in our context. For instance, \textit{D*B} problems have only diagonal quadratic (hence separable) objective function and bound constraints; as such, they read
\[
 \textstyle
 \min \; \Big\{ \,
 \sum_{j \in \Mcn} \big( \, {\textstyle \half} Q^0_j x_j^2 + b^0_j x_j \, \big) \;:\;
 l_j \leq x_j \leq u_j \quad j \in \Mcn \;\;,\;\;
 x_j \in \mathbb{Z} \quad j \in \Mcz \, \Big\}
 \;\; .
\]
Hence, their solution only requires the independent minimization of a convex quadratic univariate function in each single variable $x_j$ over a box constraint and possibly integrality requirements, which can be attained trivially in $O(1)$ operations (per variable) by closed-form formul{\ae}, projection and rounding arguments. \emph{A fortiori}, the even simpler cases \textit{L*B}, \textit{D*N} and \textit{L*N} (the latter unbounded unless $b^0 = 0$) will not be discussed here. Similarly, \textit{CCN} are immediately solved by linear algebra techniques, and therefore are of no interest in this context. At the other end of the spectrum, in general QP is a hard problem. Actually, \textit{LIQ}---linear objective function and quadratic constraints in integer variables with no finite bounds, i.e.
\[
 \textstyle
 \min \; \Big\{ \, b^0 x \;:\;
 {\textstyle \half} x^\top Q^i x + b^i x \leq c^i \quad i \in \Mcm \;\;,\;\;
 x_j \in \mathbb{Z} \quad j \in \Mcn \, \Big\}
 \;\; ,
\]
is not only $\mathcal{NP}$-hard, but undecidable \cite{jeroslow}. Hence so are the ``harder'' \{\textit{C},\textit{Q}\}\textit{IQ}.

%\smallskip
It is important to note that the relationships between the different classes can be somehow blurred because reformulation techniques may allow one to move an instance from one class to another. We already mentioned that $x^2 = x \iff x \in \{0, 1\}$, and in general \textit{*M*}---instances with only binary and continuous variables---can be recast as \textit{*CQ}; here nonconvex quadratic constraints take the place of binary variables. More generally, this is also true for \textit{*G*} as long as $\Mcu = \emptyset$, as bounded general integer variables can be represented by binary ones.
Hence, the nonconvexity due to binary variables can always be expressed by means of (nonconvex) quadratic constraints. The converse is also true: when only binary variables are present, all quadratic constraints can be converted into convex ones \cite{BEP09,BEL13}.


Another relevant reformulation trick concerns the fact that, as soon as quadratic constraints are allowed, then there is no loss of generality in assuming a linear objective function. Indeed, any \textit{Q**} (\textit{C*C}) problem can always be rewritten as
%
\begin{align*}
 \min \;\;
 & x^0 \\
 %
 & \hspace*{-5mm} -\infty \leq {\textstyle \half} x^\top Q^0 x + b^0 x \leq x^0 & \\
 %
 & c^i_l \leq {\textstyle \half} x^\top Q^i x + b^i x \leq c^i_u & i \in \Mcm \\
 %
 & l_j \leq x_j \leq u_j & j \in \Mcn  \\
 %
 & x_j \in \mathbb{Z} & j \in \Mcz
\end{align*}
%
i.e., a \textit{L*Q} (\textit{L*C}) problem. Hence, it is clear that quadratic constraints are, in a well-defined sense, the most general situation (cf.~also the result above about hardness of \textit{LIQ}).

When a $Q^i$ is positive semidefinite (PSD), i.e., the corresponding constraint/objective function is convex, general Hessians are in fact equivalent to diagonal ones. In particular, since every PSD matrix can be factorized as $Q^i = L^i (L^i)^\top$, e.g.~by the (incomplete) Cholesky factorization, the term $\half x^\top Q^i x \equiv \half \sum_{j \in \Mcn} z^{i\;2}_j$ where $z^{i\;\top} = x^\top L^i$. Hence, one might maintain that D** problems need not be distinguished from C** ones. However in reality, this is only true for ``complicated'' constraints but not for ``simple'' ones, because the above reformulation technique introduces additional linear constraints, $L^{i\;\top} x - z^i = 0$. Indeed, while \textit{C*L} (and, a fortiori, \textit{C*}\{\textit{C},\textit{Q}\}) can always be brought to \textit{D*L} (\textit{D*}\{\textit{C},\textit{Q}\}), using the above technique \textit{C*B} becomes \textit{D*L}, which may be significantly different from \textit{D*B}. In practice, a diagonal convex objective function under linear constraints is found in many applications (e.g., \cite{FrFG16,FrGe06a,FrGe09a,FGGP11}), so that it still makes sense to distinguish the \textit{D*L} case where the objective function is ``naturally'' separable from that where separability is artificially introduced.

{
Furthermore, as previously remarked, a not (positive or negative) definite $Q^i$ does not necessarily correspond to a nonconvex feasible region. For instance, it is well-known that Second-Order Cone Programs have convex feasible regions; when represented in terms of quadratic constraints, however, they correspond to $Q^i$ with one negative eigenvalue. In our taxonomy we still consider the corresponding instances as \textit{**Q} ones, with no attempt to detect the different special structures that actually correspond to convex feasible regions. Although this may lead to classify as ``potentially nonconvex'' some instances that are in fact convex, our choice is justified by the fact that not all QP solvers are capable of detecting and exploiting these structures, which means that the instance can actually be treated as a nonconvex one even if it is not.\\
}
{
One of the nontrivial choices in our library is that we made no effort to reformulate the instances, and inserted them in the library in the very same form as they have been provided to us by the original contributors. The rationale of this choice is that reformulation techniques, like the ones discussed here and others, are typically motivated by the fact that they make the instance easier to solve for one specific class of solvers. This being a bias that we do not want to add we have chosen to keep the instances in their ``natural'' form, this being the one in which the original contributor initially wrote them.
}


% -  -  -  -  -  -  -  -  -  -  -  -  -  -  -  -  -  -  -  -  -  -  -  -
\subsubsection{QP Classes}\label{ssec:classes}

The proposed taxonomy can then be used to describe the main classes of QP according to the type of algorithms that can be applied for their solution:
%
\begin{itemize}
 \item \emph{Linear Programs} \textit{LCL} and \emph{Mixed-Integer Linear Programs} \textit{LGL} have been subject of an enormous amount of research and have their well-established instance libraries \cite{Koch2011}, so they will not be explicitly addressed here.
 %
 \item \emph{Convex Continuous Quadratic Programs} \textit{CCC} can be solved in polynomial time by Interior-Point techniques \cite{Wright97}; the simpler \textit{CCL} can also be solved by means of ``simplex-like'' techniques, usually referred to as active-set methods \cite{Dost09}. Actually, a slightly larger class of problems can be solved with Interior-Point methods: those that can be represented by Second-Order Cone Programs. When written as QPs the corresponding $Q^i$ may not be positive semidefinite, but nonetheless such problems can be efficiently solved. Of course, just as for \textit{LCL}, these problems may still require considerable computational effort when the size of the instance grows. In this sense, like in the linear case, there is a significant distinction between solvers that need all the data of QP to work, and those that are ``matrix-free'', i.e., only require the application of simple operations (typically, matrix-vector products) with the problem data. When building our instance library we never exploited such characteristics, since they are not amenable to standard modeling tools, but this may be relevant for the solution of very-large-scale \textit{CIC}.
 %
 \item \emph{Nonconvex Continuous Quadratic Programs} \textit{QCQ} are generally $\mathcal{NP}$-hard, even if the constraints are very specific (\textit{QCB}) and only a single eigenvalue of $Q^0$ is negative \cite{Hemmecke2010}. They therefore require enumerative techniques, such as spatial Branch-and-Bound \cite{FV90,BeLeLiMaWa08}, to be solved to optimality. Of course, local approaches are available that are able to efficiently provide saddle points (hopefully, local optima) of the \textit{CQC}, but providing global guarantees about the quality of the obtained solutions is challenging. In our library we have specifically focused on  \emph{exact} solution of the instances.
  %
 \item \emph{Convex Integer Quadratic Programs} \textit{CGC} are, in general, $\mathcal{NP}$-hard, and therefore require enumerative techniques to be solved. However, convexity of the objective function and constraints implies that efficient techniques (see \textit{CCC}) can be used at least to solve continuous relaxations. The general view is that \textit{CGC} are not, all other things being equal,  substantially more difficult than \textit{LGL} to solve, especially if the objective function and/or the constraints have specific properties (e.g., \textit{DGL}, \textit{CGL}). Often, integer variables are in fact binary ones, so several \textit{CGC} models are \textit{C}\{\textit{B},\textit{M}\}\textit{C} ones. In practice, binary variables are considered to lead to somewhat easier problems than general integer ones (cf.~the cited result about hardness of unbounded integer quadratic programs) and several algorithmic techniques have been specifically developed for this special case. However, the general approaches for \textit{CBC} are basically the same as for \textit{CGC}, so there is seldom the need to distinguish between the two classes as far as solvability is concerned, although matters can be different regarding actual solution cost. Programs with only binary variables (\textit{CBC}) can be easier than mixed-binary or integer ones (\textit{C}\{\textit{M},\textit{I}\}\textit{C}) because some techniques are specifically known for the binary-only case, cf.~the next point \cite{BEP09}. Absence of continuous variables, even in the presence of integer ones (\textit{CIC}), can also lead to specific techniques \cite{BEL13}.
  %
  \item \emph{Nonconvex Binary Quadratic Programs} \textit{QB}\{\textit{B},\textit{N},\textit{L}\} are $\mathcal{NP}$-hard. However, the special nature of binary variables combined with quadratic forms allows for quite specific techniques to be developed, one of which is the reformulation of the problem as a \textit{LBL}. Also, many well-known combinatorial problems can be naturally reformulated as problems of this class, and therefore a considerable number of results have been obtained by exploiting specific properties of the set of constraints \cite{Rendl2008,loiola-etal:2007}.
  %
  \item \emph{Nonconvex Integer Quadratic Programs} \textit{QGQ} is the most general, and therefore is the most difficult, class. Due to the lack of convexity even when integrality requirements are removed, solution methods must typically combine several algorithmic ideas, such as enumeration (distinguishing the role of integral variables from that of continuous ones involved in nonconvex terms) and techniques that allow the efficient computation of bounds (e.g., outer approximation, semidefinite programming relaxation, \ldots). As in the convex case, \textit{QBQ}, \textit{QMQ}, and \textit{QIQ} can benefit from more specific properties of the variables \cite{Buchheim2013,Dong2016}.
\end{itemize}
%
This description is deliberately coarse; each of these classes can be subdivided into several others on the grounds of more detailed information about structures present in their constraints/objective function. These can have a significant algorithmic impact, and therefore can be of interest to researchers. Common structures are, e.g., network flows \cite{FrFG16,FrGaSc14,FrGaSt16a,FGGP11,tadayon-smith:2013} or knapsack-type linear constraints \cite{FrFG16,FrGo13b,FGGP11}, and semi-continuous variables \cite{FrFG16,FrGaSc14,FrGaSt16a,FrGe06a,FrGe07a,FrGe09a,FGGP11}
\todo{SV: reduce to 1-3 citations here},
or the fact that a nonconvex quadratic objective function/constraint can be reformulated as a second-order cone (hence, convex) one \cite{FrFG16,FrGaSc14,FrGaSt16a,FrGe09a,FGGP11}
\todo{SV: reduce to 1-3 citations here}.
It would be very hard to collect a comprehensive list of all types of structures that might be of interest to any individual researcher, since these are as varied as the different possible approaches for specialized sub-classes of QP. For this reason we do not attempt such a more refined classification, and limit ourselves to the coarser one described in this section.

%- - - - - - - - - - - - - - - - - - - - - - - - - - - - - - - - - - - -
%- - - - - - - - - - - - - - - - - - - - - - - - - - - - - - - - - - - -
%  End QPLIB-2.2.tex
%- - - - - - - - - - - - - - - - - - - - - - - - - - - - - - - - - - - -
%- - - - - - - - - - - - - - - - - - - - - - - - - - - - - - - - - - - -
