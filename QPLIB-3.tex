%- - - - - - - - - - - - - - - - - - - - - - - - - - - - - - - - - - - -
%- - - - - - - - - - - - - - - - - - - - - - - - - - - - - - - - - - - -
%  QPLIB-3.tex
%- - - - - - - - - - - - - - - - - - - - - - - - - - - - - - - - - - - -
%- - - - - - - - - - - - - - - - - - - - - - - - - - - - - - - - - - - -
\section{Library Construction}\label{sec:lib}

In this section we present all the steps performed in order to build the
new library. In \S \ref{subsec:instColl}, we describe the set of gathered
instances. In \S \ref{subsec:selection} we present the features used to
classify the instances and we discuss the issues concerning the format of
the instances. Finally, in \S \ref{subsec:final set}, we describe the
selection process used to filter the instances and we graphically present
the main features of the selected instances.

%- - - - - - - - - - - - - - - - - - - - - - - - - - - - - - - - - - - -
\subsection{Instance Collection}\label{subsec:instColl}

In this section we describe the procedure we adopted to gather the
instances. In January $2014$, we issued an online call for instances
using the main international mailing lists of the mathematical
optimization and numerical analysis communities, reaching in this way
the largest possible set of interested researchers and practitioners.
The call remained open for 10 months, during which we received a large
number of contributions of different nature. The instances we gathered
come both from theoretical studies as well as from real-world
applications.

In addition to spontaneous contribution we analysed the other generic
libraries of instances available  on internet and containing QP
instances. Namely, the libraries from which we gathered instances are
%
\begin{itemize}
 \item the {\tt BARON} library
 \url{http://www.minlp.com/nlp-and-minlp-test-problems};
%
\item the {\tt CUTEst} library
 \url{https://ccpforge.cse.rl.ac.uk/gf/project/cutest/wiki};
%
\item the {\tt GAMS} library
 \url{http://www.gamsworld.org/performance/performlib.htm};
%
\item the {\tt MacMINLP} library
 \url{https://wiki.mcs.anl.gov/leyffer/index.php/MacMINLP};
%
\item the {\tt Meszaros} library
 \url{http://www.doc.ic.ac.uk/~im/00README.QP};
%
\item the {\tt MINLP} library
 \url{http://www.gamsworld.org/minlp/minlplib.htm};
%
\item the {\tt POLIP} library
 \url{http://polip.zib.de/pipformat.php}.
\end{itemize}

Other quadratic instances were found in online libraries devoted to
specific QP problems as Max-Cut, Quadratic Assignment, Portfolio
Optimization, and several others.

At the end of this process we gathered more than eight thousand
instances. Three fourths of them contained discrete variables, while
the remaining ones contained only continuous variables. More in details,
we gathered $\approx 1800$ Quadratic Binary Linear (QBL) instances,
$\approx 2000$ Quadratic Continuous Quadratic (QCQ) instances, and
and $\approx 2500$ Quadratic General Quadratic (QGQ) instances. W
also gathered $\approx 1000$ Convex General Convex (CGC) instances. We
gathered relatively fewer Quadratic Binary Quadratic (QBQ), Convex
Continuous Convex (CCC) and Convex Mixed Convex (CMC) instances, i.e.,
$\approx 150$ instances, $\approx 200$ instances and $\approx 200$
instances respectively. Finally, we gathered only $17$ Quadratic Mixed
Linear (QML) instances. In the call for instances, no specific formats
requirements were imposed for the submissions.

To evaluate the instances we decided, for practical reasons, to use
Gams as common platform for all the experiments involving commercial and
non-commercial solvers. For this reason, we decided to translate all
instances into the Gams format (\texttt{.gms}). 
% In a preliminary phase, all the instances received were divided
%according to their format and subsequently translated.
%In \S~\ref{subsec:tools}
%%~\ref{subsec:CP_convex}
%the tools used
%to translate an instance from a given format to the \texttt{.gms} format
%%to translate an instance from and to a given format to the
%\texttt{.gms} format
%are described in more detail.\\
%
In addition, we have introduced a new, specific \texttt{.qplib} format. This
new format is capable of describing all the instances of the library in a sparse form.
In comparison to a more \emph{high level} format like \texttt{.gms}, the new
format presents two advantages: it is easier to read by a self-made parser and
it produces smaller files. See the Appendix~\label{sec:format} for more details.

%\subsection{Instance Features}\label{subsec:feature}

For each instance of the starting set, we collected important characteristics
which allowed us to classify the instances into the QP categories described in
Section~\ref{sec:QPbasic}. As far as the variable types are concerned, we
collected the following information:
%
\begin{itemize}
 \item number of binary variables; % (\emph{\# bin}),
 %
 \item number of integer variables; % (\emph{\# int}),
 %
 \item number of continuous variables. % (\emph{\# cont}).
\end{itemize}
%
In case at least one binary or integer variable is present, then the instance is
categorized as \emph{discrete}, otherwise it is categorized as \emph{continuous}.
As fas as the objective function is concerned, we gathered the following
information:
%
\begin{itemize}
 \item percentage of negative eigenvalues of the $Q^0$ matrix;
       % (\emph{\% neg eig}),
 %
 \item density of the $Q^0$ matrix (number of nonzero entries over the total
       number). %(\emph{\% dens}).
\end{itemize}
%
The number of negative eigenvalues of $Q^0$ allows us to identify the
objective function type, as in presence of at least one negative eigenvalue
the objective function is non convex. Finally, as far as the constraint types
are concerned, we collected the following information:
%
\begin{itemize}
 \item number of linear constraints, % (\emph{\# lin}),
 %
 \item number of quadratic constraints, % (\emph{\# quad}),
 %
 \item number of convex constraints, % (\emph{\# conv}),
 %
 \item number of box constraints. %(\emph{\# box}).
\end{itemize}
%
A constraint is considered quadratic if it contains at least one nonzero in
the quadratic term. Among the quadratic constraints, the ones with only
non-negative eigenvalues are classified as convex constraints. All this
information allowed us to analyze the gathered instances and to perform the
filters described in the the next paragraph.

%- - - - - - - - - - - - - - - - - - - - - - - - - - - - - - - - - - - -
\subsection{Instance Selection}\label{subsec:selection}

During the development of the library, a discussion ensued about what
the expected gols that we wanted to achieve. The following four goals
where finally identified:
%
\begin{enumerate}
 \item represent as much as possible all the different categories of QP
       problems;
 %
 \item gather ``challenging'' instances, i.e., ones which can not be easily
       solved by  state-of-the-art solvers;
 %
 \item include for each of the categories a set of well-diversified
       instances;
 %
 \item obtain a set of instances which is neither too small, so as to
       preserve statistical relevance, nor too large so as to being
       computationally manageable.
\end{enumerate}
%
To achieve the aforementioned goals, we performed the following two
filters, applied in cascade.
%
\begin{itemize}
 \item \emph{First Instances Filter.}\\
       The first filter was designed to drastically reduce the number of
       instances by eliminating the ``easy'' ones. An empirical measure
       for the hardness of an instance is the CPU time needed by a
       complete solver (cf.~\S \ref{sec:algo}) to solve it to
       global optimality. Accordingly, for each of the gathered instance we
       ran the complete solvers in {\tt Gams} (cf.~Table \ref{t:solvers})
       capable of solving it, whose number depends on the category of the
       instance under consideration. We then filtered according to a relative
       measure of computational difficulty, i.e., we discarded all instances
       that are solved by at least 30\% of the complete solvers within a time
       limit of 30 seconds.
 %
 \item \emph{Second Instances Filter.}\\
       The goal of the second filter was to eliminate ``similar'' instances.
       We carefully analysed the instances one by one, and we clustered them
       according to their features; for each cluster we kept only a few
       representatives. Finally, in order to only keep computationally
       challenging instances we ran a general purpose solver able to tackle
       all QP categories with a time limit of 120 seconds; all the instances
       which have been solved to proven optimality within this time limit were
       discarded.
\end{itemize}
%
In Table~\ref{tab:filters} we summarize the two filter steps, which
allowed us to identify the final set of $251$ discrete instances and
$116$ continuous instances.

\begin{center}
\begin{table}[]
 \centering
 \setlength{\tabcolsep}{5pt}
%  \arraystretch{1}
\begin{tabular}{cccc}
Starting set& \multicolumn{ 2}{c}{ $\approx$ 8500 Instances }& \\
& \multicolumn{ 2}{c}{$\Downarrow$}& \\
& $\approx$ 6000 Discr. Inst.  & $\approx$ 2500 Cont. inst. & \\
First Filter  & $\Downarrow$  & $\Downarrow$ & \\
 & $\approx$ 3000 Discr. Inst.  & $\approx$ 1000 Cont. Inst. & \\
Second Filter & $\Downarrow$  & $\Downarrow$  & \\
% & 600 Discr. Inst.  & 250  Cont. inst. & \\
  & 251 Discr. Inst.  & 116  Cont. inst. & \\
\end{tabular}
%\begin{center}\end{center}
\caption{Instance filter steps} \label{tab:filters}
\end{table}
\end{center}

%- - - - - - - - - - - - - - - - - - - - - - - - - - - - - - - - - - - -
\subsection{Analysis of the final set of instances}\label{subsec:final set}

In this section we analyse the feature of the instances selected to be
part of library.
The characteristics of the instances in the final library are
presented in Table \ref{tab:DD} for \emph{discrete} instances
(*\{B,M,I,G\}*) and in Table \ref{tab:CC} for
\emph{continuous} ones (*C*).
For each category, the Tables report in column $\#$ the corresponding
number of instances.
The final set well respects the original distribution of the gathered
instances among the different categories.
It is worth noticing that the classical discrete categories (such as
(L,M,Q) or (Q,B,L)) are well represented by $118$ and $59$  instances
respectively. On the other side, the classical continuous categories
(such as (L,C,Q) and (QCQ)) are also well represented by $50$ and $17$
 instances respectively.
Moreover, the library covers a large majority of all possible
categories of instances.


\begin{center}
\begin{table}
 \centering
 %\scriptsize
 \setlength{\tabcolsep}{18pt}
 \renewcommand \arraystretch{1.1}
\begin{tabular}{lllr}
\toprule
Obj. Fun. & Variables & Constraints & \#\\
\cmidrule(lr){1-4}
%
\multirow{5}*{Linear}
          & \multirow{1}*{Binary}
%                    & None      &   \\
%          &         & Linear    &  \\
                    & Quadratic &   9 \\[1.2 ex]
\cmidrule(lr){2-4}
          & \multirow{2}*{Mixed}
                    & Convex    &   2\\[1.2 ex]
          &         & Quadratic &    118\\[1.2 ex]
\cmidrule(lr){2-4}
          & \multirow{1}*{Integer}
%                    & Linear    &    \\
                   & Quadratic &    2\\[1.2 ex]
\cmidrule(lr){2-4}
          & \multirow{1}*{General}
%                    & Linear    &    \\
                   & Quadratic &    3\\[1.2 ex]
\cmidrule(lr){1-4}
%\multirow{5}*{Linear}
%          & Binary  & Quadratic &   9\\
%          & \multirow{2}*{Mixed}
%                    & Convex    &   2\\
%          &         & Quadratic &  118\\
%%          & \multirow{2}*{Integer}
%%                    & Convex    &  15 \\
%%          &         & Quadratic &   5 \\
%          & {General} & Quadratic &   3 \\
%\hline
\multirow{3}*{Convex}
          & Binary  & Linear    &  2 \\[1.2 ex]
\cmidrule(lr){2-4}
          & \multirow{2}*{Mixed}
                    & Linear    &   13\\[1.2 ex]
          &         & Quadratic &    1\\[1.2 ex]
%          & General & Linear    &    \\
%\hline
\cmidrule(lr){1-4}
\multirow{7}*{Quadratic}
          & \multirow{3}*{Binary}
                    & None      &   23\\[1.2 ex]
          &         & Linear    &  59\\[1.2 ex]
          &         & Quadratic &   5 \\[1.2 ex]
\cmidrule(lr){2-4}
          & \multirow{2}*{Mixed}
                    & Linear    &   10\\[1.2 ex]
          &         & Quadratic &    1\\[1.2 ex]
%          & Integer & Linear    &    \\
\cmidrule(lr){2-4}
          & \multirow{1}*{Integer}
                    & Linear    &    2\\[1.2 ex]
\cmidrule(lr){2-4}
          & \multirow{1}*{General}
                    & Quadratic    &    1\\[1.2 ex]
%          &         & Quadratic &    \\
\hline
Total     &         &           & 251\\
%
\bottomrule
\end{tabular}
\caption{Classification of the final set of discrete instances}\label{tab:DD}
\end{table}
\end{center}

\begin{center}
\begin{table}
 \centering
 %\scriptsize
 \setlength{\tabcolsep}{18pt}
 \renewcommand \arraystretch{1.1}
\begin{tabular}{llr}
\toprule
Obj. Fun. & Constraints & \#\\
\cmidrule(lr){1-3}
%
\multirow{2}*{Linear}    & Convex    &   11\\[1.2 ex]
                         & Quadratic &   50\\[1.2 ex]
\cmidrule(lr){1-3}
\multirow{4}*{Convex}
                         & Box       &   3 \\[1.2 ex]
                         & Linear    &   12\\[1.2 ex]
                         & Convex    &    2\\[1.2 ex]
                         & Quadratic &    5\\[1.2 ex]
\cmidrule(lr){1-3}
\multirow{3}*{Quadratic}
                         & Linear    &   5\\[1.2 ex]
                         & Convex    &   11\\[1.2 ex]
                         & Quadratic &   17\\[1.2 ex]
\hline
Total                    &           & 116 \\
%
\bottomrule
\end{tabular}
\caption{Classification of the final set of continuous instances}\label{tab:CC}
\end{table}
\end{center}








To graphically represent the library we report some
graphical representations of the instance main features.

In Figure~\ref{fig:distribution} we represent 
on the horizontal (resp. vertical) axis the number of variables (resp.
constrains). Both axis are in logarithmic scale.
The ``$+$'' points represent  the continuous instances, while the 
``$\times$'' points represent the discrete instances. The figure shows that
the library contains instances of different size in terms of variables
and constraints. The number of constraints goes from few units up to
one hundred thousands constraints, while the number of variables goes
up to almost forty-five thousands.

In Figures~\ref{fig:pic_var_small}~\ref{fig:pic_var_medium}~\ref{fig:pic_var_large},
for each instances we report the total number of variables with a
``$+$'' point and  the total number of discrete variables with a
``$\times$'' point (i.e. binary and integer variables). The instances are
sorted accordingly to the total number of variables and, for reason of
readability, we splitted the figure into three sub figures. The first one hundred
instances are presented in Figures~\ref{fig:pic_var_small},
then the second two hundred instances are presented in
Figure~\ref{fig:pic_var_medium} and finally, the remaining $67$
instances are presented in Figure~\ref{fig:pic_var_large}.
From the picture we evince that for all instance dimensions it is
possible to find discrete instances with different percentages of
continuous variables and purely continuous ones.



In Figures~\ref{fig:pic_constr_small}~\ref{fig:pic_constr_medium}~\ref{fig:pic_constr_big},
for each instances we report the total number of constraints with a
``$+$'' point and the total number of quadratic (either convex of non
convex) constraints with a ``$\times$'' point. The instances are sorted
accordingly to the total number of constraints  and divided in the
same manner as for Figures~\ref{fig:pic_var_small}~\ref{fig:pic_var_medium}~\ref{fig:pic_var_large}.
Also in this case, we can see that for all sizes it is possible to
find instances with only linear constraints and instances with
different percentage of linear and quadratic constraints.

In Figures~\ref{fig:pic_density} and~\ref{fig:pic_neg_eig},  we
focused on the instances with a quadratic objective function.
In Figures~\ref{fig:pic_density} we plot the density of the $Q^0$
matrix and we sort the instance according to it. We observe that the
majority of the instances have either a low or high density. However,
also intermediate values are present. It is worth noticing that around 60 instances have a 
density less that 5\% (see the Appendix for further details). 
In Figures~\ref{fig:pic_neg_eig} we present  the percentage of
negative eigenvalues and we sort the instance according to it. As
expected, a vast majority of the instances have $50\% $ of negative
eigenvalues. In addition, around $40$ instances are convex (i.e. $0\%$
of negative eigenvalues) but also instances of high or low percentages
of negative eigenvalues are present in the library.


%- - - - - - - - - - - - - - - - - - - - - - - - - - - - - - - - - - - -



%%%%%%%%%%%%%%%%%%%%%%%%%%%%%%%%%%%
\begin{figure}\centering
  \includegraphics[width=0.85\textwidth]{pic_overview.png}
  \caption{Distribution of variables and constrains  of the qplib
instances \label{fig:1}}
\end{figure}

%%%%%%%%%%%%%%%%%%%%%%%%%%%%%%%
\begin{figure}\centering
  \includegraphics[width=0.85\textwidth]{pic_var_small.png}
  \caption{Number of variables (a) \label{fig:pic_var_small}}
\end{figure}

\begin{figure}\centering
  \includegraphics[width=0.85\textwidth]{pic_var_medium.png}
  \caption{Number of variables (b) \label{fig:pic_var_medium}}
\end{figure}

\begin{figure}\centering
  \includegraphics[width=0.85\textwidth]{pic_var_big.png}
  \caption{Number of variables (c) \label{fig:pic_var_large}}
\end{figure}

%%%%%%%%%%%%%%%%%%%%%%%%%%%%%%%%%%%%%%%
\begin{figure}\centering
  \includegraphics[width=0.85\textwidth]{pic_constr_small.png}
  \caption{Number of constraints (a) \label{fig:pic_constr_small}}
\end{figure}

\begin{figure}\centering
  \includegraphics[width=0.85\textwidth]{pic_constr_medium.png}
  \caption{Number of constraints (b) \label{fig:pic_constr_medium}}
\end{figure}

\begin{figure}\centering
  \includegraphics[width=0.85\textwidth]{pic_constr_big.png}
  \caption{Number of constraints (c) \label{fig:pic_constr_big}}
\end{figure}
%%%%%%%%%%%%%%%%%%%%%%%%%%%%%%%%%%%%%%%%%

%\begin{figure}\centering
%  \includegraphics[width=0.85\textwidth]{pic_quad_vs_convex_constr.png}
%  \caption{Number of quadratic constraints (convex and non-convex)
%\label{fig:pic_quad_vs_convex_constr}}
%\end{figure}
%%%%%%%%%%%%%%%%%%%%%%%%%%%%%%%%%%%%%%%%


%\begin{figure}\centering
%  \includegraphics[width=0.85\textwidth]{pic_fuffa.png}
%  \caption{...\label{fig:9}}
%\end{figure}

\begin{figure}\centering
  \includegraphics[width=0.85\textwidth]{pic_density.png}
  \caption{Density \label{fig:pic_density}}
\end{figure}

\begin{figure}\centering
  \includegraphics[width=0.85\textwidth]{pic_neg_eig.png}
  \caption{Negative eigenvalues \label{fig:pic_neg_eig}}
\end{figure}









%- - - - - - - - - - - - - - - - - - - - - - - - - - - - - - - - - - - -








%- - - - - - - - - - - - - - - - - - - - - - - - - - - - - - - - - - - -
%- - - - - - - - - - - - - - - - - - - - - - - - - - - - - - - - - - - -
%  End QPLIB-3.tex
%- - - - - - - - - - - - - - - - - - - - - - - - - - - - - - - - - - - -
%- - - - - - - - - - - - - - - - - - - - - - - - - - - - - - - - - - - -
