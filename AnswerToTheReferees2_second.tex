\documentclass[11pt]{article}
\usepackage[american]{babel}
%\usepackage{a4wide,paralist,url,xcolor,xspace}
\usepackage{a4wide,url,xcolor,xspace}
\definecolor{acblue}{rgb}{0,0,1}
\definecolor{acgreen}{rgb}{0,1,0}
\definecolor{acred}{rgb}{1,0,0}
\newcommand{\rep}[1]{{\textcolor{acblue}{#1}}}
%\newcommand{\comm}[1]{{\textcolor{acgreen}{#1}}}
\newcommand{\comm}[1]{}
\newcommand{\todo}[1]{{\textcolor{acred}{#1}}}

\usepackage{amsmath}
\usepackage{amsfonts}
\usepackage{booktabs}
\usepackage{multirow}

\newcommand{\eg}{e.g.,\xspace}
\newcommand{\ie}{i.e.,\xspace}

\parindent0pt
\parskip.8ex


\newcommand{\leo}[1]{{\color{red}#1}}
\newcommand{\fabio}[1]{{\color{cyan}#1}}
\newcommand{\emiliano}[1]{{\color{blue}#1}}   
\newcommand{\antonio}[1]{{\color{ForestGreen}#1}}     
\newcommand{\andrea}[1]{{\color{blue}{\bf #1}}}   

\begin{document}

\section*{QPLIB: a Library of a Quadratic Programming Instances (MPC-022017-00003)-- Comments of the minor revision}

\rep{
 Following the Associate and Area Editor suggestions, we 
 performed a minor revision of our manuscript. We carefully took into account all the comments made by the AE.
 The new submitted document highlights in red the text that has been
added or changed.
In the following, the comments of the AE, together with our point-by-point replies.
}


\subsection*{Reply to the Associate Editor}

{\it
I thank the authors for their thorough revision. One of the two reviewers is now satisfied with the revision and is recommending acceptance. The other reviewer has some remaining concerns. I disagree with most of those concerns, however, and am nearly ready to recommend this paper for acceptance in MPC. I would like to ask the authors to make one additional minor revision, addressing the following minor comments (some are from the referee, others are my own):  
}

\rep{We would like to thank the AE for his/her positive opinion on the revision we performed of our manuscript. 
%\leo{Togliamo la spiegazione degli algoritmi e sostituiamola con una citazione di un book-tutorial-survey-introduction.}
}

\bigskip
\textbf{Main points:}

\begin{itemize}
\item 
{\it
[from referee, paraphrased] 8. Please try to find a more precise definition of asypmtotically complete algorithm to 
cite.
} 

\rep{\dots}

\item 
{\it
[from referee, paraphrased] 9. Please add more explanation of how instances were determined to be similar. 
}

\rep{\dots}


\item 
{\it
The paragraph toward the bottom of page 19, beginning "One of the nontrivial choices..." is very nearly identical to a paragraph on page 10. Please remove this repitition. 
}

\rep{\dots}


\item Page 18, first instances filter: The way this is written it is not clear if the stated rule (30\% of complete solvers within a time limit of 30 seconds) was what was actually used, or if it an example of what might have been used. Please clarify. 

\rep{\dots}

 
\end{itemize}

\bigskip
\textbf{Errata:}

 \begin{itemize}
 
\item 
{\it
- p9, l4. Please remove the word "downright" here. While I appreciate the colloquial style, I'm afraid this might actually cause some confusion. 
}

\rep{\dots}

\item 
{\it
- Please consider searching for the word "obvious" in the paper, and removing it. 
}

\rep{\dots}

\item 
{\it

- p17, l12. Something wrong with this sentence: perhaps ``to use GAMS'' should be ``we use GAMS''? 
}

\rep{\dots}

\item 
{\it

- p18, l5-6. You could perhaps leave out the part about a "discussion ensued" and simply start this section with something like: "We chose instances based on the following four goals:" 
}

\rep{\dots}

 
\end{itemize}


%%%%%%%%%%%%%%%%%%%%%%%%%%%%%%%%%%%%%%%%%%%%%%%%%%%%%%%%%%%%%%%%%%%55
\subsection*{Point-by-point reply to the Referee 2}

\bigskip
\textbf{
Detailed comments on this first revision: 
}



 \begin{enumerate}
 
 
\item 
{\it
 How do you propose to update the library? The criteria for problems include 
the solution time required for these problem. Presumably, as better algorithms 
are developed, these criteria change. Please comment on how the library will 
be updated. 
}

\rep{\dots}

\item 
{\it
 P.6. Please change the first sentence in Section 2.1. to be something like 
"We refer to quadratic programs (QPs) as ..." This is YOUR definition of QPs, 
which does not agree with any current textbook definitions, so you need to 
make it clear that you change QPs to means MIQCQPs, or drop QPs (better). 
}

\rep{\dots}

\item 
{\it
 The statement on p.8, that "QPs with both finite bounds cannot ever be convex" 
is wrong. Consider the set $$\{ x | 1 <= x^2 <= 2, x=1.5\}$$ which is trivially 
convex and has $Q<>0$. I would say that quadratic constraints with finite lower 
and upper bounds are typically nonconvex. 
}

\rep{\dots}

\item 
{\it
 P.9. The statement that "there is no loss of generality in assuming a linear 
objective" is somewhat inaccurate, because algorithms behave differently if 
the objective is linearized in this way. 
}

\rep{\dots}

\item 
{\it
 The fact that the library does not support matrix-free solvers is somewhat 
disappointing. Ideally, a library should be agnostic to these solver requirements, 
as AMPL is, for example. 
}

\rep{\dots}

\item 
{\it
 You must justify the statement (p.11) that QGQ "is the most general and therefore 
the most difficult class". Generality alone does not increase difficulty. Instead, 
I would add a reference to the classical result by Jeroslov that these problems 
are undecidable, which makes them more difficult than NP-hard problems. 
}

\rep{\dots}

\item 
{\it
 Add references to surveys on QP methods to your review of methods in Section 2.3. 
}

\rep{\dots}

\item 
{\it
 The definition of asymptotically complete methods in Section 2.3.3 is very sloppy, 
and must be made more precise (mathematical). Something like: for every $\epsilon>0$ 
there exist $T>0$ such that solution is $\epsilon$ optimal after time $T$ and 
$\lim_{T\to\infty}$ ... Either provide a reference to a rigorous definition, or create 
your own. 
}

\rep{\dots}

\item 
{\it
 Section 3.2 on Instance Selection is not reproducible, and somewhat sloppy. The authors 
must describe properly how the instances were selected. The current lack of transparency 
is not acceptable (it might leave contributors with the feeling that their problem was 
not selected for obscure reasons). 
}


\rep{\dots}

\item 
{\it
Please address how you decided that problems were "similar" in the second phase? 
In particular, what were the "features" in the clustering approach. If these features 
are the characteristics described earlier, then this selection is fundamentally flawed, 
because problems with the same number of variables and constraints are not even remotely 
similarly difficult to solve ... just look at the HS selection in CUTEst. 
}

\rep{\dots}

\item 
{\it
 Following on from the previous point is my most serious concern. There are seven classes 
of problems that have three or fewer instances. Some classes have only a single instance. 
Testing any algorithm on a single instance is totally inadequate. This imbalance of 
test problems must be addressed. You can either go back to your filtering and find more 
instances, or contact the people who submitted these problems, or create new ones. 
Tables 5 and 6 on p.20 should show a healthy number of test problems in each class (I 
would like to see a dozen, but I am OK with fewer, 1-3 is not OK).
}

\rep{\dots}

 \end{enumerate}


\end{document}

%%% Local Variables: 
%%% mode: latex
%%% TeX-master: t
%%% End: 
