%- - - - - - - - - - - - - - - - - - - - - - - - - - - - - - - - - - - -
%- - - - - - - - - - - - - - - - - - - - - - - - - - - - - - - - - - - -
%  Appendix.tex
%- - - - - - - - - - - - - - - - - - - - - - - - - - - - - - - - - - - -
%- - - - - - - - - - - - - - - - - - - - - - - - - - - - - - - - - - - -

\section{Instance details}\label{sec:instance_details}


Table \ref{tab:A1} provides detailed data on all the instances of the final library.
Column ``name'' is the name of the instance with the prefix ``QPLIB\_'' stripped.
Column ``type'' is the classification of the instance according to the taxonomy from \S\ref{ssec:taxonomy}.
Column ``\%~h.e.'' provides the fraction of hard eigenvalues of $Q^0$, the coefficient matrix of the objective function: a positive number implies that the instance is a Q**, ``0.0'' implies that the instance is a C**, a blank implies that $Q^0 = 0$, i.e., the objective function is linear (hence, the instance is a L**).
Column ``\%~d.'' describes the density of the $Q^0$ matrix: a blank implies that the corresponding instance has a linear objective function.
For both columns (``\%~n.e.'' and ``\%~d.''), nonzeros values below $0.1$ were rounded up to $0.1$.
The following three columns describe the variables by reporting the number of binary ones (``\#~b.''), general integer ones (``\#~i.''), and continuous ones (``\#~c.''). Finally, the last four columns describe the constraints reporting the number of linear ones (``\#~l.''), nonconvex quadratic ones (``\#~q.''), convex quadratic ones (``\#~c.''), and variable bounds (``\#~v.'').
\textcolor{red}{The numbering of the instances reflects the initial order in which we gathered them and the non-consecutiveness of the instance names is due to the filtering.}


{\tiny
\begin{longtable}[c]{llrrrrrrrrrr}
\caption{Features of QPLIB instances.} \\
\toprule
& & \multicolumn{2}{c}{$Q^0$} & \multicolumn{3}{c}{Variables} & \multicolumn{4}{c}{Constraints} \\
\cmidrule(lr){3-4} \cmidrule(lr){5-7} \cmidrule(lr){8-11}
name & type & \% h.e. & \% d. & \# b. & \# i. & \# c. & \# l. & \# q. & \# c. & \# v. \\
\midrule
\endfirsthead

\caption{Features of QPLIB instances (continued).} \\
\toprule
& & \multicolumn{2}{c}{$Q^0$} & \multicolumn{3}{c}{Variables} & \multicolumn{4}{c}{Constraints} \\
\cmidrule(lr){3-4} \cmidrule(lr){5-7} \cmidrule(lr){8-11}
name & type & \% h.e. & \% d. & \# b. & \# i. & \# c. & \# l. & \# q. & \# c. & \# v. \\
\midrule
\endhead

0018 & QCL & 48.0 & 100.0 & 0 & 0 & 50 & 1 & 0 & 0 & 50 \\
0031 & QML & 18.3 & 99.8 & 30 & 0 & 30 & 32 & 0 & 0 & 30 \\
0032 & QML & 25.0 & 99.9 & 50 & 0 & 50 & 52 & 0 & 0 & 50 \\
0067 & QBL & 47.5 & 88.9 & 80 & 0 & 0 & 1 & 0 & 0 & 0 \\
0343 & QCL & 48.0 & 100.0 & 0 & 0 & 50 & 1 & 0 & 0 & 100 \\
0633 & QBL & 58.7 & 98.7 & 75 & 0 & 0 & 1 & 0 & 0 & 0 \\
0678 & LMQ & & & 9600 & 0 & 5537 & 7457 & 960 & 0 & 1474 \\
0681 & LMQ & & & 72 & 0 & 143 & 419 & 48 & 0 & 200 \\
0682 & LMQ & & & 71 & 0 & 190 & 501 & 96 & 0 & 296 \\
0684 & LMQ & & & 101 & 0 & 260 & 815 & 128 & 0 & 408 \\
0685 & LMQ & & & 256 & 0 & 519 & 1603 & 192 & 0 & 728 \\
0686 & LMQ & & & 692 & 0 & 1512 & 4440 & 640 & 0 & 2200 \\
0687 & LMQ & & & 672 & 0 & 1651 & 4875 & 800 & 0 & 2520 \\
0688 & LMQ & & & 1964 & 0 & 3824 & 20568 & 1600 & 0 & 6256 \\
0689 & LMQ & & & 756 & 0 & 1112 & 9800 & 288 & 0 & 1608 \\
0690 & LMQ & & & 6428 & 0 & 10048 & 112400 & 3200 & 0 & 17376 \\
0696 & LMQ & & & 187 & 0 & 207 & 390 & 33 & 0 & 260 \\
0698 & LMQ & & & 55 & 0 & 63 & 126 & 15 & 0 & 56 \\
0752 & QBL & 50.0 & 10.0 & 250 & 0 & 0 & 1 & 0 & 0 & 0 \\
0911 & QCQ & 44.0 & 50.5 & 0 & 0 & 50 & 0 & 50 & 0 & 100 \\
0975 & QCQ & 50.0 & 50.6 & 0 & 0 & 50 & 0 & 10 & 0 & 100 \\
1055 & QCQ & 50.0 & 100.0 & 0 & 0 & 40 & 0 & 20 & 0 & 80 \\
1143 & QCQ & 50.0 & 97.1 & 0 & 0 & 40 & 4 & 20 & 0 & 80 \\
1157 & QCC & 25.0 & 94.5 & 0 & 0 & 40 & 8 & 0 & 1 & 80 \\
1353 & QCC & 26.0 & 95.8 & 0 & 0 & 50 & 5 & 0 & 1 & 100 \\
1423 & QCQ & 75.0 & 95.4 & 0 & 0 & 40 & 4 & 11 & 9 & 80 \\
1437 & QCQ & 50.0 & 95.6 & 0 & 0 & 50 & 10 & 1 & 0 & 100 \\
1451 & QCQ & 50.0 & 49.1 & 0 & 0 & 60 & 6 & 60 & 0 & 120 \\
1493 & QCQ & 50.0 & 97.3 & 0 & 0 & 40 & 4 & 1 & 0 & 80 \\
1507 & QCQ & 26.7 & 95.8 & 0 & 0 & 30 & 3 & 22 & 8 & 60 \\
1535 & QCQ & 50.0 & 94.3 & 0 & 0 & 60 & 6 & 60 & 0 & 120 \\
1619 & QCQ & 50.0 & 95.5 & 0 & 0 & 50 & 5 & 25 & 0 & 100 \\
1661 & QCQ & 50.0 & 95.4 & 0 & 0 & 60 & 12 & 1 & 0 & 120 \\
1675 & QCQ & 51.7 & 48.8 & 0 & 0 & 60 & 12 & 1 & 0 & 120 \\
1703 & QCQ & 51.7 & 97.9 & 0 & 0 & 60 & 6 & 30 & 0 & 120 \\
1745 & QCQ & 50.0 & 48.8 & 0 & 0 & 50 & 5 & 50 & 0 & 100 \\
1773 & QCQ & 50.0 & 94.8 & 0 & 0 & 60 & 6 & 1 & 0 & 120 \\
1886 & QCQ & 50.0 & 50.0 & 0 & 0 & 50 & 0 & 50 & 0 & 100 \\
1913 & QCQ & 50.0 & 24.9 & 0 & 0 & 48 & 0 & 48 & 0 & 96 \\
1922 & QCQ & 50.0 & 49.6 & 0 & 0 & 30 & 0 & 60 & 0 & 60 \\
1931 & QCQ & 50.0 & 49.9 & 0 & 0 & 40 & 0 & 40 & 0 & 80 \\
1940 & QCQ & 50.0 & 25.0 & 0 & 0 & 48 & 0 & 96 & 0 & 96 \\
1967 & QCQ & 50.0 & 99.8 & 0 & 0 & 50 & 0 & 75 & 0 & 100 \\
1976 & QBQ & 38.2 & 7.0 & 152 & 0 & 0 & 136 & 16 & 0 & 0 \\
2017 & QBQ & 39.3 & 5.5 & 252 & 0 & 0 & 231 & 21 & 0 & 0 \\
2022 & QBQ & 38.5 & 5.2 & 275 & 0 & 0 & 253 & 22 & 0 & 0 \\
2029 & QBQ & 40.1 & 5.1 & 299 & 0 & 0 & 276 & 23 & 0 & 0 \\
2036 & QBQ & 39.2 & 4.8 & 324 & 0 & 0 & 300 & 24 & 0 & 0 \\
2047 & LBQ & & & 136 & 0 & 0 & 2040 & 17 & 0 & 0 \\
2055 & LBQ & & & 153 & 0 & 0 & 2448 & 18 & 0 & 0 \\
2060 & LBQ & & & 171 & 0 & 0 & 2907 & 19 & 0 & 0 \\
2067 & LBQ & & & 190 & 0 & 0 & 3420 & 20 & 0 & 0 \\
2073 & LBQ & & & 210 & 0 & 0 & 3990 & 21 & 0 & 0 \\
2077 & LBQ & & & 231 & 0 & 0 & 4620 & 22 & 0 & 0 \\
2085 & LBQ & & & 253 & 0 & 0 & 5313 & 23 & 0 & 0 \\
2087 & LBQ & & & 276 & 0 & 0 & 6072 & 24 & 0 & 0 \\
2096 & LBQ & & & 300 & 0 & 0 & 6900 & 25 & 0 & 0 \\
2165 & LMQ & & & 683 & 0 & 1376 & 1366 & 683 & 0 & 683 \\
2166 & LMQ & & & 345 & 0 & 697 & 690 & 345 & 0 & 345 \\
2167 & LMQ & & & 61 & 0 & 131 & 122 & 61 & 0 & 61 \\
2168 & LMQ & & & 214 & 0 & 438 & 428 & 214 & 0 & 214 \\
2169 & LMQ & & & 297 & 0 & 608 & 594 & 297 & 0 & 297 \\
2170 & LMQ & & & 351 & 0 & 736 & 702 & 351 & 0 & 351 \\
2171 & LMQ & & & 150 & 0 & 305 & 300 & 150 & 0 & 150 \\
2173 & LMQ & & & 215 & 0 & 436 & 430 & 215 & 0 & 215 \\
2174 & LMQ & & & 768 & 0 & 1545 & 1536 & 768 & 0 & 768 \\
2181 & LMQ & & & 90 & 0 & 190 & 180 & 90 & 0 & 90 \\
2187 & LMQ & & & 90 & 0 & 195 & 180 & 90 & 0 & 90 \\
2192 & LMQ & & & 90 & 0 & 200 & 180 & 90 & 0 & 90 \\
2195 & LMQ & & & 90 & 0 & 205 & 180 & 90 & 0 & 90 \\
2202 & LMQ & & & 90 & 0 & 185 & 180 & 90 & 0 & 90 \\
2203 & LMQ & & & 100 & 0 & 205 & 200 & 100 & 0 & 100 \\
2204 & LMQ & & & 110 & 0 & 225 & 220 & 110 & 0 & 110 \\
2205 & LMQ & & & 958 & 0 & 1926 & 1916 & 958 & 0 & 958 \\
2206 & LMQ & & & 194 & 0 & 421 & 388 & 194 & 0 & 194 \\
2315 & QBL & 44.7 & 7.5 & 595 & 0 & 0 & 13090 & 0 & 0 & 0 \\
2353 & QML & 50.0 & 23.7 & 147 & 0 & 93 & 2240 & 0 & 0 & 186 \\
2357 & QBL & 50.0 & 7.8 & 240 & 0 & 0 & 2240 & 0 & 0 & 0 \\
2359 & QBL & 44.4 & 4.2 & 306 & 0 & 0 & 3264 & 0 & 0 & 0 \\
2416 & LCQ & & & 0 & 0 & 25 & 153 & 527 & 6 & 48 \\
2430 & LCQ & & & 0 & 0 & 125 & 27 & 65 & 0 & 240 \\
2445 & LCQ & & & 0 & 0 & 143 & 14 & 66 & 0 & 160 \\
2456 & LCD & & & 0 & 0 & 5477 & 4131 & 0 & 1369 & 0 \\
2468 & LCD & & & 0 & 0 & 14885 & 11203 & 0 & 3721 & 0 \\
2480 & LCQ & & & 0 & 0 & 399 & 199 & 200 & 1 & 400 \\
2482 & LCD & & & 0 & 0 & 1806 & 1418 & 0 & 361 & 0 \\
2483 & LCQ & & & 0 & 0 & 760 & 40 & 240 & 0 & 1320 \\
2492 & QBL & 25.5 & 86.2 & 196 & 0 & 0 & 28 & 0 & 0 & 0 \\
2505 & LCQ & & & 0 & 0 & 1039 & 302 & 480 & 0 & 540 \\
2512 & QBL & 46.0 & 77.4 & 100 & 0 & 0 & 20 & 0 & 0 & 0 \\
2519 & LCD & & & 0 & 0 & 4806 & 3802 & 0 & 961 & 0 \\
2540 & LCQ & & & 0 & 0 & 498 & 341 & 210 & 0 & 130 \\
2546 & CCQ & 0.0 & 0.7 & 0 & 0 & 1015 & 592 & 400 & 0 & 15 \\
2590 & LCQ & & & 0 & 0 & 25 & 93 & 401 & 0 & 48 \\
2626 & LCD & & & 0 & 0 & 22327 & 14763 & 0 & 3721 & 0 \\
2635 & LCC & & & 0 & 0 & 176 & 0 & 0 & 1154 & 0 \\
2650 & LCQ & & & 0 & 0 & 1110 & 228 & 904 & 0 & 1072 \\
2658 & LCQ & & & 0 & 0 & 184 & 57 & 133 & 0 & 192 \\
2676 & LCD & & & 0 & 0 & 1445 & 1095 & 0 & 361 & 0 \\
2693 & LCQ & & & 0 & 0 & 791 & 183 & 631 & 0 & 754 \\
2696 & QCQ & 1.0 & 2.5 & 0 & 0 & 3500 & 1995 & 1500 & 0 & 5 \\
2698 & LCQ & & & 0 & 0 & 196 & 36 & 11 & 0 & 280 \\
2702 & QML & 4.6 & 1.2 & 259 & 0 & 1 & 212 & 0 & 0 & 0 \\
2703 & LCQ & & & 0 & 0 & 799 & 399 & 400 & 1 & 800 \\
2707 & LCQ & & & 0 & 0 & 634 & 151 & 466 & 0 & 640 \\
2708 & LMQ & & & 108 & 0 & 526 & 327 & 30 & 0 & 520 \\
2712 & QCL & 50.0 & 100.0 & 0 & 0 & 200 & 1 & 0 & 0 & 400 \\
2714 & LCQ & & & 0 & 0 & 352 & 301 & 298 & 0 & 1 \\
2733 & QBL & 25.9 & 89.2 & 324 & 0 & 0 & 36 & 0 & 0 & 0 \\
2738 & LCQ & & & 0 & 0 & 199 & 99 & 100 & 1 & 200 \\
2758 & LCQ & & & 0 & 0 & 303 & 139 & 112 & 0 & 140 \\
2761 & QCL & 50.0 & 100.0 & 0 & 0 & 500 & 1 & 0 & 0 & 1000 \\
2784 & LCD & & & 0 & 0 & 4501 & 3680 & 0 & 900 & 0 \\
2819 & LCQ & & & 0 & 0 & 334 & 24 & 132 & 0 & 500 \\
2823 & LCQ & & & 0 & 0 & 390 & 103 & 283 & 0 & 396 \\
2834 & LCQ & & & 0 & 0 & 156 & 14 & 72 & 0 & 200 \\
2862 & LCD & & & 0 & 0 & 40501 & 32640 & 0 & 8100 & 0 \\
2880 & QBL & 48.8 & 90.3 & 625 & 0 & 0 & 50 & 0 & 0 & 0 \\
2881 & LCC & & & 0 & 0 & 1512 & 0 & 0 & 720 & 0 \\
2882 & LMQ & & & 56 & 0 & 88 & 257 & 16 & 0 & 32 \\
2894 & LCQ & & & 0 & 0 & 17 & 55 & 154 & 0 & 32 \\
2935 & LMQ & & & 72 & 0 & 108 & 325 & 18 & 0 & 36 \\
2957 & QBL & 23.1 & 60.3 & 484 & 0 & 0 & 44 & 0 & 0 & 0 \\
2958 & LMQ & & & 42 & 0 & 70 & 197 & 14 & 0 & 28 \\
2967 & QCC & 47.4 & 5.0 & 0 & 0 & 38 & 1 & 0 & 190 & 38 \\
2981 & CCQ & 0.0 & 0.7 & 0 & 0 & 2015 & 1192 & 800 & 0 & 15 \\
2987 & LCQ & & & 0 & 0 & 208 & 114 & 90 & 0 & 90 \\
2993 & LCQ & & & 0 & 0 & 266 & 235 & 84 & 0 & 66 \\
3029 & LCD & & & 0 & 0 & 5767 & 3783 & 0 & 961 & 0 \\
3034 & LCQ & & & 0 & 0 & 780 & 40 & 240 & 0 & 1320 \\
3049 & QCQ & 0.4 & 2.5 & 0 & 0 & 7000 & 3995 & 3000 & 0 & 5 \\
3060 & QML & 0.2 & 6.2 & 48 & 0 & 792 & 1192 & 0 & 0 & 0 \\
3080 & CCQ & 0.0 & 0.7 & 0 & 0 & 4015 & 2392 & 1600 & 0 & 15 \\
3083 & LCQ & & & 0 & 0 & 243 & 107 & 126 & 0 & 120 \\
3088 & LCD & & & 0 & 0 & 3601 & 2780 & 0 & 900 & 0 \\
3089 & LCQ & & & 0 & 0 & 132 & 12 & 72 & 0 & 228 \\
3105 & LCD & & & 0 & 0 & 18606 & 14802 & 0 & 3721 & 0 \\
3120 & LCQ & & & 0 & 0 & 662 & 40 & 204 & 0 & 924 \\
3122 & QML & 2.8 & 0.1 & 17136 & 0 & 3988 & 36703 & 0 & 0 & 0 \\
3147 & LCQ & & & 0 & 0 & 419 & 32 & 108 & 0 & 550 \\
3170 & LCQ & & & 0 & 0 & 660 & 40 & 160 & 0 & 1160 \\
3177 & LCQ & & & 0 & 0 & 1599 & 799 & 800 & 1 & 1600 \\
3181 & LMQ & & & 84 & 0 & 308 & 180 & 16 & 0 & 222 \\
3185 & LCD & & & 0 & 0 & 18001 & 14560 & 0 & 3600 & 0 \\
3192 & LCQ & & & 0 & 0 & 479 & 32 & 145 & 0 & 702 \\
3225 & LCQ & & & 0 & 0 & 136 & 14 & 66 & 0 & 160 \\
3240 & LCQ & & & 0 & 0 & 516 & 187 & 220 & 0 & 260 \\
3247 & LCQ & & & 0 & 0 & 361 & 322 & 8 & 148 & 1 \\
3279 & LMQ & & & 56 & 0 & 251 & 148 & 16 & 0 & 222 \\
3297 & CCQ & 0.0 & 0.7 & 0 & 0 & 8015 & 4792 & 3200 & 0 & 15 \\
3307 & QBL & 19.9 & 61.5 & 256 & 0 & 0 & 32 & 0 & 0 & 0 \\
3312 & LCD & & & 0 & 0 & 41406 & 33002 & 0 & 8281 & 0 \\
3318 & LCQ & & & 0 & 0 & 25 & 93 & 381 & 0 & 48 \\
3326 & QCQ & 1.4 & 2.5 & 0 & 0 & 1750 & 995 & 750 & 0 & 5 \\
3334 & LCQ & & & 0 & 0 & 715 & 40 & 210 & 0 & 990 \\
3337 & LCQ & & & 0 & 0 & 297 & 0 & 198 & 0 & 396 \\
3338 & LCQ & & & 0 & 0 & 320 & 26 & 110 & 0 & 432 \\
3347 & QBL & 51.8 & 85.8 & 676 & 0 & 0 & 52 & 0 & 0 & 0 \\
3358 & LCQ & & & 0 & 0 & 158 & 66 & 106 & 0 & 136 \\
3361 & QBL & 10.0 & 35.5 & 1024 & 0 & 0 & 64 & 0 & 0 & 0 \\
3369 & LCQ & & & 0 & 0 & 485 & 32 & 116 & 0 & 650 \\
3380 & QBL & 3.4 & 0.1 & 8904 & 0 & 0 & 823 & 0 & 0 & 0 \\
3385 & LCQ & & & 0 & 0 & 155 & 77 & 60 & 0 & 80 \\
3387 & LCQ & & & 0 & 0 & 170 & 18 & 65 & 0 & 160 \\
3402 & QBL & 47.2 & 81.5 & 144 & 0 & 0 & 24 & 0 & 0 & 0 \\
3413 & QBL & 45.0 & 9.0 & 400 & 0 & 0 & 40 & 0 & 0 & 0 \\
3416 & LCQ & & & 0 & 0 & 424 & 32 & 96 & 0 & 400 \\
3496 & LGQ & & & 200 & 56 & 72 & 623 & 64 & 0 & 120 \\
3502 & LMQ & & & 10920 & 0 & 2090 & 209 & 3130 & 0 & 2090 \\
3505 & LMQ & & & 201 & 0 & 603 & 605 & 2 & 0 & 2 \\
3506 & QBN & 48.4 & 0.8 & 496 & 0 & 0 & 0 & 0 & 0 & 0 \\
3508 & LMQ & & & 2450 & 0 & 891 & 99 & 1332 & 0 & 891 \\
3510 & LMQ & & & 105 & 0 & 919 & 4568 & 21 & 0 & 38 \\
3511 & LMQ & & & 2450 & 0 & 3292 & 4950 & 1283 & 0 & 891 \\
3512 & LMQ & & & 72 & 0 & 119 & 403 & 24 & 0 & 152 \\
3513 & LMQ & & & 123 & 0 & 1897 & 2569 & 763 & 0 & 1880 \\
3514 & LMQ & & & 15 & 0 & 1800 & 960 & 900 & 0 & 1800 \\
3515 & LMQ & & & 352 & 0 & 382 & 720 & 48 & 0 & 540 \\
3522 & LMQ & & & 42 & 0 & 588 & 212 & 42 & 0 & 588 \\
3523 & QML & 50.0 & 13.2 & 155 & 0 & 27 & 1456 & 0 & 0 & 54 \\
3524 & LMQ & & & 132 & 0 & 949 & 3165 & 192 & 0 & 288 \\
3525 & QGQ & 47.5 & 0.1 & 0 & 1662 & 87 & 52 & 39 & 0 & 3324 \\
3529 & LMQ & & & 38 & 0 & 1488 & 1580 & 544 & 0 & 800 \\
3533 & LMQ & & & 240 & 0 & 143 & 176 & 25 & 0 & 8 \\
3547 & DML & 0.0 & 16.7 & 462 & 0 & 1536 & 3137 & 0 & 0 & 6 \\
3549 & LMQ & & & 650 & 0 & 1033 & 1326 & 583 & 0 & 408 \\
3554 & QML & 3.9 & 100.0 & 14 & 0 & 370 & 556 & 0 & 0 & 0 \\
3562 & LIQ & & & 7 & 56 & 0 & 35 & 7 & 0 & 112 \\
3565 & QBN & 47.8 & 1.4 & 276 & 0 & 0 & 0 & 0 & 0 & 0 \\
3580 & LMQ & & & 108 & 0 & 24 & 45 & 18 & 0 & 24 \\
3582 & LMQ & & & 184 & 0 & 32 & 60 & 24 & 0 & 32 \\
3584 & QBL & 43.9 & 8.0 & 528 & 0 & 0 & 10912 & 0 & 0 & 0 \\
3587 & QBL & 50.0 & 12.7 & 240 & 0 & 0 & 46 & 0 & 0 & 0 \\
3588 & LMQ & & & 600 & 0 & 392 & 49 & 584 & 0 & 392 \\
3592 & QML & 50.0 & 0.2 & 225 & 0 & 225 & 255 & 0 & 0 & 0 \\
3596 & LMQ & & & 104 & 0 & 921 & 1054 & 132 & 0 & 428 \\
3600 & LMQ & & & 112 & 0 & 16 & 45 & 12 & 0 & 16 \\
3605 & LMQ & & & 160 & 0 & 1076 & 4315 & 192 & 0 & 288 \\
3614 & QBL & 50.0 & 12.7 & 210 & 0 & 0 & 44 & 0 & 0 & 0 \\
3620 & LMQ & & & 187 & 0 & 3285 & 4071 & 1344 & 0 & 3398 \\
3621 & LMQ & & & 109 & 0 & 1655 & 2213 & 665 & 0 & 1624 \\
3622 & LMQ & & & 25 & 0 & 2000 & 1040 & 1000 & 0 & 2000 \\
3624 & LMQ & & & 40 & 0 & 6400 & 3280 & 3200 & 0 & 6400 \\
3625 & LMQ & & & 46 & 0 & 598 & 191 & 46 & 0 & 598 \\
3631 & LMQ & & & 750 & 0 & 143 & 210 & 25 & 0 & 8 \\
3642 & QBN & 48.9 & 0.4 & 1035 & 0 & 0 & 0 & 0 & 0 & 0 \\
3643 & LGQ & & & 216 & 72 & 72 & 825 & 68 & 0 & 152 \\
3645 & LMQ & & & 101 & 0 & 302 & 304 & 1 & 1 & 1 \\
3646 & LMQ & & & 20 & 0 & 2000 & 1050 & 1000 & 0 & 2000 \\
3648 & LMQ & & & 40 & 0 & 680 & 306 & 40 & 0 & 80 \\
3650 & QBN & 48.8 & 0.4 & 946 & 0 & 0 & 0 & 0 & 0 & 0 \\
3651 & LMQ & & & 137 & 0 & 2139 & 2942 & 861 & 0 & 2136 \\
3659 & LGQ & & & 0 & 960 & 4577 & 5537 & 960 & 0 & 1474 \\
3661 & LMQ & & & 10816 & 0 & 12997 & 11024 & 3221 & 0 & 12906 \\
3662 & LMQ & & & 144 & 0 & 32 & 55 & 24 & 0 & 32 \\
3670 & LMQ & & & 54 & 0 & 864 & 305 & 54 & 0 & 108 \\
3676 & LMQ & & & 30 & 0 & 9000 & 4650 & 4500 & 0 & 9000 \\
3677 & LMQ & & & 30 & 0 & 6000 & 3100 & 3000 & 0 & 6000 \\
3678 & LMD & & & 200 & 0 & 400 & 402 & 0 & 1 & 0 \\
3680 & LMQ & & & 92 & 0 & 16 & 40 & 12 & 0 & 16 \\
3683 & LMQ & & & 126 & 0 & 24 & 48 & 18 & 0 & 24 \\
3690 & LMQ & & & 20 & 0 & 6000 & 3150 & 3000 & 0 & 6000 \\
3692 & LMQ & & & 128 & 0 & 1091 & 751 & 528 & 0 & 592 \\
3693 & QBN & 48.9 & 0.3 & 1128 & 0 & 0 & 0 & 0 & 0 & 0 \\
3694 & DML & 0.0 & 0.1 & 40 & 0 & 3200 & 3280 & 0 & 0 & 3200 \\
3697 & LMQ & & & 168 & 0 & 32 & 58 & 24 & 0 & 32 \\
3698 & DML & 0.0 & 0.1 & 30 & 0 & 3000 & 3100 & 0 & 0 & 3000 \\
3699 & LMQ & & & 116 & 0 & 792 & 1668 & 192 & 0 & 288 \\
3701 & LMQ & & & 60 & 0 & 1080 & 377 & 60 & 0 & 120 \\
3703 & QBL & 46.7 & 84.6 & 225 & 0 & 0 & 30 & 0 & 0 & 0 \\
3705 & QBN & 48.1 & 1.0 & 378 & 0 & 0 & 0 & 0 & 0 & 0 \\
3706 & QBN & 48.6 & 0.6 & 703 & 0 & 0 & 0 & 0 & 0 & 0 \\
3708 & DML & 0.0 & 0.1 & 14 & 0 & 12916 & 12917 & 0 & 0 & 1008 \\
3709 & QBL & 48.0 & 91.8 & 600 & 0 & 0 & 50 & 0 & 0 & 0 \\
3713 & LMQ & & & 42 & 0 & 630 & 254 & 42 & 0 & 84 \\
3714 & QBL & 97.5 & 32.5 & 120 & 0 & 0 & 40 & 0 & 0 & 0 \\
3719 & LMQ & & & 133 & 0 & 28 & 51 & 21 & 0 & 28 \\
3725 & LMQ & & & 81 & 0 & 1171 & 1552 & 469 & 0 & 1112 \\
3726 & LMQ & & & 116 & 0 & 816 & 2190 & 192 & 0 & 288 \\
3727 & LMQ & & & 20 & 0 & 1600 & 840 & 800 & 0 & 1600 \\
3728 & LMQ & & & 72 & 0 & 16 & 35 & 12 & 0 & 16 \\
3729 & LMQ & & & 650 & 0 & 408 & 51 & 608 & 0 & 408 \\
3733 & LMQ & & & 46 & 0 & 644 & 237 & 46 & 0 & 92 \\
3734 & LMQ & & & 38 & 0 & 7533 & 7690 & 2754 & 0 & 4050 \\
3738 & QBN & 48.3 & 0.9 & 435 & 0 & 0 & 0 & 0 & 0 & 0 \\
3745 & QBN & 48.0 & 1.2 & 325 & 0 & 0 & 0 & 0 & 0 & 0 \\
3748 & LMQ & & & 75 & 0 & 20 & 37 & 15 & 0 & 20 \\
3750 & QBL & 98.6 & 32.9 & 210 & 0 & 0 & 70 & 0 & 0 & 0 \\
3751 & QBL & 98.0 & 32.7 & 150 & 0 & 0 & 50 & 0 & 0 & 0 \\
3752 & QBL & 45.5 & 4.1 & 462 & 0 & 0 & 6160 & 0 & 0 & 0 \\
3757 & QBL & 34.4 & 1.7 & 552 & 0 & 0 & 8096 & 0 & 0 & 0 \\
3762 & QBL & 50.0 & 28.0 & 90 & 0 & 0 & 480 & 0 & 0 & 0 \\
3772 & QBL & 50.0 & 3.8 & 380 & 0 & 0 & 4560 & 0 & 0 & 0 \\
3775 & QBL & 98.3 & 32.8 & 180 & 0 & 0 & 60 & 0 & 0 & 0 \\
3780 & LIQ & & & 12 & 156 & 0 & 60 & 12 & 0 & 312 \\
3785 & LMQ & & & 200 & 0 & 32 & 62 & 24 & 0 & 32 \\
3790 & QML & 2.1 & 100.0 & 7 & 0 & 188 & 283 & 0 & 0 & 0 \\
3792 & DML & 0.0 & 0.1 & 20 & 0 & 3000 & 3150 & 0 & 0 & 3000 \\
3794 & LMQ & & & 576 & 0 & 986 & 624 & 602 & 0 & 968 \\
3797 & LMQ & & & 48 & 0 & 296 & 623 & 56 & 0 & 120 \\
3798 & LMQ & & & 54 & 0 & 810 & 251 & 54 & 0 & 810 \\
3803 & QBL & 42.6 & 14.1 & 190 & 0 & 0 & 2280 & 0 & 0 & 0 \\
3809 & LMQ & & & 224 & 0 & 32 & 65 & 24 & 0 & 32 \\
3813 & LMQ & & & 15 & 0 & 2400 & 1280 & 1200 & 0 & 2400 \\
3814 & QMQ & 4.2 & 16.0 & 2 & 0 & 46 & 13 & 28 & 0 & 80 \\
3815 & QBL & 50.0 & 3.1 & 192 & 0 & 0 & 64 & 0 & 0 & 0 \\
3816 & LMQ & & & 70 & 0 & 117 & 363 & 24 & 0 & 148 \\
3822 & QBN & 48.8 & 0.5 & 861 & 0 & 0 & 0 & 0 & 0 & 0 \\
3825 & LMQ & & & 60 & 0 & 1020 & 317 & 60 & 0 & 1020 \\
3832 & QBN & 48.5 & 0.7 & 561 & 0 & 0 & 0 & 0 & 0 & 0 \\
3834 & QBL & 60.0 & 98.0 & 50 & 0 & 0 & 1 & 0 & 0 & 0 \\
3838 & QBN & 48.7 & 0.5 & 780 & 0 & 0 & 0 & 0 & 0 & 0 \\
3840 & LMQ & & & 2401 & 0 & 3334 & 2499 & 1374 & 0 & 3292 \\
3841 & QBL & 44.0 & 10.2 & 300 & 0 & 0 & 4600 & 0 & 0 & 0 \\
3850 & QBN & 49.0 & 0.3 & 1225 & 0 & 0 & 0 & 0 & 0 & 0 \\
3852 & QBN & 47.6 & 1.6 & 231 & 0 & 0 & 0 & 0 & 0 & 0 \\
3854 & LMQ & & & 40 & 0 & 640 & 266 & 40 & 0 & 640 \\
3855 & LMQ & & & 400 & 0 & 2118 & 791 & 1284 & 0 & 428 \\
3856 & LMQ & & & 168 & 0 & 183 & 50 & 267 & 0 & 174 \\
3857 & LMQ & & & 201 & 0 & 602 & 604 & 1 & 1 & 1 \\
3859 & LMQ & & & 600 & 0 & 968 & 1225 & 560 & 0 & 392 \\
3860 & QBL & 44.8 & 8.7 & 435 & 0 & 0 & 8120 & 0 & 0 & 0 \\
3861 & DML & 0.0 & 0.1 & 30 & 0 & 4500 & 4650 & 0 & 0 & 4500 \\
3863 & LMQ & & & 625 & 0 & 1053 & 675 & 628 & 0 & 1033 \\
3865 & QBL & 48.0 & 90.7 & 525 & 0 & 0 & 50 & 0 & 0 & 0 \\
3870 & QML & 42.9 & 23.4 & 116 & 0 & 66 & 1456 & 0 & 0 & 132 \\
3871 & DML & 0.0 & 0.1 & 25 & 0 & 1000 & 1040 & 0 & 0 & 1000 \\
3872 & LMQ & & & 95 & 0 & 1413 & 1874 & 567 & 0 & 1368 \\
3877 & QBN & 48.6 & 0.6 & 630 & 0 & 0 & 0 & 0 & 0 & 0 \\
3879 & LMQ & & & 10920 & 0 & 12906 & 21945 & 3026 & 0 & 2090 \\
3883 & QBL & 50.0 & 17.8 & 182 & 0 & 0 & 1456 & 0 & 0 & 0 \\
3913 & CBL & 0.0 & 100.0 & 300 & 0 & 0 & 61 & 0 & 0 & 0 \\
3923 & QBL & 53.7 & 8.0 & 395 & 0 & 0 & 80 & 0 & 0 & 0 \\
3931 & QBL & 50.3 & 8.0 & 316 & 0 & 0 & 80 & 0 & 0 & 0 \\
3980 & CBL & 0.0 & 100.0 & 235 & 0 & 0 & 48 & 0 & 0 & 0 \\
4095 & CMQ & 0.0 & 100.0 & 400 & 0 & 1600 & 1603 & 400 & 0 & 400 \\
4270 & CML & 0.0 & 25.1 & 400 & 0 & 1200 & 1603 & 0 & 0 & 800 \\
4455 & LMQ & & & 3000 & 0 & 12000 & 9001 & 3000 & 0 & 3000 \\
4722 & LMQ & & & 2000 & 0 & 8000 & 6001 & 2000 & 0 & 2000 \\
4805 & LMQ & & & 2000 & 0 & 8000 & 6074 & 2000 & 0 & 4000 \\
5023 & LMQ & & & 3000 & 0 & 12000 & 9155 & 3000 & 0 & 6000 \\
5442 & LMQ & & & 2000 & 0 & 7999 & 6088 & 2000 & 0 & 3998 \\
5527 & DML & 0.0 & 0.1 & 4492 & 0 & 21117 & 64348 & 0 & 0 & 4738 \\
5543 & DML & 0.0 & 0.1 & 4514 & 0 & 21186 & 64096 & 0 & 0 & 4786 \\
5554 & LMQ & & & 4492 & 0 & 30878 & 64769 & 4800 & 0 & 4958 \\
5573 & LMQ & & & 4450 & 0 & 23692 & 72976 & 4800 & 0 & 4987 \\
5577 & DML & 0.0 & 0.1 & 1118 & 0 & 4896 & 15690 & 0 & 0 & 1186 \\
5721 & QBN & 49.0 & 76.8 & 300 & 0 & 0 & 0 & 0 & 0 & 0 \\
5725 & QBN & 50.1 & 1.7 & 343 & 0 & 0 & 0 & 0 & 0 & 0 \\
5755 & QBN & 50.0 & 1.0 & 400 & 0 & 0 & 0 & 0 & 0 & 0 \\
5875 & QBN & 50.0 & 78.9 & 200 & 0 & 0 & 0 & 0 & 0 & 0 \\
5881 & QBN & 49.2 & 29.5 & 120 & 0 & 0 & 0 & 0 & 0 & 0 \\
5882 & QBN & 49.3 & 78.1 & 150 & 0 & 0 & 0 & 0 & 0 & 0 \\
5909 & QBN & 50.0 & 9.6 & 250 & 0 & 0 & 0 & 0 & 0 & 0 \\
5922 & QBN & 49.8 & 9.8 & 500 & 0 & 0 & 0 & 0 & 0 & 0 \\
5924 & DML & 0.0 & 0.7 & 300 & 0 & 15220 & 36060 & 0 & 0 & 150 \\
5925 & LMQ & & & 100 & 0 & 1300 & 271 & 100 & 0 & 100 \\
5926 & LMQ & & & 2400 & 0 & 31200 & 11923 & 2400 & 0 & 2400 \\
5927 & LMQ & & & 2400 & 0 & 31200 & 11963 & 2400 & 0 & 2400 \\
5935 & QBL & 49.0 & 99.0 & 100 & 0 & 0 & 1237 & 0 & 0 & 0 \\
5944 & QBL & 49.0 & 99.0 & 100 & 0 & 0 & 2475 & 0 & 0 & 0 \\
5962 & QBL & 49.3 & 99.3 & 150 & 0 & 0 & 2793 & 0 & 0 & 0 \\
5971 & QBL & 49.3 & 99.3 & 150 & 0 & 0 & 5587 & 0 & 0 & 0 \\
5980 & QBL & 49.3 & 99.3 & 150 & 0 & 0 & 8381 & 0 & 0 & 0 \\
6287 & LCQ & & & 0 & 0 & 171 & 36 & 81 & 0 & 150 \\
6310 & LCQ & & & 0 & 0 & 208 & 22 & 390 & 0 & 324 \\
6311 & LCQ & & & 0 & 0 & 212 & 43 & 128 & 0 & 186 \\
6324 & QBL & 50.6 & 31.3 & 640 & 0 & 0 & 16 & 0 & 0 & 0 \\
6487 & QBL & 35.0 & 20.9 & 618 & 0 & 0 & 309 & 0 & 0 & 0 \\
6597 & QBL & 45.7 & 97.3 & 600 & 0 & 0 & 60 & 0 & 0 & 0 \\
6647 & QBL & 70.0 & 7.2 & 627 & 0 & 0 & 33 & 0 & 0 & 0 \\
6757 & QBL & 18.5 & 4.7 & 2046 & 0 & 0 & 297 & 0 & 0 & 0 \\
6764 & QBL & 19.1 & 4.7 & 2071 & 0 & 0 & 297 & 0 & 0 & 0 \\
6799 & QBL & 18.7 & 4.7 & 2075 & 0 & 0 & 297 & 0 & 0 & 0 \\
6941 & QBL & 18.7 & 4.5 & 2203 & 0 & 0 & 315 & 0 & 0 & 0 \\
7127 & QBL & 50.6 & 6.8 & 1000 & 0 & 0 & 50 & 0 & 0 & 0 \\
7139 & QBL & 53.3 & 89.2 & 180 & 0 & 0 & 100 & 0 & 0 & 0 \\
7144 & QBL & 53.2 & 89.6 & 220 & 0 & 0 & 121 & 0 & 0 & 0 \\
7149 & QBL & 53.0 & 89.6 & 264 & 0 & 0 & 144 & 0 & 0 & 0 \\
7154 & QBL & 52.9 & 89.7 & 312 & 0 & 0 & 169 & 0 & 0 & 0 \\
7159 & QBL & 52.5 & 89.7 & 364 & 0 & 0 & 196 & 0 & 0 & 0 \\
7164 & QBL & 52.4 & 89.7 & 420 & 0 & 0 & 225 & 0 & 0 & 0 \\
7579 & LMD & & & 100 & 0 & 200 & 202 & 0 & 1 & 0 \\
8009 & LMQ & & & 101 & 0 & 303 & 305 & 2 & 0 & 2 \\
8153 & LMQ & & & 31 & 0 & 93 & 95 & 2 & 0 & 2 \\
8381 & LMQ & & & 51 & 0 & 153 & 155 & 2 & 0 & 2 \\
8495 & DCL & 0.0 & 0.1 & 0 & 0 & 27543 & 8000 & 0 & 0 & 22743 \\
8505 & QCL & 49.9 & 0.1 & 0 & 0 & 20050 & 10001 & 0 & 0 & 40100 \\
8515 & CCL & 0.0 & 0.1 & 0 & 0 & 16002 & 8002 & 0 & 0 & 16002 \\
8559 & CCL & 0.0 & 0.1 & 0 & 0 & 10000 & 5000 & 0 & 0 & 20000 \\
8567 & CCL & 0.0 & 0.1 & 0 & 0 & 10000 & 7500 & 0 & 0 & 20000 \\
8602 & DCL & 0.0 & 0.1 & 0 & 0 & 34552 & 52983 & 0 & 0 & 69104 \\
8605 & DCQ & 0.0 & 0.1 & 0 & 0 & 5000 & 0 & 1 & 0 & 1 \\
8616 & DCL & 0.0 & 0.1 & 0 & 0 & 13870 & 10404 & 0 & 0 & 409 \\
8685 & DCQ & 0.0 & 0.1 & 0 & 0 & 772 & 0 & 10000 & 0 & 0 \\
8777 & QCL & 34.6 & 0.1 & 0 & 0 & 10000 & 2500 & 0 & 0 & 20000 \\
8785 & DCL & 0.0 & 0.1 & 0 & 0 & 10399 & 11362 & 0 & 0 & 20798 \\
8790 & CCB & 0.0 & 0.1 & 0 & 0 & 39204 & 0 & 0 & 0 & 39204 \\
8792 & CCB & 0.0 & 0.1 & 0 & 0 & 15129 & 0 & 0 & 0 & 30258 \\
8845 & CCL & 0.0 & 59.8 & 0 & 0 & 1546 & 777 & 0 & 0 & 441 \\
8906 & CCL & 0.0 & 3.0 & 0 & 0 & 5223 & 838 & 0 & 0 & 1941 \\
8938 & DCL & 0.0 & 0.1 & 0 & 0 & 4001 & 11999 & 0 & 0 & 0 \\
8991 & CCB & 0.0 & 0.1 & 0 & 0 & 14400 & 0 & 0 & 0 & 28800 \\
9002 & DCL & 0.0 & 0.1 & 0 & 0 & 2890 & 1649 & 0 & 0 & 3617 \\
9004 & QCQ & 25.0 & 0.1 & 0 & 0 & 40000 & 10001 & 10001 & 0 & 20000 \\
9030 & CIL & 0.0 & 0.1 & 0 & 10000 & 0 & 5000 & 0 & 0 & 20000 \\
9048 & QIL & 29.7 & 18.2 & 0 & 202 & 0 & 1 & 0 & 0 & 404 \\


\bottomrule
\label{tab:A1}
\end{longtable}
}




%- - - - - - - - - - - - - - - - - - - - - - - - - - - - - - - - - - - -
%- - - - - - - - - - - - - - - - - - - - - - - - - - - - - - - - - - - -
%  End Appendix.tex
%- - - - - - - - - - - - - - - - - - - - - - - - - - - - - - - - - - - -
%- - - - - - - - - - - - - - - - - - - - - - - - - - - - - - - - - - - -
