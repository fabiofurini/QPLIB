\section{Final Remarks}\label{sec:conclusions}

This paper described the first comprehensive library of instances for Quadratic Programming (QP). Since QP comprises different and ``varied'' categories of problems, we proposed a classification and we briefly discussed the main classes of solution methods for QP.
We then described the steps of the adopted process used to filter the gathered instances  in order to build the new library. Our design goals were to build a library which is computationally challenging and as broad as possible, i.e., it represents the largest possible categories of QP problems, while remaining of manageable size. We also proposed a stand-alone QP format that is intended
for the convenient exchange and use of our QP instances.

We want to stress once again that we intentionally avoided to perform a
computational comparison of the performances of different solution methods or
solver implementations. Our goal was instead to provide a broad test bed of
instances for researchers and practitioners in the field. This new library will
hopefully serve as a point of reference to inspire and test new ideas and
algorithms for QP problems.

Finally, we want to emphasize that this QP collection can only be a snapshot of
the types of problems that researchers and practitioners have worked on in the
past.  With the growing interest in this area, we hope that new applications and
instances will become available and that the library can be extended dynamically
in the future.
