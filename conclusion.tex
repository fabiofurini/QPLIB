\section{Conclusions}\label{sec:conclusions}

\todo [inline]{AG: We do not draw many conclusions here, maybe call it ``Final Remarks''?}

This manuscript\todo{AG: paper/article? Manuscript usually refers to pre-final versions I think} described the first comprehensive library of instances for Quadratic Programming (QP). Since QP comprises  different and ``varied'' categories of problems, we proposed a classification and we briefly discussed the main classes of solution methods for QP.
We then described the steps of the adopted process used to filter the gathered instances  in order to build the new library. Our design goals were to build a library which is computationally challenging and as broad as possible, i.e., it represents the largest possible categories of QP problems, while remaining of manageable size. We also proposed a stand-alone QP format that is intended
for the convenient exchange and use of our QP instances.


We want to stress once again that we intentionally avoided to perform a computational comparison of the performances of different solvers\todo{AG: maybe ``solution methods'' instead? Since not all algorithms described are actually available as what I would call a solver.}. Our goal was instead to provide a common test bed of instances for practitioners and researchers in the field. This new library will hopefully serve as a point of reference to inspire and test new ideas and algorithms for QP problems.

\todo[inline]{AG: Maybe as last point add that it is supposed to be a
  dynamically growing library that will be updated as new instance types become
  available, especially from currently under- or not-represented classes.}
