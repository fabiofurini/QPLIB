%- - - - - - - - - - - - - - - - - - - - - - - - - - - - - - - - - - - -
%- - - - - - - - - - - - - - - - - - - - - - - - - - - - - - - - - - - -
%  QPLIB-2.4.tex
%- - - - - - - - - - - - - - - - - - - - - - - - - - - - - - - - - - - -
%- - - - - - - - - - - - - - - - - - - - - - - - - - - - - - - - - - - -
\subsection{Solvers}\label{subsec:solver}
\todo{\small Ambros: merge with Section~\ref{s:complete} or put list of solvers in Section~\ref{s:complete} here}

We now provide a succinct list of the solvers we have tested, using the approaches described in \S \ref{sec:algo}. In Table \ref{t:solvers}, we mark with ``I'' a pair (solver, problem) if the solver accepts the problem in input but it is an incomplete solver for the problem, with ``A'' if it is asymptotically complete, with ``C'' if it is complete, and leave it blank if the solver won't accept the problem in input.

\begin{table}
{
 \centering                
 \scriptsize                
 \setlength{\tabcolsep}{3pt}                
% \renewcommand \arraystretch{1}                
                
\begin{tabular}{lccccccc}
\toprule  
                 & CGL & QGL & CGC & QGQ & CCC & QCQ \\
\hline
{\sc Antigone}    &  C  &  C  &  C  &  C  &  C  &  C  \\
{\sc Baron}       &  C  &  C  &  C  &  C  &  C  &  C  \\
{\sc Couenne}     &  C  &  C  &  C  &  C  &  C  &  C  \\
{\sc KNitro}      &  C  &  I  &  C  &  I  &  C  &  A  \\
{\sc lindo api}   &  C  &  C  &  C  &  C  &  C  &  C  \\
{\sc SCIP}        &  C  &  C  &  C  &  C  &  C  &  C  \\
{\sc oqnlp}       &  A  &  A  &  A  &  A  &  C  &  A  \\
{\sc AlphaECP}    &  C  &  I  &  C  &  I  &  C  &  I  \\
{\sc BonMin}      &  C  &  I  &  C  &  I  &  C  &  I  \\
{\sc DICOpt}      &  C  &  I  &  C  &  I  &  C  &  I  \\
{\sc sBB}         &  C  &  I  &  C  &  I  &  C  &  I  \\
{\sc ConOpt}      &     &     &     &     &  C  &  I  \\
{\sc IpOpt}       &     &     &     &     &  C  &  I  \\
{\sc lgo}         &     &     &     &     &  A  &  A  \\
{\sc Minos}       &     &     &     &     &  C  &  I  \\
{\sc msnlp}       &     &     &     &     &  C  &  A  \\
{\sc SnOpt}       &     &     &     &     &  C  &  I  \\
{\sc XPress}      &  C  &     &  C  &     &  C  &     \\
{\sc Gurobi}      &  C  &     &  C  &     &  C  &     \\
{\sc CPLEX}       &  C  &  C  &  C  &     &  C  &     \\
{\sc mosek}       &  C  &     &  C  &     &  C  &     \\
\hline
\end{tabular}                 
\caption{Families of QP problems that can be tackled by each solver} \label{t:solvers}
}

\todo[inline]{Xpress has a solver for convex MIQCPs called Xpress-SLP.
However, it is not available within GAMS.
Should Table 2 take Xpress-SLP into account?}

\end{table}              

%- - - - - - - - - - - - - - - - - - - - - - - - - - - - - - - - - - - -
%- - - - - - - - - - - - - - - - - - - - - - - - - - - - - - - - - - - -
%  End QPLIB-2.4.tex
%- - - - - - - - - - - - - - - - - - - - - - - - - - - - - - - - - - - -
%- - - - - - - - - - - - - - - - - - - - - - - - - - - - - - - - - - - -
