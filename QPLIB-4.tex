%- - - - - - - - - - - - - - - - - - - - - - - - - - - - - - - - - - - -
%- - - - - - - - - - - - - - - - - - - - - - - - - - - - - - - - - - - -
%  QPLIB-4.tex
%- - - - - - - - - - - - - - - - - - - - - - - - - - - - - - - - - - - -
%- - - - - - - - - - - - - - - - - - - - - - - - - - - - - - - - - - - -

\section{Software tools}\label{subsec:tools}

%- - - - - - - - - - - - - - - - - - - - - - - - - - - - - - - - - - - -
\subsection{instance translator}
GAMS--LP--QPFORMAT
\todo[inline]{TASK X : write }


%- - - - - - - - - - - - - - - - - - - - - - - - - - - - - - - - - - - -
\subsection{code that computes the features of an instance}
\todo[inline]{TASK X : write }

%- - - - - - - - - - - - - - - - - - - - - - - - - - - - - - - - - - - -
\subsection{code that selects subsets of instances}
\todo[inline]{TASK X : write }

%- - - - - - - - - - - - - - - - - - - - - - - - - - - - - - - - - - - -
\subsection{website, instance collector}
Select subset or categories of instances (EXTRACT FROM THE LIBRARY A SUBSET OF INSTANCES WITH SPECIFIC CHARACTERISTICS)
\todo[inline]{TASK X : write }

The instances of QPLIB are publically accessible at the website \url{http://qplib.zib.de}.
%
Beyond links to the \texttt{gms} and \texttt{lp} format files, the website provides a rich set of metadata for each
instance: the three letter problem classification, basic properties such as the number of variables and constraints of
different types, the sense and curvature of the objective function, and information on the nonzero structure of the
problem.
%
In addition, we display a visualization of the sparsity patterns of the Jacobian and the Hessian matrix of the
Lagrangian function.  In the plots of the Jacobian nonzero pattern, the linear and nonlinear entries are distinguished
by color.  Figure... shows an example for instance QPLIB...


The entire set of instances can be explored in a searchable and sortable table of selected instance features: problem
classification, convexity of the continuous relaxation, and number of variables, constraints, and nonzeros.
%
Finally, a statistics page displays diagrams on the composition of the library according to different criteria: the
number instances according to problem type, variable types, convexity, number of variables and constraints, and density.
%
A table containing the relevant metadata for each instance can be downloaded in \texttt{csv} format and as spreadsheets
such that researchers can easily compile their own subset of instances according to these statistics.


The complete library can be downloaded as one archive, which contains the website for offline browsing and exploration.  
%
In the future, we plan to extend the website by the addition of contributor information and references to the literature.


%- - - - - - - - - - - - - - - - - - - - - - - - - - - - - - - - - - - -
\subsection{testing environment}
RUN GAMS USING A SUBSET OF SOLVERS
\todo[inline]{TASK X : write }

%- - - - - - - - - - - - - - - - - - - - - - - - - - - - - - - - - - - -
%- - - - - - - - - - - - - - - - - - - - - - - - - - - - - - - - - - - -
%  End QPLIB-4.tex
%- - - - - - - - - - - - - - - - - - - - - - - - - - - - - - - - - - - -
%- - - - - - - - - - - - - - - - - - - - - - - - - - - - - - - - - - - -
